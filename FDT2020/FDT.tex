\documentclass[11pt,notitlepage]{article}

\newcommand{\cindy}[1]{\textcolor{green!60!black}{(Cindy: #1)}}
\newcommand{\arash}[1]{\textcolor{blue}{(Arash: #1)}}
\newcommand{\bob}[1]{\textcolor{red}{(Bob: #1)}}
% To drop all comments, just comment out \iffalse and \fi

\renewcommand{\cindy}[1]{}
\renewcommand{\arash}[1]{}
\renewcommand{\bob}[1]{}


\usepackage{thumbpdf, amssymb, amsmath, amsthm, microtype, array,
graphicx, verbatim, listings, color, fancybox}
\usepackage[pdftex]{hyperref}
\usepackage{mathtools}
\usepackage{amsmath, nccmath}
\usepackage{booktabs}
\usepackage{glossaries}
\usepackage{subcaption}
	
\usepackage{setspace}
\usepackage{acro}
\usepackage{url}
\usepackage[linesnumbered,ruled]{algorithm2e}
\usepackage{subcaption}
\usepackage[utf8]{inputenc}
\usepackage[english]{babel}
\usepackage{caption}
\usepackage{subcaption}
\usepackage{thmtools}
\usepackage{bbm}
\usepackage{pgf,tikz}
\usetikzlibrary{arrows}
\pagestyle{plain}
\usepackage[margin=1.25in]{geometry}
\usepackage{multicol,multirow}

\usepackage{mathrsfs}
\usepackage{hyperref}
\usepackage{epigraph}
\usepackage{cleveref}

\newcommand{\ee}{e'}

\newcommand{\oc}{\mathrm{oc}}
\newcommand{\mcc}{\mathcal{C}}
\newcommand{\NP}{\mathrm{NP}}
\newcommand{\Pp}{\mathrm{P}}
\newcommand{\UGC}{\mathrm{UGC}}
\newcommand{\W}{\mathrm{W}}	
\newcommand{\FPT}{\mathrm{FPT}}
\newcommand{\opt}{\mathrm{OPT}}
\newcommand\floor[1]{\lfloor#1\rfloor}
\newcommand\ceil[1]{\lceil#1\rceil}
\newcommand{\FF}{\mathbb{F}}
\newcommand{\AF}{\mathbb{A}}
\newcommand{\BF}{\mathbb{A}}
\newcommand{\CF}{\mathbb{C}}
\newcommand{\EC}{\mathrm{2EC}}
\newcommand{\HK}{\mathrm{LP}}
\newcommand{\con}{\mathrm{Connector}}
\newcommand{\cov}{\mathrm{cov}}
\newcommand{\point}{\mathrm{end}}

\newcommand{\gap}{\alpha^{\HK}_{\EC}}
\newcommand{\Lu}{\mathcal{L}_\ell} 
\newcommand{\Rv}{\mathcal{R}_\ell} 

\newcommand{\Lup}{\mathcal{L}_{\ell'}} 
\newcommand{\Rvp}{\mathcal{R}_{\ell'}} 

\newcommand{\Ex}{\mathop{\mathbb{E}}}

\setlength{\tabcolsep}{0.5em} % for the horizontal padding
{\renewcommand{\arraystretch}{1.3}% for the vertical padding
	
\declaretheorem[name=Theorem]{thm}
\declaretheoremstyle[style=claim,qed=$\Diamond$]{claim}
\declaretheoremstyle[style=plain,qed=$\square$]{theorem}

\theoremstyle{plain}
\newtheorem{theorem}{Theorem}
\newtheorem{corollary}[thm]{Corollary}
\newtheorem{lemma}[thm]{Lemma}
\newtheorem{conjecture}{Conjecture}
\newtheorem{problem}{Open Problem}
\newtheorem{fact}[thm]{Fact}
\newtheorem{observation}[thm]{Observation}
\newtheorem{proposition}[thm]{Proposition}
\newtheorem{definition}[thm]{Definition}
\newtheorem{claim}{Claim}
\newtheorem*{notation}{Notation}
\newtheorem*{remark}{Remark}
\newtheorem{example}[thm]{Example}%[section]

%%%%%%%%%%%%%


%%%%%%%%%%%%
\DeclareMathOperator{\chord}{chord}
\DeclareMathOperator{\spp}{supp}
\DeclareMathOperator{\CC}{CC}
\DeclareMathOperator{\degree}{deg}
\DeclareMathOperator{\join}{-\;JOIN}
\DeclareMathOperator{\oddcut}{ODD-CUT}
\DeclareMathOperator{\cover}{COVER}
\DeclareMathOperator{\lp}{LP}

\DeclareMathOperator{\st}{ST}
\DeclareMathOperator{\connector}{SS}
\DeclareMathOperator{\LP}{LP}
\DeclareMathOperator{\IP}{IP}
\DeclareMathOperator{\dom}{\mathcal{D}}
\DeclareMathOperator{\conv}{conv}
\DeclareMathOperator{\subt}{{SEP}}
\DeclareMathOperator{\subtour}{{Subtour}}
\DeclareMathOperator{\hamil}{{{Hamilton}}}
\DeclareMathOperator{\tsp}{{{TSP}}}
\DeclareMathOperator{\graphtsp}{{{Graph-TSP}}}
\DeclareMathOperator{\nwtsp}{{{NW-TSP}}}
\DeclareMathOperator{\2ec}{{{2EC}}}
\DeclareMathOperator{\ecs}{{{2ECS}}}
\DeclareMathOperator{\DOMtoS}{{{DomToIP}}}
\DeclareMathOperator{\LPC}{{{LPC}}}
\DeclareMathOperator{\prun}{{{Pruning}}}


\newcommand{\vtree}{\mathrm{\textsc{-tree}}}

\newcommand{\TT}{\mathcal{T}}
\newcommand{\JJ}{\mathcal{J}}
\DeclareMathOperator{\ecss}{{{S2ECS}}}


\DeclareMathOperator{\tap}{{{TAP}}}
\DeclareMathOperator{\cut}{{{CUT}}}
\DeclareMathOperator{\permat}{{{PM}}}
\DeclareMathOperator{\nwec}{{{NW-2ECM}}}
\DeclareMathOperator{\nwecs}{{{NW-2ECS}}}
\newenvironment{cproof}
{\begin{proof}
 [Proof.]
 \vspace{-1.5\parsep}%-3.2\parsep} %%%For use with US letter
}
{\renewcommand{\qed}{\hfill $\Diamond$} \end{proof}}

\onehalfspacing

\title{Fractional Decomposition Tree Algorithm: A tool for studying the integrality gap of Integer Programs}


\author{\textsc{Robert Carr}\thanks{University of New Mexico  {\tt{bobcarr@unm.edu}}. This material is based upon research supported in part by
the U.S. Office of Naval Research under award number N00014-18-1-2099.}
\and \textsc{Arash Haddadan}\thanks{Carnegie Mellon University  {\tt{ahaddada@andrew.cmu.edu}}.}
\and \textsc{Cynthia A. Phillips}\thanks{Sandia National Laboratories  {\tt{caphill@sandia.gov}}. Sandia National Laboratories is a multi-mission laboratory managed and operated by National Technology and Engineering Solutions of Sandia, LLC., a wholly owned subsidiary of Honeywell International, Inc., for the U.S. Department of Energy’s National Nuclear Security Administration under contract DE-NA0003525.}}


\begin{document}{\bibliographystyle{alpha}}



\maketitle

\begin{abstract}
We present a new algorithm, Fractional Decomposition Tree (FDT) for finding a feasible solution for an integer program (IP) where all variables are binary. FDT runs in polynomial time and is guaranteed to find a feasible integer solution provided the integrality gap is bounded. The algorithm gives a construction for Carr and Vempala's theorem that any feasible solution to the IP's linear-programming relaxation, when scaled by the instance integrality gap, dominates a convex combination of feasible solutions. FDT is also a tool for studying the integrality gap of IP formulations.  We demonstrate that with experiments studying the integrality gap of two problems: optimally augmenting a tree to a 2-edge-connected graph and finding a minimum-cost 2-edge-connected multi-subgraph (2EC). We also give a simplified algorithm, Dom2IP, that more quickly determines if an instance has an unbounded integrality gap. We show that FDT's speed and approximation quality compare well to that of feasibility pump on moderate-sized instances of the vertex cover problem. For a particular set of hard-to-decompose fractional 2EC solutions, FDT always gave a better integer solution than the best previous approximation algorithm (Christofides).
\end{abstract}


\nocite{IPbook}

%\cindy{An original comment:
%There's material in this section (e.g. general discussion of combintatorial optimization) that's likely unnecessary for the journal paper. We probably want to start with what is now Section 2, but have some opening introduction giving the big picture.
%Remember when we first mention rounding to be clear this is not numerical rounding. TODO: Add English statement for things currently only given in math. Add definitions of problems so reviewers and readers don't have to look them up.}
\iffalse{
\section{Introduction}\label{chapter:intro}

In combinatorial optimization the aim is to find the optimal solution in a discrete and
usually finite yet large set of solutions. For many specific combinatorial optimization problems such a solution can be found efficiently. For many others, finding optimal or in many cases near optimal solutions is NP-hard. A common approach to deal with such problems is relaxing the discrete solution set into a continuous set, where the optimization problem becomes tractable. Obtaining feasible solutions by means of such a relaxation requires an additional step of rounding the potentially fractional solution of the continuous relaxation into integer solutions.
}\fi

\section{Introduction}
In this paper we focus on finding feasible solutions to binary Integer Linear Programs (IP). Informally, an integer program is the optimization of a linear objective function subject to linear constraints, where the variables must take integer values. Binary variables represent yes/no decisions. Integer Programming (and more generally Mixed Integer Linear Programming (MILP)) can model many practical optimization problems including scheduling, logistics and resource allocation. It is  NP-hard even to determine if an IP instance has a feasible solution~\cite{GareyJohnson}. Given its practical importance, however, there are many commercial (e.g. CPLEX, GUROBI, XPRESS) and free (e.g. CBC) solvers that for specific IP instances can often find solutions that are optimal within a given tolerance. Still we have formulated moderate-sized IP instances that no commercial solver can currently solve. Thus there is value in heuristics to find feasible solutions for general IP instances (see e.g.~\cite{HanafiT2017}).

\iffalse
However, there is substantial research into finding a feasible, provably-good approximate, and even (computationally) provably optimal solutions to specific IP instances. 
There is substantial research into finding a feasible, provably-good approximate solution for TODO, and even (computationally) provably optimal solutions to specific IP instances. \cindy{TODO: probably need to informally distinguish between a IP for a general problem, and a specific instance when the parameters are set.}
\fi

A major tool for finding feasible solutions for discrete optimization problems expressed as IPs is the {\em linear-programming (LP) relaxation} for the IP formulation. This is a new problem created by relaxing the integrality constraints for an IP instance, allowing the variables to take continuous (rational) values. Linear programs can be solved in polynomial time. The objective value of the linear programming relaxation provides a bound (lower bound for a minimization problem and upper bound for a maximization problem) on the optimal solution to the IP instance. The solutions can also provide some useful global structure, even though the fractional values might not be directly meaningful. 

{\em LP-based approximation algorithms} use LP relaxations to find provably good approximate feasible solutions to IP problems in polynomial time. At the highest level, they involve solving the LP relaxation, using special structure from the problem to find a feasible solution, and proving that the objective value of the solution is no more than $C$ times worse than the bound from the LP relaxation. The approximation factor $C$ can be a constant or depend on the input parameters of the IP, e.g. $O(\log(n))$ where $n$ is the number of variables in the formulation of the IP (the dimension of the problem).

There is an inherent limit to how small $C$ can be for a given IP. The integrality gap for an IP instance is the ratio of the best integer solution to the best solution of the LP relaxation. Any LP-based approximation cannot have an approximation factor $C$ smaller than the integrality gap because there is no feasible solution with an objective value better than a factor of $C$ worse than the optimal solution of the LP relaxation. 

If the integrality gap for an IP formulation is large, it is sometimes possible to add families of constraints to the formulation to reduce the integrality gap.  These constraints are redundant for the integer problem, but can make some fractional solutions no longer feasible for the LP. These families of constraints (cuts) can have exponential size (number of constraints) as long as we can provide a polynomial-time separation algorithm. A separation algorithm for a family of constraints takes an optimal solution to an LP instance that explicitly enforces only a (potentially empty) subset of the family. It either confirms that all constraints are satisfied or returns a most violated constraint. Thus one can add this constraint and repeat at most a polynomial number of times until all are satisfied.

Reducing the integrality gap of an IP formulation has two advantages.  It can lead to better LP-based approximation algorithm bounds as described above.  It can also help exact solvers run faster or solve instances it could not before. Exact IP solvers are based on intelligent branch-and-bound strategies.  As mentioned above, commercial and open-source MILP solvers can find exact solutions (or near-optimal solutions with a provable bound) to many specific instances of NP-hard combinatorial optimization problems. These solvers use the LP relaxation to get lower bounds (for minimization problems).  The search requires exponential time in the worst case. But this search is practically feasible when the solver can prune large amounts of the search space.  This happens when the lower bound for a region of the search space (subproblem) is worse than the value of a known feasible solution.  This requires a way to find a good heuristic solution and it requires good lower bounds that are as close to the actual optimal value of an IP subproblem as possible.

In this paper, we give a method to find feasible solutions for IPs if the integrality gap is bounded. The method is also a tool for evaluating the integrality gap for a formulation.  Researcher can use it to determine whether they should expend effort to find new classes of cuts.  They can also use it to help guide theory for finding tighter bounds on the integrality gap for classic problems like the traveling salesman problem.

% A common approach to Integer programming is considering their linear relaxation.
% \cindy{TODO: this has to move to a logical place in the more formal treatment below:
%For some problems such as the \textsc{Minimum Cost Spanning Tree Problem} there are linear programming relaxations whose basic feasible solutions coincide with  integral solutions, i.e. spanning trees.}

%A common and successful approach is to round these (potentially) fractional solutions into integer solutions for the optimization problem at hand. The Integrality gap of a linear relaxation of an integer programming problem is the worst case ratio between the objective values of the discrete problem and the continuous problem. Equivalently, the integrality gap of the linear programming relaxation is a limit to the rounding approach: rounding a fractional solution into an integer solution incurs a multiplicative cost proportional to the integrality gap. In this paper we study integrality gaps for different combinatorial optimization problems and introduce new rounding algorithms that imply bounds on their respective integrality gaps. 

\iffalse{

Let $S$ denote the set of feasible solutions to a combinatorial optimization problem. For instance, for many problems in network optimization, set $S$ is a subset of $ \{0,1\}^n$ where each coordinate of a point in $S$ indicates the absence or presence of the corresponding edge in a solution, and $n$ is the number of edges in the network. Suppose set $S$ can be described as $S=\{x\in \mathbb{Z}^n: Ax\geq b, x\geq 0 \}$ for some $A\in \mathbb{R}^{m\times n}$ and $b\in \mathbb{R}^m$. (Pure) Integer Programming (IP) asks for $\min_{x\in S}cx$ for some $c\in \mathbb{R}^n$. Integer programming is NP-hard and in fact, it is even NP-complete to decide whether set $S$ is empty or not \cite{GareyJohnson}. The convex hull of $S$ denoted by $\conv(S)$ is the minimal convex set containing $S$ and can be formulated as follows.
\begin{equation*}
\conv(S) =\{\sum_{i=1}^{k} \lambda_i x^i: x^i \in S \text{ for } i =1,\ldots,k,\; \lambda_i \geq 0 \text{ for } i = 1,\ldots,k \text{, and } \sum_{i=1}^{k}\lambda_i = 1\}.
\end{equation*}
\cindy{Since most canonical formulations for IP/LP express variables bounds separately, we should mention here that they are rolled into matrix $A$.}

A fundamental fact in polyhedral theory is that $\min_{c\in S} S = \min_{c\in S} \conv(S)$. Notice that $\conv(S)$ is a polyhedron and optimizing a linear function subject to the points lying in a polyhedron can be done in polynomial time in the number of variables and constraints in the description of $\conv(S)$. Such a description, however, might have exponential size in the description of set $S$.

A natural way to bound the solution to the integer program $\min_{x\in S} cx$ is to relax the integrality constraints. Let $L= \{x\in \mathbb{R}^n: Ax\geq b, x\geq 0\}$. Contrary to integer programming, the optimal solution to $\min_{x\in L}cx$ can be efficiently found. Set $L$ is called the linear programming relaxation of $S$. Since we relaxed the integrality requirement on $x$, we have	
\begin{equation}\label{lp-lower-bound}
\min_{x\in L} cx \leq \min_{x\in S}cx.
\end{equation}
For most relevant applications and for the entirety of this dissertation we assume $c$ is a non-negative vector and  $c\neq 0$, i.e. $c$ has a positive value in at least one coordinate. Following this assumption we can rewrite (\ref{lp-lower-bound}) as 
\begin{equation}
\frac{\min_{x\in S}cx}{\min_{x\in L} cx}\geq 1.
\end{equation}
Since we are concerned with the worst-case analysis, we consider 

\vspace*{5pt}
\noindent\fbox{%
	\parbox{\textwidth}{%
		\vspace*{3pt}
		\begin{equation}\label{eq:IG}
		g=\max_{c\in \mathbb{R}^n_{\geq 0}}\frac{\min_{x\in S}cx}{\min_{x\in L} cx}.
		\end{equation}
	}%
}
\vspace*{3pt}

If $g=1$, we say that the linear programming formulation is a perfect formulation. Otherwise we have $g>1$. In this case, we cannot hope to achieve an integer solution with cost lower than $(g-\epsilon)\cdot (\min_{x\in L}cx)$, for any constant $\epsilon>0$. Thus, a lower bound on $g$ provide a certificate for impossibility of approximation via the linear relaxation for which the gap is $g$. On the other hand, an upper bound of $\alpha$ for $g$ is often accompanied with an $\alpha$-approximation algorithm. This is not always the case, as we will later discuss in details.

We refer to $g$ as the integrality gap of the linear relaxation. For a polyhedron $P\in \mathbb{R}^n$ let the \textit{dominant of $P$} be $\{x\in \mathbb{R}^n: \exists y \in P: x\geq y\}$ and denote it by $\dom(P)$. Goemans \cite{goemansblocking} gave a characterization of integrality gap based on convex combinations when $\conv(S)=\mathcal{D}(\conv(S))$. Carr and Vempala \cite{Carr2004} generalized this characterization.


\vspace*{5pt}
\noindent\fbox{%
	\parbox{\textwidth}{%
		\begin{thm}[\cite{Carr2004}]\label{CV}
			Let $S=\{x\in \mathbb{Z}^n: Ax\geq 0,x\geq 0\}$, and $L= \{x\in \mathbb{R}^n: Ax\geq 0, x\geq 0\}$ be the linear relaxation of $S$. Then
			\begin{equation*}
			\max_{c\in \mathbb{R}^n_{\geq 0}}\frac{\min_{x\in S}cx}{\min_{x\in L} cx}= \min \{\alpha: \alpha\cdot x \in \mathcal{D}(\conv(S)) \text{ for all $x\in L$}\}.
			\end{equation*}
		\end{thm}
	}%
}
\vspace*{5pt}




A polynomial time algorithm for proving an upper bound on integrality gap is called an LP-based approximation algorithm. For many well studied problems, we still do not know the exact integrality gap and the gap between the best known lower bound and the upper bound on the integrality gap are open. In some cases, there are known upper bounds, yet there is no known approximation algorithm, meaning that the proofs do not yield polynomial time algorithms.
}\fi 
We now describe IPs and our methods more formally. The set of feasible points for a pure IP (henceforth IP) is the set
\begin{equation}
S(A,b)= \{x\in \mathbb{Z}^{n}\;:\; Ax\geq b\}  \label{S},
\end{equation}
where matrix $A$ of rationals has $m$ constraints on $n$ variables and $b \in \mathbb{R}^{m}$.
If we drop the integrality constraints, we have the linear relaxation of set $S(A,b)$,
\begin{equation}
P(A,b) = \{x\in \mathbb{R}^{n}\;:\; Ax\geq b\}. \label{P}
\end{equation}
Let $I=(A,b)$ denote an instance. Then $S(I)$ and $P(I)$ denote $S(A,b)$ and $P(A,b)$, respectively. Given a linear objective function $c$, an IP is $\min \;\{cx:\; x \in S(I)\}$. 

Relaxing the integrality constraints gives the polynomial-time-solvable linear programming relaxation: $\min \;\{cx:\;x\in P(I) \}$.  The optimal value of this linear program (LP), denoted $z_{\lp}(I,c)$, is a lower bound on the optimal value for the IP, denoted $z_{\IP}(I,c)$. 

Many researchers (see \cite{davids,vazirani}) have developed polynomial time LP-based approximation algorithms that find solutions for special classes of IPs whose cost are provably smaller than $C\cdot z_{LP}(I,c)$. If the analysis uses the LP bound to prove the approximation quality, then $C$ is at least the integrality gap.
%However, for many combinatorial optimization problems there is a limit to such techniques based on LP relaxations, represented  by the {integrality gap} of the IP formulation.
\begin{definition}
The integrality gap $g(I)$ for instance $I$ is: $$g(I)= \max_{c\geq 0}\frac{z_{IP}(I,c)}{z_{LP}(I,c)},$$
where $z_{IP}(I,c)$ is the optimal solution to the integer program and $z_{LP}(I,c)$ is the solution to the linear-programming relaxation.
\end{definition}
\cindy{I'm not sure I got an answer for this, so I'll just make it clear this is the definition we use: Does the integrality gap always require positive objective coefficients?  Or is that an extra restriction in our definition?  Is there a reference that has that restriction?} 
For example consider the minimum cost 2-edge-connected multi-subgraph problem (2EC): Given a graph $G=(V,E)$ and $c\in \mathbb{R}^E_{\geq 0}$, 2EC asks for the minimum cost 2-edge-connected subgraph of $G$, with multiedges allowed. A graph is 2-edge-connected if there are at least two edge-disjoint paths between every pair of vertices.  A linear programming relaxation for this problem known as the subtour elimination relaxation  is
 \begin{equation}\min \{cx: \sum_{e\in \delta(U)}x_e \geq 2 \mbox{ for } \emptyset \subsetneq U \subsetneq V,\; x\in [0,2]^{E}\}, 
 \end{equation}
where $\delta(U)$ for vertex subset $U$ is the set of edges that cross the cut defined by $U$.  That is, each $e \in \delta(U)$ has one endpoint in $U$ and the other endpoint in $V-U$.
 In this case the instance-specific integrality gap is the integrality gap of the subtour-elimination relaxation for the 2EC on graph with $n$ vertices. Alexander et al. \cite{alexander2006integrality} showed this instance-specific integrality gap
%of the subtour elimination relaxation for the 2EC
is at most $\frac{7}{6}$ for instances with $n= 10$ .

The value of $g(I)$ depends on the constraints in (\ref{S}).  We cannot hope to find solutions for the IP with objective values better than $g(I)\cdot z_{LP}(I,c)$. More generally we can define the integrality gap for a class of instances $\mathcal{I}$ as follows.% In this case, the integrality gap for problem $\mathcal{I}$ is
\begin{equation}\label{gapproblem}
g(\mathcal{I}) = \max_{c\geq 0 , I\in\mathcal{I}}\frac{z_{IP}(I,c)}{z_{LP}(I,c)}.
\end{equation}
For example, the integrality gap of the subtour elimination relaxation for the 2EC is at most $\frac{3}{2}$ \cite{wolsey} and at least $\frac{6}{5}$ \cite{alexander2006integrality}. Therefore, we cannot hope to obtain an LP-based $(\frac{6}{5}-\epsilon)$-approximation algorithm for this problem using this LP relaxation to bound the quality of a feasible solution.

Our methods apply theory connecting integrality gaps to sets of feasible solutions. Instances $I$ with $g(I)=1$ have $P(I)=\conv(S(I))$, the convex hull of the lattice of feasible points. In this case, $P(I)$ is an \textit{integral} polyhedron. The spanning tree polytope of graph $G$, $\st(G)$, and the perfect-matching polytope of graph $G$, $\permat(G)$, have this property (\cite{Edmonds2003,edmondsPM}). Thus the linear-programming relaxation for minimum-cost spanning tree has basic feasible solutions (vertices) that are integral solutions, i.e. spanning trees. For such problems there is an algorithm to express vector $x\in P(I)$ (a feasible LP solution) as a convex combination of points in $S(I)$ (feasible IP solutions) in polynomial time \cite{cons-cara}.
% The number of feasible solutions ($k$) in the convex combination can vary by instance.
%\cindy{Should this proposition just be a corollary of the Carr-Vempala theorem?  Are we going to describe how to do it?  If so, there should be a forward pointer to that discussion.}

%\arash{not exactly: Carr-Vempala don't give a construction. The proposition below is a result of constructive version of Carathedory's theorem which I think is implied by the Ellipsoid algorithm, but there are even faster versions of it. In some way the difference between prop 1 and theorem 2 is our initial motivation for thinking of polynomial construction.}
\begin{proposition}\label{cara}
	If $g(I)=1$, then for $x\in P(I)$ there exists a positive integer $k$ and $\theta \in [0,1]^k$, where $\sum_{i=1}^{k}\theta_i =1$ and $\tilde{x}^i\in S(I)$ for $i\in [k]$ such that $\sum_{i=1}^{k}\theta_i \tilde{x}^i\leq x$. Moreover, we can find such a convex combination in polynomial time.
\end{proposition}

An equivalent way of describing Proposition \ref{cara} is the following Theorem of Carr and Vempala \cite{Carr2004}. The {\em dominant} of $P(I)$, which we denote by $\dom(P(I))$, is the set of points $x'$ such that there exists a point $x\in P$ with $x'\geq x$ in every component. A polyhedron is of \textit{blocking type} if it is equal to its dominant.

\begin{thm}[Carr, Vempala \cite{Carr2004}] \label{CV2}
	We have $g(I) \leq C$ if and only if for  $x\in P(I)$ there exists $\theta \in [0,1]^k$ where $\sum_{i=1}^{k}\theta_i =1$ and $\tilde{x}^i\in \dom(S(I))$ for $i\in [k]$ such that $\sum_{i=1}^{k}\theta_i \tilde{x}^i\leq Cx$.
\end{thm}
Goemans \cite{goemansblocking} first introduced Theorem \ref{CV2} for blocking-type polyhedra. While there is an exact algorithm for problems with gap $1$, as stated in Proposition \ref{cara}, Theorem~\ref{CV2} is existential, with no construction.
%In contrast to Proposition \ref{cara} which implies exact algorithms for problems with a gap of 1, Theorem \ref{CV} does not imply an approximation algorithm, since it does not suggest how to find such a convex combination in polynomial time.
%This points to an interesting open question. 
% We show later, for $I$ with $g(I)<\infty$, the notation of dominant is in fact useful.
\iffalse

\begin{question*}\label{question1}
	Assume reasonable complexity assumptions (such as UGC or $\textrm{P}\neq \textrm{NP}$). Given instance $I$ with $1<g(I)<\infty$ and $(x,y)\in P(I)$, can we find $\theta \in [0,1]^k$, where $\sum_{i=1}^{k}\theta_i =1$ and $(\tilde{x}^i,\tilde{y}^i)\in \dom(\conv(S(I)))$ for $i=1,\ldots,k$ such that $\sum_{i=1}^{k}\theta_i \tilde{x}^i\leq g(I)x$ in polynomial time?
\end{question*}

This seems to be a very hard question. A more specific question is of more interest.

\begin{question}\label{question2}
	Assume reasonable complexity assumptions, a specific problem $\mathcal{I}$ with  $1<g({\mathcal{I}})<\infty$, and $(x,y)\in P(I)$ for some $I\in \mathcal{I}$, can we find $\theta \in [0,1]^k$, where $\sum_{i=1}^{k}\theta_i =1$ and $(\tilde{x}^i,\tilde{y}^i)\in S(I)$ for $i=1,\ldots,k$ such that $\sum_{i=1}^{k}\theta_i \tilde{x}^i\leq g(\mathcal{I})x$ in polynomial time?
\end{question}
Although Question \ref{question2} is wide open, for some problems there are polynomial time algorithms closing the gap. For example, for generalized Steiner forest problem \cite{jain} gave an LP-based 2-approximation algorithm. The gap for this problem is also lower bounded by 2. Same holds for the set covering problem \cite{randomizedrounding}. In fact, for set cover the approximation algorithm achieving the same factor as the integrality gap lower bound, is a \textit{randomized rounding} algorithm. Raghavan and Thompson \cite{randomizedrounding} showed that this technique achieves provably good approximation for many combinatorial optimization problems.  

If we relax Question \ref{question1} (resp. Question \ref{question2}), but multiplying $g(I)$ (resp. $g(\mathcal{I})$) by a factor $C$, they are still very interesting, since they will provide upper bounds on the integrality gap of the instance (resp. the problem). The results in this paper serve this purpose.
\fi
To study integrality gaps, we wish to decompose a suitably scaled linear-programming solution into a convex combination of feasible integer solutions {\bf constructively}.  That is, we ask: assuming reasonable complexity assumptions, given a specific problem $\mathcal{I}$ with  $1<g(\mathcal{I})<\infty$, and $x\in P(I)$ for some $I\in \mathcal{I}$, can we find $\theta \in [0,1]^k$, where $\sum_{i=1}^{k}\theta_i =1$ and $\tilde{x}^i\in S(I)$ for $i\in [k]$ such that $\sum_{i=1}^{k}\theta_i \tilde{x}^i\leq Cx$ in polynomial time? We wish to find the smallest factor $C$ possible.

\subsection{Algorithms and Theory Contributions} 
 
We give a general approximation framework for solving binary IPs.
Consider the set of points described by sets $S(I)$ and $P(I)$ as in (\ref{S}) and (\ref{P}), respectively. Assume in addition that $S(I),P(I)\subseteq [0,1]^n$.
\cindy{Is it $S(I) \subseteq \{0,1\}^n$}
For a vector $x\in \mathbb{R}_{\geq 0}^n$ such that $x\in P(I)$, let the {\em support} of $x$ be $\spp(x)= \{i \in [n]: x_i \neq 0\}$. For an integer $\beta$ let $\{\beta\}^n$ be the vector $y\in \mathbb{R}^n$ with $y_i=\beta$ for $i\in [n]$.


We introduce the \textit{Fractional Decomposition Tree Algorithm} (FDT) which runs in polynomial time algorithm. Given a point $x\in P(I)$ FDT produces a convex combination of feasible points in $S(I)$ that are dominated by a ``factor" $C$ of $x$ in the coordinates corresponding to $x$. If $C = g(I)$, it would be optimal. However we can only guarantee a factor of $g(I)^{|\spp(x)|}$. FDT relies on iteratively solving linear programs that are about the same size as the description of $P(I)$.

\begin{restatable}{thm}{binaryFDT}
	\label{binaryFDT}
	Assume $1\leq g(I) 	<\infty$. 	
	The Fractional Decomposition Tree (FDT) algorithm, given $x^*\in P(I)$, produces in polynomial time $\lambda\in [0,1]^k$ and $z^1,\ldots,z^k \in S(I)$ such that $k\leq |\spp(x^*)|$, $\sum_{i=1}^{k}\lambda_i z^i\leq \min(Cx^*,\{1\}^{n})$, and $\sum_{i=1}^{k}\lambda_i = 1$. Moreover, $C\leq g(I)^{|\spp(x^*)|}$.
\end{restatable}

A subroutine of the FDT, called the DomToIP algorithm, finds feasible solutions to any IP with finite gap. This can be of independent interest, especially in proving that a model has unbounded gap.
\begin{restatable}{thm}{DomToIP}
	\label{domtoIP}
	Assume $1\leq g(I) < \infty$. The DomToIP algorithm finds $\hat{x}\in S(I)$ in polynomial time.
\end{restatable}

For a generic IP instance $I$ it is NP-hard to even decide if the set of feasible solutions $S(I)$ is empty or not. There are a number of heuristics for this purpose, such as the feasibility pump algorithm \cite{fp1,fp2}. These heuristics are often very effective and fast in practice, however, they can sometimes fail to find a feasible solution. Moreover, these heuristics do not provide any bounds on the quality of the solution they find. 

Here is how the FDT algorithm works in a high level: in iteration $i$ the algorithm maintains a convex combination of  vectors in $\mathcal{D}(L(I))$ that have a 0 or 1 value for coordinates indexed $0,\ldots,i-1$. Let $y$ be a vector in the convex combination in iteration $i$ of the algorithm. We solve a linear programming problem that gives us $\theta\in [0,1]$ and $y^0,y^1\in \mathcal{D}(L(I))$ such that $g(I) y\geq \theta_1y^0 + (1-\theta) y^1$ and $y^0_i=0$ and $y^1_i=1$. We then replace $y$ in the convex combination with $\frac{\theta}{g(I)}y^0 +\frac{1-\theta}{g(I)}y^1$. Repeating this for every vector in the convex combination from previous iteration yields a convex combination  of points that is ``more'' integral. If in any iteration there are too many points in the convex combination we solve a linear programming problem that ``prunes'' the convex combination. At the end we find a convex combination of integer solutions $\mathcal{D}(L(I))$. For each such solution $z$ we invoke the DomToIP algorithm (see Section \ref{domTOIP}) to find $z'\in S(I)$ where $z'\leq z$.


One can extend the FDT algorithm for binary IPs into covering $\{0,1,2\}$ IPs by losing a factor $2^{|\spp(x)|}$ on top of the loss for FDT. In order to eradicate this extra factor, we need to treat the coordinate $i$ with $x_i=1$ differently. We focus on the \textsc{2-edge-connected multi-subgraph graph problem (2EC)}: Given a graph $G=(V,E)$ and $c\in \mathbb{R}^{E}_{\geq 0}$ find a 2-edge-connected multi-subgraph of $G$ with minimum cost. The natural linear programming relaxation for this problem is 
\begin{equation}
\min \{cx \; : \; x(\delta(U))\geq 2 \; \text{ for }\emptyset \subset U \subset V, \; x\in [0,2]^E\}
\end{equation}
We denote the feasible region of this LP by $\subtour(G)$. Let $\2ec(G)$ be the convex hull of incidence vectors of 2-edge-connected multi-subgraphs of graph $G$. Following the definition in (\ref{gapproblem}) have
\begin{equation}
g(\2ec) = \max_{c\geq 0 , G}\frac{\min_{x\in \2ec(G)} cx}{\min_{x\in \subtour(G)} cx}.
\end{equation}

\begin{restatable}{thm}{FDTEC}
	\label{FDT2EC}
	Let $G=(V,E)$ and $x$ be an extreme point of  $\subtour(G)$. The FDT algorithm for 2EC produces $\lambda\in [0,1]^k$ and 2-edge-connected multi-subgraphs $F_1,\ldots,F_k$ such that $k\leq 2|V|-1$, $\sum_{i=1}^{k}\lambda_i \chi^{F_i}\leq \min(Cx,\{2\}^n)$, and $\sum_{i=1}^{k}\lambda_i = 1$. Moreover, $C\leq g(\2ec)^{|E_x|}$.
\end{restatable}

\subsection{Experiments.} Although the bound guaranteed in both Theorems \ref{binaryFDT} and \ref{FDT2EC} are very large, we show that in practice, the algorithm works very well for network design problems described above. We show how one might use FDT to investigate the integrality gap for such well-studied problems. \cindy{TODO: forward pointers to the sections with more details.}


\subsubsection{Minimum vertex cover problem}

In the \textsc{minimum vetex cover problem (VC)} we are given a graph $G=(V,E)$ and $c\in \mathbb{R}^E_{\geq 0}$. A subset of $U$ of $V$ is a \textit{vertex cover} if for $e\in E$ at least one endpoint of $e$ is in $U$. The goal in VC is to find the minimum cost vertex cover. The linear programming relaxation for VC is
\begin{equation}
\min \{cx \; : \; x_u + x_v \geq	 1 \text{ for } e=uv \in E, \; x\in [0,1]^{V}\}.
\end{equation}
The integrality gap of this formulation is exactly 2 \cite{davids}. Austrin, Khot and Safra~\cite{UGhardVC} show that it is UG-hard to approximte VC within any factor sctrictly better than 2. We compare  FDT and the feasbility pump heuristic \cite{fp1} on the small instances of PACE\footnote{ Parameterized Algorithms and Computational Experiments: \url{https://pacechallenge.org/2019/}} 2019 challenge test cases \cite{PACE}. 
\subsubsection{Tree augmentation problem}
In the \textsc{Tree Augmentation Problem (TAP)} we are given a  graph $G=(V,E)$, a spanning tree $T$ of $G$. We also have a cost vector $c\in \mathbb{R}^{E\setminus T}_{\geq 0}$. A subset $F$ of $E\setminus T$ is called a \textit{feasible augmentation} if $(V,T\cup F)$ is a 2-edge-connected graph. In TAP we seek the minimum cost feasible augmentation. The natural linear programming relaxation for TAP is 
\begin{equation}
\min \{cx\; : \; \sum_{\ell \in \cov(e)} x_{\ell} \geq 1 \text{ for } e\in T, \; x\in [0,1]^{E\setminus T}\}.
\end{equation}
where $\cov(e)$ is set of edges $\ell \in E\setminus T$ such that $e$ is in the unique cycle of $T\cup \{\ell\}$. We call the LP above the cut-LP. The integrality gap of the cut-LP is known to be between $\frac{3}{2}$ \cite{32gaptap} and $2$ \cite{FJ81}. We create random fractional extreme points of the cut-LP and apply FDT to find integral solutions. For the instances that we create, the blow-up factor is always below $\frac{3}{2}$ providing an upper bound for such instances.

\subsubsection{2-edge-connected multi-subgraph problem}
Known polyhedral structure makes it easier to study integrality gaps for such problems. We use the idea of fundamental extreme point \cite{carrravi,boydcarr,Carr2004} to create the ``hardest'' LP solutions to decompose.


There are fairly good bounds for the integrality gap for TSP or 2EC. Benoit and Boyd \cite{TSPcompute} used a quadratic program to show the integrality gap of the subtour elimination relaxation for the TSP, $g(\tsp)$, is at most $\frac{20}{17}$ for graphs with at most 10 vertices. Alexander et al. \cite{alexander2006integrality} used the same ideas to provide an upper bound of $\frac{7}{6}$ for $g(\2ec)$ on graphs with at most 10 vertices. 

Consider a graph $G=(V,E)$. A \textit{Carr-Vempala point} $x\in \mathbb{R}^E$ is a fractional point in $\subtour(G)$ where the edges $e$ with $0<x_e<1$ form a single cycle in $G$ and the vertices on the cycle are connected via vertex-disjoint paths of edges $e$ with $x_e =1$. Carr and Vempala \cite{Carr2004} showed that $g(\2ec)$ is achieved for instances where the optimal solution to $\min_{x\in \subtour(G)}cx$ is a Carr-Vempala point. We show that the integrality gap is at most $\frac{6}{5}$ for Carr-Vempala points with at most 12 vertices on the cycle formed by the fractional edges. Note that the number of vertices in these instances can be arbitrarily high since the paths of edges with $x$-value 1 can be arbitrarily long.






\iffalse{
	\subsection{Notation}
	For vectors $x,y\in \mathbb{R}_{n}$ we say $x$ dominates $y$ if $x_i\geq y_i$ for $i= 1,\ldots,n$. For $m\times n$ matrix $A$, let $A_j$ be the $j$-th row of $A$ and $A^j$ be the $j$-th column of $A$. For a set $S$ of vectors in $\mathbb{R}_{n}$, $\conv(S)$ is the convex hull of all the points in $S$.
}\fi
\section{Finding a Feasible Solution}\label{sec:domTOIP}
In this section we give the algorithm for DomToIP and prove its performance (Theorem~\ref{domtoIP}).

Consider an instance $I=(A,b)$ of the IP formulation. Define sets $S(I)$ and $P(I)$ as in (\ref{S}) and (\ref{P}), respectively. Assume $S(I)\subseteq \{0,1\}^n$ and $P(I)\subseteq [0,1]^n$. For simplicity in the notation we assume an instance $I$ and denote $P(I),S(I),$ and $g(I)$ by $P$, $S$, and $g$ for this section and the next section. Also, for both sections let $x^*$ be the optimal solution to the LP formulation and assume $t=|\spp(x^*)|$. Without loss of generality we can assume $x^*_i = 0$ for $i=t+1,\ldots,n$.
\cindy{TODO: Is this something that should be clear to the reader with what they know at this point? Does it need an explanation? Does it need an explanation at the end of this section or the next?}\arash{this is just a relabeling of the indices.} \cindy{Sorry.  I didn't understand that you meant the solution has that property. Right? The vector $x$ isn't defined in this paragraph and it's not in the next lemma either. If $x$ is the output of DomToIP, then we should say that.  ``Without loss of generality the solution $x$ returned by DomToIP has ...''}\arash{I see your point, please see if the change made it better} \cindy{Yes, thanks.  We can delete this conversation after you see this response.}

In this section we prove Theorem \ref{domtoIP}. In fact, we prove a stronger result. 
\begin{lemma}\label{domlemma}
	Given $\tilde{x}\in \dom(P)$ and $\tilde{x}\in \{0,1\}^n$, there is an algorithm (the DomToIP algorithm) that finds $\bar{x}\in S$ in polynomial time, such that $\bar{x}\leq \tilde{x}$.\end{lemma}
Notice that Lemma \ref{domlemma} implies Theorem \ref{domtoIP}, since it is easy to obtain an integer point in $\dom(P)$: numerically rounding up $x^*$ (or any fractional point in $P$) gives us a point in $\dom(P)$. Hence, we can assume in the proof below that $\tilde{x}_i= 0$ for $i=t+1,\ldots,n$. 
\arash{I also made a change based on the change above}\cindy{Thanks!}
\subsection{Proof of Lemma \ref{domlemma}: The DomToIP Algorithm}

We start by introducing an algorithm that ``fixes" the variables iteratively, starting from the first coordinate and ending at the $t$-th coordinate. Suppose we run the algorithm for $\ell\in \{0,\ldots,t-1\}$ iterations and in each iteration we find $x^{(\ell)}\in \dom(P)$  such that $x^{(\ell)}_i\in \{0,1\}$ for $i=1,\ldots,\ell$. Notice that we can set $x^{(0)}=\tilde{x}$. Now consider the following linear program. The variables of this linear program are the $z\in \mathbb{R}^n$ variables.
\begin{align}
\DOMtoS(x^{(\ell)})\quad\quad& \min\quad \;z_{\ell+1}\\
&\;\text{s.t.} \quad \;\;Az\geq b \\
&\;{\color{white}{\text{s.t.}} }\quad \;\; \; z_j = x^{(\ell)}_j \quad \; j =1,\ldots, \ell\\
&\;{\color{white}{\text{s.t.}} }\quad \; \;\; z_j \leq x^{(\ell)}_j \quad \; j = \ell+1,\ldots,n\\
&\;{\color{white}{\text{s.t.}} }\quad \; \;\; z\;\geq 0
\end{align}

If the optimal value to $\DOMtoS(x^{(\ell)})$ is 0, then let $x^{(\ell+1)}_{\ell+1} = 0$. Otherwise if the optimal value is strictly positive let $x^{(\ell+1)}_{\ell+1} = 1$. Let $x^{(\ell+1)}_j = x^{(\ell)}_j$ for $j\in [n]\setminus \{\ell+1\}$ (See Algorithm \ref{domtoIPalg}).

The above procedure suggests how to find $x^{(\ell+1)}$ from $x^{(\ell)}$. The DomToIP algorithm initializes with $x^{(0)}=\tilde{x}$ and  iteratively calls this procedure in order to obtain $x^{(t)}$. 

\vspace*{10pt}
\begin{algorithm}[h]
	\KwIn{$\tilde{x}\in \dom(P)$, $\tilde{x}\in \{0,1\}^n$ }
	\KwOut{$x^{(t)} \in S$, $x^{(t)}\leq \tilde{x}$}
	$x^{(0)}\leftarrow \tilde{x}$\\
	\For{$\ell = 0$ \textbf{to} $t-1$}{
		$x^{(\ell+1)} \leftarrow x^{(\ell)}$\\
		$\eta \leftarrow$ optimal value of $ \DOMtoS(x^{(\ell)})$\\
		\eIf{$\eta = 0$}{
			$x^{(\ell+1)}_{\ell+1} \leftarrow 0$\
		}{
			$x^{(\ell+1)}_{\ell+1} \leftarrow 1$
		}
	}
	\caption{The DomToIP algorithm}
	\label{domtoIPalg}
\end{algorithm}
\vspace*{10pt}

We prove that indeed $x^{(t)}\in S$. First, we need to show that in any iteration $\ell=  0,\ldots,t-1$ of DomToIP that $\DOMtoS(x^{(\ell)})$ is feasible. We show something stronger. For $\ell=0,\ldots,t-1$ let
\begin{align*}
\LP^{(\ell)}&= \{z\in P\; : \; z\leq x^{(\ell)} \mbox{ and } z_j=x_j^{(\ell)} \mbox{ for } j\in [\ell]\}, \text{ and}\\
\IP^{(\ell)}&= \{z\in \LP^{(\ell)}\; : \; z\in \{0,1\}^n\}.
\end{align*}
Notice that if $\LP^{(\ell)}$ is a non-empty set then $\DOMtoS(x^{(\ell)})$ is feasible. We show by induction on $\ell$ that $\LP^{(\ell)}$ and $\IP^{(\ell)}$ are not empty sets for $\ell=0,\ldots,t-1$. First notice that $\LP^{(0)}$ is clearly feasible since by definition $x^{(0)}\in \dom(P)$, meaning there exists $z\in P$ such that $z\leq x^{(0)}$. By Theorem \ref{CV2}, there exists $\tilde{z}^i\in S$ and $\theta_i\geq 0$ for $i\in [k]$ such that $\sum_{i=1}^{k} \theta_i = 1$ and $\sum_{i=1}^{k}\theta_i \tilde{z}^i \leq gz$. Hence, $\sum_{i=1}^{k}\theta_i \tilde{z}^i \leq gz\leq gx^{(0)}$. So if $x^{(0)}_j=0$, then $ \sum_{i=1}^{k}\theta_i \tilde{z}_j^i =0$, which implies that $\tilde{z}^i_j=0$ for all $i\in [k]$ and $j\in [n]$ where $x^{(0)}_j=0$. Hence, $\tilde{z}^i\leq x^{(0)}$ for $i\in [k]$. Therefore $\tilde{z}^i\in \IP^{(0)}$ for $i\in [k]$, which implies $\IP^{(0)}\neq \emptyset$.

Now assume $\IP^{(\ell)}$ is non-empty for some $\ell \in [t-2]$. Since $\IP^{(\ell)}\subseteq\LP^{(\ell)}$ we have $\LP^{(\ell)}\neq \emptyset$ and hence the $\DOMtoS(x^{(\ell)})$ has an optimal solution $z^*$.

We consider two cases. In the first case, we have $z^*_{\ell+1}=0$. In this case we have $x^{(\ell+1)}_{\ell+1}=0$. Since $z^*\leq x^{(\ell+1)}$, we have $z^*\in \LP^{(\ell+1)}$. Also, $z^*\in P$. By Theorem \ref{CV2} there exists $\tilde{z}^i\in S$ and $\theta_i\geq 0$ for $i\in [k]$ such that $\sum_{i=1}^{k} \theta_i = 1$ and  $\sum_{i=1}^{k}\theta_i \tilde{z}^i \leq gz^*$. We have $\sum_{i=1}^{k}\theta_i \tilde{z}^i \leq gz^*\leq gx^{(\ell+1)}$.
So for $j\in [n]$ where $x^{(\ell+1)}_j=0$, we have $z^i_j=0$ for $i\in [k]$. This implies $\tilde{z}^i\leq x^{(\ell+1)}$ for $i=1,\ldots,k$. Hence, there exists $z\in S$ such that $z\leq x^{(\ell+1)}$. We claim that $z\in \IP^{(\ell+1)}$. If $z\notin \IP^{(\ell+1)}$ we must have $1\leq j \leq \ell$ such that $z_j < x^{(\ell+1)}_{j}$, and thus $z_j = 0$ and $x^{(\ell+1)}_j=1$. Without loss of generality assume $j$ is minimum number satisfying $z_j < x^{(\ell+1)}_{j}$. Consider iteration $j$ of the DomToIP algorithm. Notice that $z\leq x^{(\ell+1)}\leq x^{(j)}$. We have $x^{(j)}_j=1$ which implies when we solved $\DOMtoS(x^{(j-1)})$ the optimal value was strictly larger than zero. However, $z$ is a feasible solution to $\DOMtoS(x^{(j-1)})$ and gives an objective value of 0. This is a contradiction, so $z\in \IP^{(\ell+1)}$.

Now for the second case, assume $z^*_{\ell+1} > 0$. We have $x^{(\ell+1)}_{\ell+1}=1$. Notice that for each point $z\in \LP^{(\ell)}$ we have $z_{\ell+1} >0$, so for each $z\in \IP^{(\ell)}$ we have $z_{\ell+1}>0$, i.e. $z_{\ell+1}=1$. This means that $z\in \IP^{(\ell+1)}$, and $\IP^{(\ell+1)} \neq \emptyset$.

Now consider $x^{(t)}$. Let $z$ be the optimal solution to $\LP^{(t-1)}$. If $x^{(t)}_t = 0$, we have $x^{(t)} = z$, which implies that $x^{(t)}\in P$, and since $x^{(t)}\in \{0,1\}^n$ we have $x^{(t)}\in S$. If $x^{(t)}_t =1$, it must be the case that $z_t > 0$. By the argument above there is a point $z'\in \IP^{(t-1)}$. We show that $x^{(t)} = z'$. For $j\in [t-1]$ we have $z'_j= x_j^{(t-1)}=x_j^{(t)}$. We just need to show that $z'_t = 1$. Assume $z'_t	 = 0$ for contradiction, then $z'\in \LP^{(t-1)}$ has objective value of $0$ for $\DOMtoS(x^{(t-1)})$, this is a contradiction to $z$ being the optimal solution. This concludes the proof of Lemma \ref{domlemma}. 





\section{FDT on Binary IPs}
\label{sec:binaryfdt}
\arash{I made a small change here based on our discussion in the bottom of page 8}\cindy{Thanks!}
Recall that $x^*$ was the optimal solution of minimizing a cost function $cx$ over set $P$, which provides a lower bound on $\min_{(x,y)\in S(I)} cx$.  In this section, we prove Theorem \ref{binaryFDT} by describing the Fractional Decomposition Tree (FDT) algorithm. We also remark that if $g(I)=1$, then the algorithm will give an exact decomposition of any feasible solution. 


The FDT algorithm grows a tree similar to the classic branch-and-bound search tree for integer programs. Each node represents a partially integral vector $\bar{x}$ in $\dom(P)$ together with a multiplier $\bar{\lambda}$. The solutions contained in the nodes of the tree become progressively more integral at each level. In each level of the tree, the algorithm maintain a conic combination of points with the properties mentioned above. Leaves of the FDT tree contain solutions with integer values for all the $x$ variables that dominate a point in $P$. In Lemma  \ref{domlemma} we saw how to turn these into points in $S$. 

\paragraph{Branching on a node.}
We begin with the following lemmas that show how the FDT algorithm branches on a variable.
\begin{lemma}\label{LPClemma}
	Given $x'\in \dom(P)$ and $\ell\in [n]$ where $x'_{\ell}<1$, we can find in polynomial time vectors $\hat{x}^0,\hat{x}^1$ and scalars $\gamma_0,\gamma_1 \in [0,1]$ such that: (i) $\gamma_0 + \gamma_1  \geq 1/g$, (ii) $\hat{x}^0$ and $\hat{x}^1$ are in  $ P$
	,(iii) $\hat{x}^0_\ell=0$ and $\hat{x}^1_\ell=1$, (iv) $\gamma_0 \hat{x}^0 + \gamma_1\hat{x}^1 \leq x'$.
\end{lemma}


\begin{proof}	
	Consider the following linear program which we denote by $\LPC(\ell,x')$. The variables of $\LPC(\ell,x')$ are $\gamma_0,\gamma_1$ and $x^0$ and $x^1$. 
	\begin{align}
	\LPC(\ell,x')\quad\quad& \max\quad \;\lambda_0+\lambda_1\\
	&\;\text{s.t.} \quad Ax^j \geq b\lambda_j & \mbox{ for $j=0,1$} \label{feasibility}\\
	&\;{\color{white}{\text{s.t.}} }\quad 0 \leq x^j \leq \lambda_j &\mbox{ for $j=0,1$}\label{bound}\\
	&\;{\color{white}{\text{s.t.}} }\quad x^0_\ell = 0,\; x^1_\ell =\lambda_1\label{branchcoordinate}\\
	&\;{\color{white}{\text{s.t.}} }\quad x^0 + x^1 \leq x'\label{packing}\\
	&\;{\color{white}{\text{s.t.}} }\quad \lambda_0,\lambda_1 \geq 0
	\end{align}
	
	Let $x^0,x^1$, and $\gamma_0,\gamma_1$ be an optimal solution to the LP above. Let $\hat{x}^0 = x^0/\gamma_0$, $\hat{x}^1=x^1/\gamma_1$. This choice satisfies  (ii), (iii), (iv). To show that (i) is also satisfied we prove the following claim.
	
	\begin{claim}\label{CVexists}
		We have $\gamma_0 + \gamma_1\geq 1/g$.
	\end{claim}
	\begin{cproof}
		We show that there is a feasible solution that achieves the objective value of $\frac{1}{g}$. By Theorem \ref{CV2} there exists $\theta \in [0,1]^k$, with $\sum_{i=1}^{k}\theta_i = 1$ and $\tilde{x}^i\in S$ for $i\in[k]$ such that 
		$\sum_{i=1}^{k}\theta_i \tilde{x}^i\leq gx'$. So
		
		\begin{equation}\label{splitting}
		x'\geq \sum_{i=1}^{k}\frac{\theta_i}{g} \tilde{x}^i
		={\sum_{i\in [k]: \tilde{x}^i_\ell =0}\frac{\theta_i}{g} \tilde{x}^i}+{\sum_{i\in [k]: \tilde{x}^i_\ell =1}\frac{\theta_i}{g} \tilde{x}^i}.
		\end{equation}
		For $j=0,1$, let $x^j = \sum_{i\in [k]:\tilde{x}^i_\ell=j} \frac{\theta_i}{g}\tilde{x}^i$. Also let $\lambda_0=\sum_{i\in [k]: \tilde{x}^i_\ell =0}\frac{\theta_i}{g}$ and $\lambda_1 = \sum_{i\in [k]: \tilde{x}^i_\ell =1}\frac{\theta_i}{g}$. Note that $\lambda_0+\lambda_1 =1/g$. Constraint (\ref{packing}) is satisfied by Inequality (\ref{splitting}). Also, for $j=0,1$ we have
		\begin{equation}
		Ax^j= \sum_{i\in[k], \tilde{x}^i_\ell = j} \frac{\theta_i}{g} A\tilde{x}^i  \geq b \sum_{i\in[k], \tilde{x}^i_\ell = j} \frac{\theta_i}{g} = b\lambda_j.
		\end{equation}
		Hence, Constraints (\ref{feasibility}) holds. Constraint (\ref{branchcoordinate}) also holds since $x^0_\ell$ is obviously $0$ and $x^1_\ell= \sum_{i\in [k]: \tilde{x}^i_\ell = 1}\frac{\theta_i}{g}= \lambda_1$. The rest of the constraints trivially hold. 
	\end{cproof}
	This concludes the proof of Lemma \ref{LPClemma}.	
\end{proof}

We now show if $x'$ in the statement of Lemma \ref{LPClemma} is partially integral, we can find solutions with more integral components.
\begin{lemma}\label{round-up}
	Given $x'\in \dom(P)$ where $x'_1,\ldots,x'_{\ell-1}\in \{0,1\}$ and $x'_{\ell}<1$ for some $\ell\geq 1$ we can find in polynomial time vectors $\hat{x}^0,\hat{x}^1$ and scalars $\gamma_0,\gamma_1 \in [0,1]$ such that: (i) ${ 1}/{g}\leq \gamma_0 + \gamma_1  \leq 1$, (ii) $\hat{x}^0$ and $\hat{x}^1$ are in  $\dom( P)$, (iii) $\hat{x}^0_\ell=0$ and $\hat{x}^1_\ell=1$, (iv) $ \gamma_0\hat{x}^0 +\gamma_1 \hat{x}^1 \leq
	x'$,(v) $\hat{x}^i_j\in \{0,1\}$ for $i=0,1$ and $j\in[\ell-1]$.
\end{lemma} 
\begin{proof}
	By Lemma \ref{LPClemma} we can find $\bar{x}^0$, $\bar{x}^1$, $\gamma_0$ and $\gamma_1$ that satisfy (i), (ii), (iii), and (iv). We define $\hat{x}^0$ and $\hat{x}^1$ as follows. For $i=0,1$, for $j\in[\ell-1]$, let $\hat{x}^i_j= \ceil{\bar{x}^i_j}$, for $j=\ell,\ldots,t$ let $\hat{x}^i_j = \bar{x}^i_j$.
	
	
	We now show that $\hat{x}^0$, $\hat{x}^1$, $\gamma_0$, and $\gamma_1$ satisfy all the conditions. Note that conditions (i), (ii), (iii), and (v) are trivially satisfied. Thus we only need to show (iv) holds. We need to show that $\gamma_0 \hat{x}^0_j+\gamma_1\hat{x}^1_j\leq gx'_j$. If $j=\ell,\ldots,t$, then this clearly holds. Hence, assume $j\leq \ell-1$. By the property of $x'$ we have $x'_j\in \{0,1\}$. If $x'_j= 0$, then by Constraint (\ref{packing}) we have $\bar{x}^0_j = \bar{x}^1_j=0$. Therefore, $\hat{x}^i_j=0$ for $i=0,1$, so (iv) holds. Otherwise if $x'_j = 1$, then we have
	$\gamma_0\hat{x}^0_j+\gamma_1\hat{x}^1_j\leq \gamma_0+\gamma_1\leq 1\leq x'_j.$ 
	Therefore (v) holds.
\end{proof}

\paragraph{Growing and Pruning FDT tree.} The FDT algorithm maintains nodes $L_i$ in iteration $i$ of the algorithm. The nodes in $L_i$ correspond to the nodes in level $L_i$ of the FDT tree. The points in the leaves of the FDT tree, $L_t$, are points in $\dom(P)$ and are integral for all integer variables.


\begin{lemma}\label{prune}
	There is a polynomial time algorithm that produces sets $L_0,\ldots,L_t$ of pairs of $x\in \dom(P)$ together with multipliers $\lambda$ with the following properties for $i=0,\ldots,t$:
	(a) If $x\in L_i$, then $x_j \in \{0,1\}$ for $j\in [i]$, i.e. the first $i$ coordinates of a solution in level $i$ are integral, (b) $\sum_{[x,\lambda]\in L_i} \lambda\geq\frac{1}{g^i}$, (c) $\sum_{[x,\lambda]\in L_i}\lambda x \leq x^*$, (d) $|L_i|\leq t$.
\end{lemma}
\begin{proof}
	We prove this lemma using induction but one can clearly see how to turn this proof into a polynomial time algorithm. Let $L_0$ be the set that contains a single node (\textit{root of the FDT tree}) with $x^*$ and multiplier 1. It is easy to check all the requirements in the lemma are satisfied for this choice.
	
	Suppose by induction that we have constructed sets $L_0,\ldots,L_i$. Let the solutions in $L_i$ be $x^j$ for $j\in [k]$ and $\lambda_j$ be their multipliers, respectively. For each $j\in[k]$ if $x^j_{i+1}=1$ we add the pair $(x^j,\lambda_j)$ to $L'$. Otherwise,	applying Lemma \ref{round-up} (setting $x'= x^j$ and $\ell = i+1$) we can find $x^{j0}$, $x^{j1}$, $\lambda^0_j$ and $\lambda^1_j$ with the properties (i) to (v) in Lemma \ref{round-up}. Add the pairs  $(x^{j0} ,\lambda_j\lambda^0_j)$ and  $(x^{j1} ,\lambda_j\lambda^1_j)$ to $L'$. It is easy to check that set $L'$ is a suitable candidate for $L_{i+1}$, i.e. set $L'$ satisfies (a), (b) and (c). However we can only ensure that $|L'|\leq 2k\leq 2t$, and might have $|L'|>t$. We call the following linear program $\prun(L')$. Let $L' = \{[x^1,\gamma_1],\ldots,[x^{|L'|},\gamma_{|L'|}]\}$. The variables of $\prun(L')$ are scalar variables $\theta_j$ for each node $j$ in $L'$.  
		\begin{equation}
		\prun(L')\quad\quad\quad \{\max \sum_{j=1}^{|L'|} \theta_j\;:\; \sum_{j=1}^{|L'|} \theta_j x^j_i\leq x^*_i \mbox{ for $i\in [t]$},\; \theta\geq 0 \}
		\end{equation}
		
		Notice that $\theta = \gamma$ is in fact a feasible solution to $\prun(L')$. Let $\theta^*$ be the optimal vertex solution to this LP. Since the problem is in $\mathbb{R}^{|L'|}$,  $\theta^*$ has to satisfy $|L'|$ linearly independent constraints at equality. However, there are only $t$ constraints of type $ \sum_{j=1}^{|L'|} \theta_j x^j_i\leq x^*_i$. Therefore, there are at most $t$ coordinates of $\theta^*_j$ that are non-zero. Set $L_{i+1}$ which consists of $x^j$ for $j=1,\ldots,|L'|$ and their corresponding multipliers $\theta^*_j$ satisfy the properties in the statement of the lemma. Notice that, we can discard the nodes in $L_{i+1}$ that have $\theta^*_j=0$, so $|L_{i+1}| \leq t$. Also, since $\theta^*$ is optimal and $\gamma$ is feasible for $\prun(L')$, we have $\sum_{j=1}^{|L'|} \theta^*_j \geq \sum_{j=1}^{|L'|}\gamma_j \geq \frac{1}{g^{i+1}}$. \end{proof}
	
	\paragraph{From leaves of FDT to feasible solutions.}
	For the leaves of the FDT tree,  $L_t$, we have that every solution $x$ in $L_t$ has $x\in\{0,1\}^n$ and $x\in \dom(P)$. By applying Lemma \ref{domlemma} we can obtain a point $x'\in S$ such that $x'\leq x$. This concludes the description of the FDT algorithm and proves Theorem \ref{binaryFDT}. See Algorithm \ref{FDTFull} for a summary of the FDT algorithm.
	
	\vspace*{8pt}
	
	
	\begin{algorithm}[H]\label{FDTFull}
		\KwIn{$P= \{x\in \mathbb{R}^{n}: Ax\geq b\}$ and $S=\{x\in P: x\in \{0,1\}^n\}$ such that $g=\max_{c\in \mathbb{R}^n_+ }\frac{\min_{x\in S}cx}{\min_{x\in P}cx}$ is finite, $x^*\in P$}
		\KwOut{$z^i\in S$ and $\lambda_i\geq 0$ for $i\in[k]$ such that $\sum_{i=1}^{k}\lambda_i = 1$, and $\sum_{i=1}^{k}\lambda_iz^i\leq g^tx^*$ }
		$L^0\leftarrow [x^*,1]$\\
		\For{$i=1$ \textbf{to} $t$}{
			$L'\leftarrow \emptyset$\\
			\For{$[x,\lambda] \in L^i$}{
				Apply Lemma \ref{round-up} to obtain $[\hat{x}^0,\gamma_0]$ and $[\hat{x}^1,\gamma_1]$\\
				$L' \leftarrow L' \cup \{[\hat{x}^0,\lambda\cdot\gamma_0]\} \cup \{[\hat{x}^1,\lambda\cdot\gamma_1]\}$\\			
			}
			Apply Lemma \ref{prune} to prune $L'$ to obtain $L^{i+1}$. 
		}
		\For{$[x,\lambda] \in L^t$}{
			Apply Algorithm \ref{domtoIPalg} to $x$ to obtain $z\in S$\\
			$F \leftarrow F \cup \{[z,\lambda]\}$
		}
		\textbf{return} $F$
		\caption{Fractional Decomposition Tree Algorithm}
	\end{algorithm}
	
	
\vspace{8pt}
It is not difficut to see that the number of nodes in the FDT tree is $O(n^2)$. A faster way to achieve feasible solutions with good quality for an IP with bounded integrality gap is an algorithm that takes a random dive into the FDT tree, hence only visiting $O(n)$ nodes. 

	
	\vspace*{8pt}
	
	\begin{algorithm}[H]\label{FDT-dive}
		\KwIn{$P= \{x\in \mathbb{R}^{n}: Ax\geq b\}$ and $S=\{x\in P: x\in \{0,1\}^n\}$ such that $g=\max_{c\in \mathbb{R}^n_+ }\frac{\min_{x\in S}cx}{\min_{x\in P}cx}$ is finite, $x^*\in P$}
		\KwOut{$z \in S$}
		$y = x^*$\\
		\For{$i=1$ \textbf{to} $t$}{
			
			Apply Lemma \ref{round-up} to obtain $[\hat{x}^0,\gamma_0]$ and $[\hat{x}^1,\gamma_1]$\\
			$i \sim\text{Bernoulli}(\frac{\gamma_0}{\gamma_0+\gamma_1})$\\
			$y\rightarrow \hat{x}^i$
			
		}
		Apply Algorithm \ref{domtoIPalg} to $y$ to obtain $z\in S$\\
		\textbf{return} $z$
		\caption{Dive FDT Algorithm}
	\end{algorithm}
	



\section{FDT for 2EC}\label{sec:2EC}

In Section~\ref{sec:binaryfdt} our focus was on binary IPs. In this section, in an attempt to extend FDT to \{0,1,2\} problems we introduce an FDT algorithm for a 2-edge-connected multi-subgraph problem. Given a graph $G=(V,E)$ a multi-subset of edges $F$ of $G$ is a 2-edge-connected multi-subgraph of $G$ if for each set $\emptyset\subset U \subset V$, the number of edges in $F$ that have one endpoint in $U$ and one not in $U$ is at least 2. In 2EC, we are given non-negative costs on the edges of $G$ and the goal is to find the minimum cost 2-edge-connected multi-subgraph of $G$. We want to prove Theorem \ref{FDT2EC}.
\FDTEC*
We do not know the exact value for $g(\2ec)$, but we know $\frac{6}{5} \leq g(\2ec) \leq \frac{3}{2}$ \cite{alexander2006integrality,wolsey}. The FDT algorithm for 2EC is very similar to the one for binary IPs, but there are some differences as well. A natural thing to do is to have three branches for each node of the FDT tree, however, the branches that are equivalent to setting a variable to $1$, might need further decomposition. That is the main difficulty when dealing with $\{0,1,2\}$-IPs.

First, we need a branching lemma. Observe that  the following branching lemma is essentially a translation of Lemma \ref{LPClemma} for $\{0,1,2\}$ problems except for one additional clause. 

\begin{restatable}{lemma}{2ECLPC}
	\label{LPC2EC}
	Given $x\in \subtour(G)$, and $e\in E$ we can find in polynomial time vectors $x^0,x^1$ and $x^2$ and scalars $\gamma_0,\gamma_1$, and $\gamma_2$ such that: (i) $\gamma_0 + \gamma_1 +\gamma_2 \geq { 1}/{g(\2ec)}$, (ii) $x^0,x^1,$ and $x^2$ are in  $ \subtour(G)$, (iii) $x^0_e=0$, $x^1_e=1$, and $x^2_e=2$, (iv) $\gamma_0 x^0 + \gamma_1{x}^1  + \gamma_2x^2\leq {x}$, (v) for $f\in E$ with ${x}_f\geq 1$, we have $x^j_f\geq 1$ for $j=0,1,2$.
\end{restatable}

\begin{proof}
	Consider the following LP with variables $\lambda_j$ and $x^j$ for $j=0,1,2$. 
	\begin{align}
	\quad\quad& \max\quad \;\sum_{j=0,1,2}\lambda_j\\
	&\;\text{s.t.} \quad x^j(\delta(U))\geq 2\lambda_j \;& \mbox{ for $\emptyset \subset U \subset V$, and $j=0,1,2$} \label{feasibility2ec}\\
	&\;{\color{white}{\text{s.t.}} }\quad 0 \leq x^j \leq 2\lambda_j\; &\mbox{ for $j=0,1,2$}\label{bound2ec}\\
	&\;{\color{white}{\text{s.t.}} }\quad x^j_e =j\cdot \lambda_j\; &\mbox{ for $j=0,1,2$}\label{branchcoordinate2ec}\\
	&\;{\color{white}{\text{s.t.}} }\quad x^j_f \geq \lambda_j \; &\mbox{ for $f\in E$ where $x_f \geq 1$, and $j=0,1,2$}\label{1edges2ec}\\
	&\;{\color{white}{\text{s.t.}} }\quad x^0 + x^1+x^2 \leq x\label{packing2ec}\\
	&\;{\color{white}{\text{s.t.}} }\quad \lambda_0,\lambda_1,\lambda_2 \geq 0
	\end{align}	Let $x^j$, $\gamma_j$ for $j=0,1,2$ be an optimal solution solution to the LP above. Let $\hat{x}^{j}={x^j}/{\gamma_j}$ for $j=0,1,2$ where $\gamma_j>0$. If $\gamma_j=0$, let $\hat{x}^{j}=0$. Observe that  (ii), (iii), (iv), and (v) are satisfied with this choice. We can also show that $\gamma_0+\gamma_1+\gamma_2\geq {1}/{g(\2ec)}$, which means that (i) is also satisfied. The proof is similar to the proof of the claim in Lemma \ref{LPClemma}, but we need to replace each $f\in E$ with $x_f\geq 1$ with a suitably long path to ensure that Constraint (\ref{1edges2ec}) is also satisfied.	
	\begin{claim}\label{CVexists}
		We have $\gamma_0 + \gamma_1+\gamma_2\geq \frac{1}{g(\2ec)}$.
	\end{claim}
	\begin{cproof}
		Suppose for contradiction $\sum_{j=0,1,2}\gamma_j = \frac{1}{g(\2ec)} - \epsilon$ for some $\epsilon >0$. Construct graph $G'$ by removing edge $f$ with $x_f\geq 1$ and replacing it with a path $P_f$ of length $\ceil{\frac{2}{\epsilon}}$. Define $x'_h = x_h$ for each edge $h$ such that $x_h<1$. For each $h\in P_f$ let $x'_h= x_f$ for all $f$ with $x_f\geq 1$. It is easy to check that $x'\in \subtour(G')$. By Theorem \ref{CV2} there exists $\theta \in [0,1]^k$, with $\sum_{i=1}^{k}\theta_i = 1$ and 2-edge-connected multi-subgraphs $F'_i$ of $G'$ for $i=1,\ldots,k$ such that 
		$\sum_{i=1}^{k}\theta_i \chi^{F'_i}\leq g(\2ec)x'$. 
		
		Note that each $F'_i$ contains at least one copy of every edge in any path $P_f$, except for at most one edge in the path. We will obtain 2-edge-connected multi-subgraphs $F_1,\ldots,F_k$ of $G$ using $F'_1,\ldots,F'_k$, respectively. To obtain $F_i$ first remove all $P_f$ paths from $F'_i$. Suppose there is an edge $h$ in $P_f$ such that $\chi^{F'_i}_h=0$, this means that for any edge $p\in P_f$ such that $p\neq h$, $\chi^{F'_i}_p=2$. In this case, let $\chi^{F_i}_f=2$, i.e. add two copies of $f$ to $F_i$. If there are at least one edge $h\in P_f$ with $\chi^{F'_i}_h= 1$, let $\chi^{F_i}_f=1$, i.e. add one copy of $f$ to $F_i$. If for all edges $h\in P_f$, we have $\chi^{F'_i}_h=2$, then let $\chi^{F_i}_f=2$. For $f\in E$ with $x_f<1$ we have
		\begin{equation}
		\sum_{i=1}^{k}\theta_i \chi^{F_i}_f=\sum_{i=1}^{k}\theta_i \chi^{F'_i}_f\leq g(\2ec)x'_f= g(\2ec)x_f.
		\end{equation}
		In addition for $f\in E$ with $x_f\geq 1$ we have $\chi^{F_i}_f \leq \frac{\sum_{h\in P_f}\chi^{F'_i}_h}{\ceil{\frac{2}{\epsilon}}-1}$ by construction.
		\begin{align*}
		\sum_{i=1}^{k}\theta_i \chi^{F_i}_f&\leq \sum_{i=1}^{k}\theta_i\frac{\sum_{h\in P_f}\chi^{F'_i}_h}{\ceil{\frac{2}{\epsilon}}-1}\\
		&= \frac{\sum_{h\in P_f} \sum_{i=1}^{k}\theta_i\chi^{F'_i}_h}{\ceil{\frac{2}{\epsilon}}-1}\\
		&\leq \frac{\sum_{h\in P_f} g(\2ec)x'_h}{\ceil{\frac{2}{\epsilon}}-1}\\
		&= \frac{\sum_{h\in P_f} g(\2ec)x_f}{\ceil{\frac{2}{\epsilon}}-1}\\
		&= \frac{\ceil{\frac{2}{\epsilon}}}{\ceil{\frac{2}{\epsilon}}-1}g(\2ec)x_f.
		\end{align*}
		Therefore, since $\frac{\ceil{\frac{2}{\epsilon}}}{\ceil{\frac{2}{\epsilon}}-1}\geq 1$, we have 
		\begin{equation}
		x \geq\sum_{i\in [k]: \chi^{F_i}_e=0}\frac{\theta_i(\ceil{\frac{2}{\epsilon}}-1)}{g(\2ec)\ceil{\frac{2}{\epsilon}}}\chi^{F_i}+ \sum_{i\in [k]: \chi^{F_i}_e=1}\frac{\theta_i(\ceil{\frac{2}{\epsilon}}-1)}{g(\2ec)\ceil{\frac{2}{\epsilon}}}\chi^{F_i}+\sum_{i\in [k]: \chi^{F_i}_e=2}\frac{\theta_i(\ceil{\frac{2}{\epsilon}}-1)}{g(\2ec)\ceil{\frac{2}{\epsilon}}}\chi^{F_i}.
		\end{equation}
		Let $x^j = \sum_{i\in [k]: \chi^{F_i}_e=j}\frac{\theta_i(\ceil{\frac{2}{\epsilon}}-1)}{g(\2ec)\ceil{\frac{2}{\epsilon}}}\chi^{F_i}$ and $\theta_j =  \sum_{i\in [k]: \chi^{F_i}_e=j}\frac{\theta_i(\ceil{\frac{2}{\epsilon}}-1)}{g(\2ec)\ceil{\frac{2}{\epsilon}}}$ for $j=0,1,2$. It is easy to check that $x^j$ , $\theta_j$ for $j=0,1,2$ is a feasible solution to the LP above. Notice that $\sum_{j=0,1,2}\theta_j = \frac{\ceil{\frac{2}{\epsilon}}-1}{g(\2ec)\ceil{\frac{2}{\epsilon}}}$. By assumption, we have $\frac{\ceil{\frac{2}{\epsilon}}-1}{g(\2ec)\ceil{\frac{2}{\epsilon}}}\leq  \frac{1}{g(\2ec)}-\epsilon$, which is a contradiction.
	\end{cproof}
	This concludes the proof. \end{proof}
In contrast to FDT for binary IPs where we round up the fractional variables that are already branched on at each level, in FDT for 2EC we keep all coordinates as they are and perform a rounding procedure at the end. Formally, let $L_i$ for $i=1,\ldots,|\spp(x^*)|$ be collections of pairs of feasible points in $\subtour(G)$ together with their multipliers. Let $t=|\spp(x^*)|$ and assume without loss of generality that $\spp(x^*)=\{e_1,\ldots,e_t\}$. 

\begin{lemma}\label{2ecpruning}
	The FDT algorithm for 2EC in  polynomial time produces sets $L_0,\ldots,L_t$ of pairs $x\in \2ec(G)$ together with multipliers $\lambda$ with the following properties for $i\in [t]$:
	(a) If $x\in L_i$, then $x_{e_j}=0$ or $x_{e_j}\geq 1$ for $j=1,\ldots,i$, (b) $\sum_{(x,\lambda)\in L_i }\lambda \geq \frac{1}{g(\2ec)^i}$, (c) $\sum_{(x,\lambda)\in L_i }\lambda x \leq x^*$, (d) $|L_i|\leq t$.
\end{lemma}
The proof is similar to Lemma \ref{prune}, but we need to use property (v) in Lemma \ref{LPC2EC} to prove that (a) also holds.
\begin{proof}
	We proceed by induction on $i$. Define $L_0=\{(x^*,1)\}$. It is easy to check all the properties are satisfied. Now, suppose by induction we have $L_{i-1}$ for some $i=1,\ldots,t$ that satisfies all the properties. For each solution $x^\ell$ in $L_{i-1}$ apply Lemma \ref{LPC2EC} on $x^\ell$ and $e_{i}$ to obtain $x^{\ell j}$ and $\lambda_{\ell j}$ for $j=0,1,2$. Let $L'$ be the collection that contains $(x^{\ell j},\lambda_\ell \cdot \lambda_{\ell j})$ for $j=0,1,2$, when applied to all $(x^\ell,\lambda_\ell)$ in $L_{i-1}$. Similar to the proof in Lemma \ref{prune} one can check that $L_i$ satisfies properties (b), (c). We now verify property (a). Consider a solution $x^\ell$ in $L_{i-1}$. For $e\in \{e_1,\ldots,e_{i-1}\}$ if $x^\ell_e =0$, then by property (iv) in Lemma \ref{LPC2EC} we have $x^{\ell j}=0$ for $j=0,1,2$. Otherwise by induction we have $x^{\ell}_{e}\geq 1$ in which case property (v) in Lemma \ref{LPC2EC} ensures that $x^{\ell j}_e\geq 1$ for $j=0,1,2$. Also, $x^{\ell j}_{e_i}= j$, so $x^{\ell j}_{e_i}=0$ or $x^{\ell j}_{e_i}\geq 1$ for $j=0,1,2$. 
	
	Finally, if $|L'|\leq t$ we let $L_i=L'$, otherwise apply $\prun(L')$ to obtain $L_{i}$.
\end{proof}

Consider the solutions $x$ in $L_t$. For each variable $e$ we have $x_e=0$ or $x_e\geq 1$. 
\begin{lemma}\label{rounddown}
	Let $x$ be a solution in $L_t$. Then $\floor{x} \in \subtour(G)$. 
\end{lemma}
\begin{proof}
	Suppose not. Then there is a set of vertices $\emptyset \subset U \subset V$ such that $\sum_{e\in \delta(U)}\floor{x_e}<2$. Since $x\in \subtour(G)$ we have $\sum_{e\in \delta(U)}x_e \geq 2$. Therefore, there is an edge $f\in \delta(U)$ such that $x_f$ is fractional. By property (a) in Lemma \ref{2ecpruning}, we have $1<  x_f < 2$. Therefore, there is another edge $h$ in $\delta(U)$ such that $x_h>0$, which implies that $x_h\geq 1$. But in this case $\sum_{e\in \delta(U)}\floor{x_e}\geq  \floor{x_f}+\floor{x_h}  \geq 2$. This is a contradiction.
\end{proof}

The FDT algorithm for 2EC iteratively applies Lemmas \ref{LPC2EC} and \ref{2ecpruning} to variables $x_1,\ldots,x_t$ to obtain leaf point solutions $L_t$. Finally, we just need to apply Lemma \ref{rounddown} to obtain the 2-edge-connected multi-subgraphs from every solution in $L_t$. Notice that since $x$ is an extreme point we have $t\leq 2|V|-1$ \cite{boydpulley}. By Lemma \ref{2ecpruning} we have
\begin{align*}
\sum_{(x,\lambda)\in L_t} \frac{\lambda}{\sum_{(x,\lambda)\in L_t}\lambda} \floor{x} \leq \frac{1}{\sum_{(x,\lambda)\in L_t}\lambda} \sum_{(x,\lambda)\in L_t} \lambda {x} \leq g^t_{\2ec} x^*.
\end{align*}
\section{Computational Experiments with FDT}\label{sec:experiment}
We ran FDT on three network design problems: VC, TAP and 2EC. 

We implemented the experiments for VC and TAP in Python running on a linux workstation (Ubuntu 18.04.3) with 8 cores of Intel(R) Core(TM) i7-8565U CPU  1.80GHz processors and 1Mb of cache. We used the CPLEX 12.9.0.0 solver to solve the pyomo LP models. We ran the experiments for 2EC on a Windows machine, coded in AMPL with CPLEX as the solver.
\paragraph{FDT on VC instances from (PACE 2019) \cite{PACE}.}


We compared Dive FDT (Algorithm \ref{FDT-dive}) with feasbility pump \cite{fp1} in terms of running time
% spent solving LP relaxations
and the quality of solution provided by each algorithm. We used the small (200 vertex) test cases from the PACE 2019 vertex-cover challenge. The results are presented in Figure \ref{fpvsfdt}. 

\begin{figure}[h!]
\begin{subfigure}{.5\textwidth}
\centering
	\includegraphics[width=8cm,scale=1]{"fpvsfdt".png}
	\caption{Dive FDT vs feasbility pump on the instances of PACE 2019 \cite{PACE} with 200 vertices.}
	\label{fpvsfdt}
	\end{subfigure}
	$\quad\;$
	\begin{subfigure}{.5\textwidth}
	\centering
	\includegraphics[width=8cm,scale=1]{"fjvsfdt".png}
%	\caption{FDT vs the 2-approximation for TAP\cite{FJ81} on randomly generated extreme points of the cut-LP.}
	\caption{TAP on our random instances: FDT run on LP optimal vs the 2-approximation for TAP\cite{FJ81}.}
	\label{fjvsfdt}
	\end{subfigure}
	\caption{Computational experiments with the FDT algorithm}
	\label{fdtcomp}
\end{figure}
\paragraph{FDT on randomly generated instances of TAP.}
Recall that in the tree-augmentation problem (TAP) we are given a tree $T=(V,E)$, a set of non-tree links $L$ between vertices in $V$ and costs $c\in \mathbb{R}^{L}_{\geq 0}$. A feasible augmentation is $L'\subseteq L$ such that $T+L'$ is 2-edge-connected. In TAP we wish to find the minimum-cost feasible augmentation. The integrality gap of the cut LP for TAP (given in (\ref{eq:cutLP})) is
\begin{equation*}
g(\tap) = \max_{c\in \mathbb{R}^L_{\geq 0}} \frac{\min_{x\in\tap(T,L)} cx}{\min_{x\in\cut(T,L)} cx},
\end{equation*}  
where $\tap(T,L)$ is the feasible set for the IP and $\cut(T,L)$ is the feasible set for the cut LP (relaxation).
%\arash{Did you intentionally change the TAP and CUT module? They print as italic now as opposed to how they look in $\tap(T,L)$ and $\cut(T,L)$}
%\cindy{Did you mean macro?  No, I didn't intentionally change it.  I just hadn't noticed you had a macro for those two and was in a rush.  At least for the previous section, I changed to the macros. I don't immediately see other places where it's used in math mode, and therefore needs the macro.}
We know $\frac{3}{2}\leq g(\tap)\leq 2$~\cite{FJ81,32gaptap}. Notice that $\min_{x\in \tap(T,L)}cx$ is a binary IP. 

As input for our experiments, we considered full binary trees with 3 to 7 levels with a link for each pairs of leaves. We set the link costs uniformly at random.  We summarize these test instances in Table~\ref{tableTAP}. We ran binary FDT on each test instance and chose the solution with minimum cost. We compare the FDT solutions to those from the circulation-based 2-approximation algorithm of Frederickson and J\'{a}J\'{a}~\cite{FJ81} in Figure \ref{fjvsfdt}.  
\cindy{We should say something about running time.  Since we showed running time in part a, the reader will just assume otherwise that our running time is terrible compared to the Fredrickson and J\'{a}J\'{a} algorithm.}
%\cindy{It may be too late now, but it might be easier to understand the results if there is another figure that shows instance number vs the ratio  FDT gap/FJ gap.  It's hard to pair corresponding points with the current figure.  I would suggest replacing the current figure with that, but then we'd lose the absolute gap values.}
%\cindy{Please check my new caption for Figure 2b.  The old one is just commented out.}


\begin{table}[h!]
	\begin{small}
		\centering
		\begin{tabular}{c c c c}
			\hline
			& number of edges in $T$ & number of links in $L$ & number of instances $(T,L)$\\ \hline
			 & $6$ & $6$ & $100$ \\ 
			 & $14$ & $28$ & $100$ \\ 
			 & $30$ & $120$ & $100$ \\
			 & $62$ & $496$ & $100$ \\ 
			  & $126$ & $2016$ & $100$ \\  \hline \\
		\end{tabular}\caption{Summary of the randomly generated instances of TAP.} \label{tableTAP}
	\end{small}
\end{table}
%\cindy{Please check the edits to the next paragraph carefully to ensure I captured what you were trying to say.}
For all 500 instances in our experiments, running FDT on the LP optimal (fractional extreme point) of the cut LP gave a feasible solution with value at most a factor $\frac{3}{2}$ larger than the LP lower bound.  Such a feasible solution gives an upper bound on the integrality gap $g(\tap)$ of that specific instance of at most $\frac{3}{2}$.
%\cindy{There are 450 instances, so you tested one extreme point (the LP optimal) per instance?  I'm assuming that was the LP optimal.}
%when restricted to the instances considered.
In fact, the integrality gap upper bound derived this way was equal to $\frac{3}{2}$ for only one instance.
%the upper bound we provided by our experiment is at $\frac{3}{2}$.
For 480 instances, the integrality gap upper bound was $\frac{4}{3}$, for 16 instances it was $\frac{6}{5}$, for 2 instances it was $\frac{8}{7}$, for 1 instance it was $\frac{10}{9}$. 
\arash{it's all correct, I'm running the experiments again to get the running time so these numbers might change.}
\paragraph{Computational comparison between Christofides' algorithm and FDT for 2EC on Carr-Vempala points.} 

We first describe the polyhedral version of Christofides' algorithm specifically for Carr-Vempala points.
\cindy{I think it would be very helpful for the reader for you to remind them of the classic (graph-based, not polyhedral) metric (triangle-inequality) Christofides algorithm here and argue that it applies to 2EC.  Most people will remember Christofides' TSP approximation algorithm because it's so classic.  But it's easy to remind people.  Then it would be helpful to explain that when you give the LP in (\ref{ojoinaverage}) how that relates to Christofides.  If this paper is accepted, all kinds of CS/OR researchers may want to read it, so we should make their job as easy as possible.  It would be nice for others to use FDT and to build off it.}
Let $x$ be a Carr-Vempala point defined on a graph $G=(V,E)$. It is well known that $\frac{|V|-1}{|V|}\cdot x$ is in the convex hull of incidence vectors of spanning trees of $G$.
\cindy{Is this for any feasible solution, or extreme point of the subtour LP or just Carr-Vempala points?  It seems that it's for any feasible point since you don't remind the reader about Carr-Vempala points until later.}
\cindy{The statement before this depends very much on the audience.  For those who study polyhedral combinatorics especially for TSP and who know what a Carr-Vempala point is off the top of their heads, it may be well known.  For people who generally do combinatorial optimization, I would not expect them to know this. So this kind of phrasing might annoy a reader.  General rule: either a statement is really obvious (so just state it), or it is not (so give a reference or justification). Otherwise (and I know you don't intend this), you are implying the reader is stupid if they don't see the argument right away. Of course, it's the authors' job to teach and convince the reader. Thus one should avoid phrases like ``It's obvious/trivial that...'' and similar phrases. I would change the sentence before this comment to ``The vector $\frac{|V|-1}{|V|}\cdot x$ is in the convex hull...'' and give a short justification or reference. Or ``Foo and Barr (cite Foo and Bar's paper) argue that ... is in the convex hull...'' I'm sure I could figure this out if a had a little time to think, but I don't have that time now.}
\cindy{I'm going to remove things like ``It's easy to see that,'' where I can. But we may need to add a justification or reference.}
Hence, we can write $\frac{|V|-1}{|V|}\cdot x= \sum_{i=1}^{k}\lambda_i\chi^{T_i}$ where $T_i$ is spanning tree of $G$, $\chi^{T_i}$ is its incidence vector, $\sum_{i=1}^{k}\lambda_i=1$, and $\lambda_i\geq 0$ for $i\in [k]$. Let $O_i$ be the set of odd-degree vertices of $T_i$. We have $\frac{x}{2}$ is in the convex hull of incidence vectors of $O_i$-join of $G$, denoted $O_i\join(G)$~\footnote{For graph $G=(V,E)$ and $O\subseteq V$ with $|O|$ even, an $O$-join of $G$ is a subgraph of $G$ that has odd degree on the vertices in $O$ and even degree on vertices in $V\setminus O$.}.
\cindy{There was another instance of ``It's easy to see that...'' Given that (quite reasonably) you do not expect the reader to know what an O-join is, you should probably just give the short justification for this statement and remove ``It's easy to see that.''}


\cindy{Do you have to precompute the joins (If they are more than the spanning trees themselves)?  Does that mean generating spanning trees before setting up the LP?  Or generating them on the fly?  Only minimum-weight spanning trees?  It seems that with a little more explanation, a reader who has not been immersed in 2EC and TSP could follow this.}

 We solve the following LP that allows us to find parity corrections that are good for the whole convex combination.
\begin{equation}\label{ojoinaverage}
\min \{ \alpha:\;\sum_{i=1}^{k} \lambda_i y^i = \alpha \cdot x,\;  y^i \in O_i\join(G) \; \mbox{for $i\in [k]$}\}.
\end{equation}
The variables in the above LP are $y^i\in \mathbb{R}^{E_x}_{
\geq 0}$ for $i\in [k]$. For each $i\in [k]$ we have $y^i\in O_i\join(G_x)$.
\cindy{You say the $y^i$ are LP variables, not integers. It even looks like they can even be more than $1$. What does it mean for a non-binary vector to be in a set of binary variables?  I would have expected the lambdas to be the variables, given their use as the convex combination multipliers in most of the paper.}
This formulation allows the instance-specific approximation ratio of Christofides' algorithm to be below $\frac{3}{2}$. Recall that a Carr-Vempala point consists of a single cycle of fractional edges.  Figure \ref{fdtvschris} shows FDT's solutions on all Carr-Vempala points that have 10 vertices on the cycle formed by fractional edges. We show for these points the apporoximation factor provided by FDT is always better than those from the polyhedral version of Christofides' algorithm. In Figure \ref{fdtvschris} the horizontal axis of the plot is indexed with the 60 Carr-Vempala points that we considered. For each Carr-Vempala point $x$, there are two data points. The value of the first data point depicted by a circle on the vertical axis is $\frac{|V|-1}{|V|}+\alpha$  and $\alpha$ is the optimal solution to (\ref{ojoinaverage}). \cindy{TODO: say when giving the LP~(\ref{ojoinaverage}) how that gives a solution to TAP with this value.}
The value of the second data point depicted by a cross on the vertical axis is $C$ where $C$ is obtained from applying Theorem \ref{FDT2EC} to $x$. In other words, Figure \ref{fdtvschris} is comparing the instance-specific upper bound on integrality gap certified by Christofides' algorithm to the approximation factor of the FDT algorithm for 2EC.

\begin{figure}
\centering
\begin{tikzpicture}[scale=0.4]


\draw [dashed] [black, line width=0.2mm] plot [smooth, tension=0] coordinates {(0,4) (0,-4)};
\draw [dashed] [black, line width=0.2mm] plot [smooth, tension=0] coordinates {(2.83,2.83) (-4,0)};
\draw [dashed] [black, line width=0.2mm] plot [smooth, tension=0] coordinates {(4,0) (-2.83,-2.83)};
\draw [dashed] [black, line width=0.2mm] plot [smooth, tension=0] coordinates {(-2.83,2.83) (2.83,-2.83)};

\draw [-] [black, line width=0.2mm] plot [smooth, tension=0] coordinates {(0,4) (2.83,2.83)};

\draw [-] [black, line width=0.2mm] plot [smooth, tension=0] coordinates {(4,0) (2.83,2.83)};


\draw [-] [black, line width=0.2mm] plot [smooth, tension=0] coordinates {(4,0) (2.83,-2.83)};

\draw [-] [black, line width=0.2mm] plot [smooth, tension=0] coordinates {(0,-4) (2.83,-2.83)};

\draw [-] [black, line width=0.2mm] plot [smooth, tension=0] coordinates {(0,-4) (-2.83,-2.83)};

\draw [-] [black, line width=0.2mm] plot [smooth, tension=0] coordinates {(-4,0) (-2.83,-2.83)};

\draw [-] [black, line width=0.2mm] plot [smooth, tension=0] coordinates {(-4,0) (-2.83,2.83)};

\draw [-] [black, line width=0.2mm] plot [smooth, tension=0] coordinates {(0,4) (-2.83,2.83)};


\draw[black,fill=white] (0,4) ellipse (0.5 cm  and 0.5 cm);
\draw[black,fill=white] (4,0) ellipse (0.5 cm  and 0.5 cm);
\draw[black,fill=white] (0,-4) ellipse (0.5 cm  and 0.5 cm);
\draw[black,fill=white] (-4,0) ellipse (0.5 cm  and 0.5 cm);
\draw[black,fill=white] (2.83,2.83) ellipse (0.5 cm  and 0.5 cm);
\draw[black,fill=white] (-2.83,-2.83) ellipse (0.5 cm  and 0.5 cm);
\draw[black,fill=white] (2.83,-2.83) ellipse (0.5 cm  and 0.5 cm);
\draw[black,fill=white] (-2.83,2.83) ellipse (0.5 cm  and 0.5 cm);

\node (1) at (0,4) {{1}};
\node (2) at (2.83,2.83) {{2}};
\node (3) at (4,0) {{3}};
\node (4) at (2.83,-2.83) {{4}};
\node (5) at (0,-4) {{5}};
\node (6) at (-2.83,-2.83) {{6}};
\node (7) at (-4,0) {{7}};
\node (8) at (-2.83,2.83) {{8}};
\end{tikzpicture}
\caption{A Carr-Vempala point with 8 vertices on its cycle. Solid lines are edges with value strictly between $0$ and $1$. Dashed edges represent paths where each edge on the path has $x_e =1$. There can be an arbitrary number of degree-$2$ vertices on the dashed paths.}
\label{fig:CVpoint}
\end{figure}

\begin{figure}[h!]
	\centering
	\includegraphics[width=9cm,scale=1.4]{christofides-vs-fdt.png}
	\caption{Polyhedral version of Christofides' algorithm vs FDT on all Carr-Vempala points that have 10 vertices on the single cycle formed by fractional edges.}
	\label{fdtvschris}
\end{figure}
\paragraph{FDT for 2EC on Carr-Vempala points.}
We ran FDT for 2EC on 963 fractional extreme points of $\subtour(G)$. We enumerated all (fractional) Carr-Vempala points with $10$ and $12$ vertices. Table \ref{table2EC} shows that again FDT found solutions better than the integrality-gap lower bound for most instances. 
\begin{table}[h!]
	\begin{small}
		\centering
		\begin{tabular}{c c c c c}
			\hline
			& $C\in [1.08,1.11]\;$ & $\;C\in (1.11,1.14]\;$ &
			$\;C\in (1.14,1.17]$ &\; $C\in (1.17,1.2]\;$ \\ \hline
			2EC & $79$ & $201$ & $605$ & $78$ \\ \hline\\
		\end{tabular}	\caption{FDT for $\2ec$ implemented applied to all Carr-Vempala with 10 or 12 vertices. A Carr-Vempala point with $k$ vertices has $\frac{3k}{2}$ edges. Thus, the upper bound provided by Theorem \ref{FDT2EC} is $g(\2ec)^{3k/2}$. The lower bound on $g(\2ec)$ is $\frac{6}{5}$.}
		\label{table2EC}
	\end{small}
\end{table}





\bibliographystyle{abbrv}
\bibliography{FDT}


\end{document}



	