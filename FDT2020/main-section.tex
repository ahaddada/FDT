
\iffalse{
	\subsection{Notation}
	For vectors $x,y\in \mathbb{R}_{n}$ we say $x$ dominates $y$ if $x_i\geq y_i$ for $i= 1,\ldots,n$. For $m\times n$ matrix $A$, let $A_j$ be the $j$-th row of $A$ and $A^j$ be the $j$-th column of $A$. For a set $S$ of vectors in $\mathbb{R}_{n}$, $\conv(S)$ is the convex hull of all the points in $S$.
}\fi
\section{Finding a Feasible Solution}\label{domTOIP}
Consider an instance $I=(A,b)$ of the IP formulation. Define sets $S(I)$ and $P(I)$ as in (\ref{S}) and (\ref{P}), respectively. Assume $S(I)\subseteq \{0,1\}^n$ and $P(I)\subseteq [0,1]^n$. For simplicity in the notation we denote $P(I),S(I),$ and $g(I)$ with $P$, $S$, and $g$ for this section and the next section. Also, for both sections we assume $t=|\spp(x)|$. Without loss of generality we can assume $x_i = 0$ for $i=t+1,\ldots,n$.

In this section we prove Theorem \ref{domtoIP}. In fact, we prove a stronger result. 
\begin{lemma}\label{domlemma}
	Given $\tilde{x}\in \dom(P)$ and $\tilde{x}\in \{0,1\}^n$, there is an algorithm (the DomToIP algorithm) that finds $\bar{x}\in S$ in polynomial time, such that $\bar{x}\leq \tilde{x}$.\end{lemma}
Notice that Lemma \ref{domlemma} implies Theorem \ref{domtoIP}, since it is easy to obtain an integer point in $\dom(P)$: rounding up any fractional point in $P$ gives us a point in $\dom(P)$.

\subsection{Proof of Lemma \ref{domlemma}: The DomToIP Algorithm}

We start by introducing an algorithm that ``fixes" the variables iteratively, starting from the first coordinate and ending at the $t$-th coordinate. Suppose we run the algorithm for $\ell\in \{0,\ldots,t-1\}$ iterations and in each iteration we find $x^{(\ell)}\in \dom(P)$  such that $x^{(\ell)}_i\in \{0,1\}$ for $i=1,\ldots,\ell$. Notice that we can set $x^{(0)}=\tilde{x}$. Now consider the following linear program. The variables of this linear program are the $z\in \mathbb{R}^n$ variables.
\begin{align}
\DOMtoS(x^{(\ell)})\quad\quad& \min\quad \;z_{\ell+1}\\
&\;\text{s.t.} \quad \;\;Az\geq b \\
&\;{\color{white}{\text{s.t.}} }\quad \;\; \; z_j = x^{(\ell)}_j \quad \; j =1,\ldots, \ell\\
&\;{\color{white}{\text{s.t.}} }\quad \; \;\; z_j \leq x^{(\ell)}_j \quad \; j = \ell+1,\ldots,n\\
&\;{\color{white}{\text{s.t.}} }\quad \; \;\; z\;\geq 0
\end{align}

If the optimal value to $\DOMtoS(x^{(\ell)})$ is 0, then let $x^{(\ell+1)}_{\ell+1} = 0$. Otherwise if the optimal value is strictly positive let $x^{(\ell+1)}_{\ell+1} = 1$. Let $x^{(\ell+1)}_j = x^{(\ell)}_j$ for $j\in [n]\setminus \{\ell+1\}$ (See Algorithm \ref{domtoIPalg}).

The above procedure suggests how to find $x^{(\ell+1)}$ from $x^{(\ell)}$. The DomToIP algorithm initializes with $x^{(0)}=\tilde{x}$ and  iteratively calls this procedure in order to obtain $x^{(t)}$. 

\vspace*{10pt}
\begin{algorithm}[h]
	\KwIn{$\tilde{x}\in \dom(P)$, $\tilde{x}\in \{0,1\}^n$ }
	\KwOut{$x^{(t)} \in S$, $x^{(t)}\leq \tilde{x}$}
	$x^{(0)}\leftarrow \tilde{x}$\\
	\For{$\ell = 0$ \textbf{to} $t-1$}{
		$x^{(\ell+1)} \leftarrow x^{(\ell)}$\\
		$\eta \leftarrow$ optimal value of $ \DOMtoS(x^{(\ell)})$\\
		\eIf{$\eta = 0$}{
			$x^{(\ell+1)}_{\ell+1} \leftarrow 0$\
		}{
			$x^{(\ell+1)}_{\ell+1} \leftarrow 1$
		}
	}
	\caption{The DomToIP algorithm}
	\label{domtoIPalg}
\end{algorithm}
\vspace*{10pt}

We prove that indeed $x^{(t)}\in S$. First, we need to show that in any iteration $\ell=  0,\ldots,t-1$ of DomToIP that $\DOMtoS(x^{(\ell)})$ is feasible. We show something stronger. For $\ell=0,\ldots,t-1$ let
\begin{align*}
\LP^{(\ell)}&= \{z\in P\; : \; z\leq x^{(\ell)} \mbox{ and } z_j=x_j^{(\ell)} \mbox{ for } j\in [\ell]\}, \text{ and}\\
\IP^{(\ell)}&= \{z\in \LP^{(\ell)}\; : \; z\in \{0,1\}^n\}.
\end{align*}
Notice that if $\LP^{(\ell)}$ is a non-empty set then $\DOMtoS(x^{(\ell)})$ is feasible. We show by induction on $\ell$ that $\LP^{(\ell)}$ and $\IP^{(\ell)}$ are not empty sets for $\ell=0,\ldots,t-1$. First notice that $\LP^{(0)}$ is clearly feasible since by definition $x^{(0)}\in \dom(P)$, meaning there exists $z\in P$ such that $z\leq x^{(0)}$. By Theorem \ref{CV2}, there exists $\tilde{z}^i\in S$ and $\theta_i\geq 0$ for $i\in [k]$ such that $\sum_{i=1}^{k} \theta_i = 1$ and $\sum_{i=1}^{k}\theta_i \tilde{z}^i \leq gz$. Hence, $\sum_{i=1}^{k}\theta_i \tilde{z}^i \leq gz\leq gx^{(0)}$. So if $x^{(0)}_j=0$, then $ \sum_{i=1}^{k}\theta_i \tilde{z}_j^i =0$, which implies that $\tilde{z}^i_j=0$ for all $i\in [k]$ and $j\in [n]$ where $x^{(0)}_j=0$. Hence, $\tilde{z}^i\leq x^{(0)}$ for $i\in [k]$. Therefore $\tilde{z}^i\in \IP^{(0)}$ for $i\in [k]$, which implies $\IP^{(0)}\neq \emptyset$.

Now assume $\IP^{(\ell)}$ is non-empty for some $\ell \in [t-2]$. Since $\IP^{(\ell)}\subseteq\LP^{(\ell)}$ we have $\LP^{(\ell)}\neq \emptyset$ and hence the $\DOMtoS(x^{(\ell)})$ has an optimal solution $z^*$.

We consider two cases. In the first case, we have $z^*_{\ell+1}=0$. In this case we have $x^{(\ell+1)}_{\ell+1}=0$. Since $z^*\leq x^{(\ell+1)}$, we have $z^*\in \LP^{(\ell+1)}$. Also, $z^*\in P$. By Theorem \ref{CV2} there exists $\tilde{z}^i\in S$ and $\theta_i\geq 0$ for $i\in [k]$ such that $\sum_{i=1}^{k} \theta_i = 1$ and  $\sum_{i=1}^{k}\theta_i \tilde{z}^i \leq gz^*$. We have $\sum_{i=1}^{k}\theta_i \tilde{z}^i \leq gz^*\leq gx^{(\ell+1)}$.
So for $j\in [n]$ where $x^{(\ell+1)}_j=0$, we have $z^i_j=0$ for $i\in [k]$. This implies $\tilde{z}^i\leq x^{(\ell+1)}$ for $i=1,\ldots,k$. Hence, there exists $z\in S$ such that $z\leq x^{(\ell+1)}$. We claim that $z\in \IP^{(\ell+1)}$. If $z\notin \IP^{(\ell+1)}$ we must have $1\leq j \leq \ell$ such that $z_j < x^{(\ell+1)}_{j}$, and thus $z_j = 0$ and $x^{(\ell+1)}_j=1$. Without loss of generality assume $j$ is minimum number satisfying $z_j < x^{(\ell+1)}_{j}$. Consider iteration $j$ of the DomToIP algorithm. Notice that $z\leq x^{(\ell+1)}\leq x^{(j)}$. We have $x^{(j)}_j=1$ which implies when we solved $\DOMtoS(x^{(j-1)})$ the optimal value was strictly larger than zero. However, $z$ is a feasible solution to $\DOMtoS(x^{(j-1)})$ and gives an objective value of 0. This is a contradiction, so $z\in \IP^{(\ell+1)}$.

Now for the second case, assume $z^*_{\ell+1} > 0$. We have $x^{(\ell+1)}_{\ell+1}=1$. Notice that for each point $z\in \LP^{(\ell)}$ we have $z_{\ell+1} >0$, so for each $z\in \IP^{(\ell)}$ we have $z_{\ell+1}>0$, i.e. $z_{\ell+1}=1$. This means that $z\in \IP^{(\ell+1)}$, and $\IP^{(\ell+1)} \neq \emptyset$.

Now consider $x^{(t)}$. Let $z$ be the optimal solution to $\LP^{(t-1)}$. If $x^{(t)}_t = 0$, we have $x^{(t)} = z$, which implies that $x^{(t)}\in P$, and since $x^{(t)}\in \{0,1\}^n$ we have $x^{(t)}\in S$. If $x^{(t)}_t =1$, it must be the case that $z_t > 0$. By the argument above there is a point $z'\in \IP^{(t-1)}$. We show that $x^{(t)} = z'$. For $j\in [t-1]$ we have $z'_j= x_j^{(t-1)}=x_j^{(t)}$. We just need to show that $z'_t = 1$. Assume $z'_t	 = 0$ for contradiction, then $z'\in \LP^{(t-1)}$ has objective value of $0$ for $\DOMtoS(x^{(t-1)})$, this is a contradiction to $z$ being the optimal solution. This concludes the proof of Lemma \ref{domlemma}. 





\section{FDT on Binary IPs}
\label{binaryfdt}

Assume we are given a point $x^*\in P$. For instance, $x^*$ can be the optimal solution of minimizing a cost function $cx$ over set $P$, which provides a lower bound on $\min_{(x,y)\in S(I)} cx$.  In this section, we prove Theorem \ref{binaryFDT} by describing the Fractional Decomposition Tree (FDT) algorithm. We also remark that if $g(I)=1$, then the algorithm will give an exact decomposition of any feasible solution. 


The FDT algorithm grows a tree similar to the classic branch-and-bound search tree for integer programs. Each node represents a partially integral vector $\bar{x}$ in $\dom(P)$ together with a multiplier $\bar{\lambda}$. The solutions contained in the nodes of the tree become progressively more integral at each level. In each level of the tree, the algorithm maintain a conic combination of points with the properties mentioned above. Leaves of the FDT tree contain solutions with integer values for all the $x$ variables that dominate a point in $P$. In Lemma  \ref{domlemma} we saw how to turn these into points in $S$. 

\paragraph{Branching on a node.}
We begin with the following lemmas that show how the FDT algorithm branches on a variable.
\begin{lemma}\label{LPClemma}
	Given $x'\in \dom(P)$ and $\ell\in [n]$ where $x'_{\ell}<1$, we can find in polynomial time vectors $\hat{x}^0,\hat{x}^1$ and scalars $\gamma_0,\gamma_1 \in [0,1]$ such that: (i) $\gamma_0 + \gamma_1  \geq 1/g$, (ii) $\hat{x}^0$ and $\hat{x}^1$ are in  $ P$
	,(iii) $\hat{x}^0_\ell=0$ and $\hat{x}^1_\ell=1$, (iv) $\gamma_0 \hat{x}^0 + \gamma_1\hat{x}^1 \leq x'$.
\end{lemma}


\begin{proof}	
	Consider the following linear program which we denote by $\LPC(\ell,x')$. The variables of $\LPC(\ell,x')$ are $\gamma_0,\gamma_1$ and $x^0$ and $x^1$. 
	\begin{align}
	\LPC(\ell,x')\quad\quad& \max\quad \;\lambda_0+\lambda_1\\
	&\;\text{s.t.} \quad Ax^j \geq b\lambda_j & \mbox{ for $j=0,1$} \label{feasibility}\\
	&\;{\color{white}{\text{s.t.}} }\quad 0 \leq x^j \leq \lambda_j &\mbox{ for $j=0,1$}\label{bound}\\
	&\;{\color{white}{\text{s.t.}} }\quad x^0_\ell = 0,\; x^1_\ell =\lambda_1\label{branchcoordinate}\\
	&\;{\color{white}{\text{s.t.}} }\quad x^0 + x^1 \leq x'\label{packing}\\
	&\;{\color{white}{\text{s.t.}} }\quad \lambda_0,\lambda_1 \geq 0
	\end{align}
	
	Let $x^0,x^1$, and $\gamma_0,\gamma_1$ be an optimal solution to the LP above. Let $\hat{x}^0 = x^0/\gamma_0$, $\hat{x}^1=x^1/\gamma_1$. This choice satisfies  (ii), (iii), (iv). To show that (i) is also satisfied we prove the following claim.
	
	\begin{claim}\label{CVexists}
		We have $\gamma_0 + \gamma_1\geq 1/g$.
	\end{claim}
	\begin{cproof}
		We show that there is a feasible solution that achieves the objective value of $\frac{1}{g}$. By Theorem \ref{CV2} there exists $\theta \in [0,1]^k$, with $\sum_{i=1}^{k}\theta_i = 1$ and $\tilde{x}^i\in S$ for $i\in[k]$ such that 
		$\sum_{i=1}^{k}\theta_i \tilde{x}^i\leq gx'$. So
		
		\begin{equation}\label{splitting}
		x'\geq \sum_{i=1}^{k}\frac{\theta_i}{g} \tilde{x}^i
		={\sum_{i\in [k]: \tilde{x}^i_\ell =0}\frac{\theta_i}{g} \tilde{x}^i}+{\sum_{i\in [k]: \tilde{x}^i_\ell =1}\frac{\theta_i}{g} \tilde{x}^i}.
		\end{equation}
		For $j=0,1$, let $x^j = \sum_{i\in [k]:\tilde{x}^i_\ell=j} \frac{\theta_i}{g}\tilde{x}^i$. Also let $\lambda_0=\sum_{i\in [k]: \tilde{x}^i_\ell =0}\frac{\theta_i}{g}$ and $\lambda_1 = \sum_{i\in [k]: \tilde{x}^i_\ell =1}\frac{\theta_i}{g}$. Note that $\lambda_0+\lambda_1 =1/g$. Constraint (\ref{packing}) is satisfied by Inequality (\ref{splitting}). Also, for $j=0,1$ we have
		\begin{equation}
		Ax^j= \sum_{i\in[k], \tilde{x}^i_\ell = j} \frac{\theta_i}{g} A\tilde{x}^i  \geq b \sum_{i\in[k], \tilde{x}^i_\ell = j} \frac{\theta_i}{g} = b\lambda_j.
		\end{equation}
		Hence, Constraints (\ref{feasibility}) holds. Constraint (\ref{branchcoordinate}) also holds since $x^0_\ell$ is obviously $0$ and $x^1_\ell= \sum_{i\in [k]: \tilde{x}^i_\ell = 1}\frac{\theta_i}{g}= \lambda_1$. The rest of the constraints trivially hold. 
	\end{cproof}
	This concludes the proof of Lemma \ref{LPClemma}.	
\end{proof}

We now show if $x'$ in the statement of Lemma \ref{LPClemma} is partially integral, we can find solutions with more integral components.
\begin{lemma}\label{round-up}
	Given $x'\in \dom(P)$ where $x'_1,\ldots,x'_{\ell-1}\in \{0,1\}$ and $x'_{\ell}<1$ for some $\ell\geq 1$ we can find in polynomial time vectors $\hat{x}^0,\hat{x}^1$ and scalars $\gamma_0,\gamma_1 \in [0,1]$ such that: (i) ${ 1}/{g}\leq \gamma_0 + \gamma_1  \leq 1$, (ii) $\hat{x}^0$ and $\hat{x}^1$ are in  $\dom( P)$, (iii) $\hat{x}^0_\ell=0$ and $\hat{x}^1_\ell=1$, (iv) $ \gamma_0\hat{x}^0 +\gamma_1 \hat{x}^1 \leq
	x'$,(v) $\hat{x}^i_j\in \{0,1\}$ for $i=0,1$ and $j\in[\ell-1]$.
\end{lemma} 
\begin{proof}
	By Lemma \ref{LPClemma} we can find $\bar{x}^0$, $\bar{x}^1$, $\gamma_0$ and $\gamma_1$ that satisfy (i), (ii), (iii), and (iv). We define $\hat{x}^0$ and $\hat{x}^1$ as follows. For $i=0,1$, for $j\in[\ell-1]$, let $\hat{x}^i_j= \ceil{\bar{x}^i_j}$, for $j=\ell,\ldots,t$ let $\hat{x}^i_j = \bar{x}^i_j$.
	
	
	We now show that $\hat{x}^0$, $\hat{x}^1$, $\gamma_0$, and $\gamma_1$ satisfy all the conditions. Note that conditions (i), (ii), (iii), and (v) are trivially satisfied. Thus we only need to show (iv) holds. We need to show that $\gamma_0 \hat{x}^0_j+\gamma_1\hat{x}^1_j\leq gx'_j$. If $j=\ell,\ldots,t$, then this clearly holds. Hence, assume $j\leq \ell-1$. By the property of $x'$ we have $x'_j\in \{0,1\}$. If $x'_j= 0$, then by Constraint (\ref{packing}) we have $\bar{x}^0_j = \bar{x}^1_j=0$. Therefore, $\hat{x}^i_j=0$ for $i=0,1$, so (iv) holds. Otherwise if $x'_j = 1$, then we have
	$\gamma_0\hat{x}^0_j+\gamma_1\hat{x}^1_j\leq \gamma_0+\gamma_1\leq 1\leq x'_j.$ 
	Therefore (v) holds.
\end{proof}

\paragraph{Growing and Pruning FDT tree.} The FDT algorithm maintains nodes $L_i$ in iteration $i$ of the algorithm. The nodes in $L_i$ correspond to the nodes in level $L_i$ of the FDT tree. The points in the leaves of the FDT tree, $L_t$, are points in $\dom(P)$ and are integral for all integer variables.


\begin{lemma}\label{prune}
	There is a polynomial time algorithm that produces sets $L_0,\ldots,L_t$ of pairs of $x\in \dom(P)$ together with multipliers $\lambda$ with the following properties for $i=0,\ldots,t$:
	(a) If $x\in L_i$, then $x_j \in \{0,1\}$ for $j\in [i]$, i.e. the first $i$ coordinates of a solution in level $i$ are integral, (b) $\sum_{[x,\lambda]\in L_i} \lambda\geq\frac{1}{g^i}$, (c) $\sum_{[x,\lambda]\in L_i}\lambda x \leq x^*$, (d) $|L_i|\leq t$.
\end{lemma}
\begin{proof}
	We prove this lemma using induction but one can clearly see how to turn this proof into a polynomial time algorithm. Let $L_0$ be the set that contains a single node (\textit{root of the FDT tree}) with $x^*$ and multiplier 1. It is easy to check all the requirements in the lemma are satisfied for this choice.
	
	Suppose by induction that we have constructed sets $L_0,\ldots,L_i$. Let the solutions in $L_i$ be $x^j$ for $j\in [k]$ and $\lambda_j$ be their multipliers, respectively. For each $j\in[k]$ if $x^j_{i+1}=1$ we add the pair $(x^j,\lambda_j)$ to $L'$. Otherwise,	applying Lemma \ref{round-up} (setting $x'= x^j$ and $\ell = i+1$) we can find $x^{j0}$, $x^{j1}$, $\lambda^0_j$ and $\lambda^1_j$ with the properties (i) to (v) in Lemma \ref{round-up}. Add the pairs  $(x^{j0} ,\lambda_j\lambda^0_j)$ and  $(x^{j1} ,\lambda_j\lambda^1_j)$ to $L'$. It is easy to check that set $L'$ is a suitable candidate for $L_{i+1}$, i.e. set $L'$ satisfies (a), (b) and (c). However we can only ensure that $|L'|\leq 2k\leq 2t$, and might have $|L'|>t$. We call the following linear program $\prun(L')$. Let $L' = \{[x^1,\gamma_1],\ldots,[x^{|L'|},\gamma_{|L'|}]\}$. The variables of $\prun(L')$ are scalar variables $\theta_j$ for each node $j$ in $L'$.  
		\begin{equation}
		\prun(L')\quad\quad\quad \{\max \sum_{j=1}^{|L'|} \theta_j\;:\; \sum_{j=1}^{|L'|} \theta_j x^j_i\leq x^*_i \mbox{ for $i\in [t]$},\; \theta\geq 0 \}
		\end{equation}
		
		Notice that $\theta = \gamma$ is in fact a feasible solution to $\prun(L')$. Let $\theta^*$ be the optimal vertex solution to this LP. Since the problem is in $\mathbb{R}^{|L'|}$,  $\theta^*$ has to satisfy $|L'|$ linearly independent constraints at equality. However, there are only $t$ constraints of type $ \sum_{j=1}^{|L'|} \theta_j x^j_i\leq x^*_i$. Therefore, there are at most $t$ coordinates of $\theta^*_j$ that are non-zero. Set $L_{i+1}$ which consists of $x^j$ for $j=1,\ldots,|L'|$ and their corresponding multipliers $\theta^*_j$ satisfy the properties in the statement of the lemma. Notice that, we can discard the nodes in $L_{i+1}$ that have $\theta^*_j=0$, so $|L_{i+1}| \leq t$. Also, since $\theta^*$ is optimal and $\gamma$ is feasible for $\prun(L')$, we have $\sum_{j=1}^{|L'|} \theta^*_j \geq \sum_{j=1}^{|L'|}\gamma_j \geq \frac{1}{g^{i+1}}$. 
	
	\end{proof}
	
	\paragraph{From leaves of FDT to feasible solutions.}
	For the leaves of the FDT tree,  $L_t$, we have that every solution $x$ in $L_t$ has $x\in\{0,1\}^n$ and $x\in \dom(P)$. By applying Lemma \ref{domlemma} we can obtain a point $x'\in S$ such that $x'\leq x$. This concludes the description of the FDT algorithm and proves Theorem \ref{binaryFDT}. See Algorithm \ref{FDTFull} for a summary of the FDT algorithm.
	
	\vspace*{10pt}
	
	\begin{algorithm}[H]\label{FDTFull}
		\KwIn{$P= \{x\in \mathbb{R}^{n}: Ax\geq b\}$ and $S=\{x\in P: x\in \{0,1\}^n\}$ such that $g=\max_{c\in \mathbb{R}^n_+ }\frac{\min_{x\in S}cx}{\min_{x\in P}cx}$ is finite, $x^*\in P$}
		\KwOut{$z^i\in S$ and $\lambda_i\geq 0$ for $i\in[k]$ such that $\sum_{i=1}^{k}\lambda_i = 1$, and $\sum_{i=1}^{k}\lambda_iz^i\leq g^tx^*$ }
		$L^0\leftarrow [x^*,1]$\\
		\For{$i=1$ \textbf{to} $t$}{
			$L'\leftarrow \emptyset$\\
			\For{$[x,\lambda] \in L^i$}{
				Apply Lemma \ref{round-up} to obtain $[\hat{x}^0,\gamma_0]$ and $[\hat{x}^1,\gamma_1]$\\
				$L' \leftarrow L' \cup \{[\hat{x}^0,\lambda\cdot\gamma_0]\} \cup \{[\hat{x}^1,\lambda\cdot\gamma_1]\}$\\			
			}
			Apply Lemma \ref{prune} to prune $L'$ to obtain $L^{i+1}$. 
		}
		\For{$[x,\lambda] \in L^t$}{
			Apply Algorithm \ref{domtoIPalg} to $x$ to obtain $z\in S$\\
			$F \leftarrow F \cup \{[z,\lambda]\}$
		}
		\textbf{return} $F$
		\caption{Fractional Decomposition Tree Algorithm}
	\end{algorithm}
	
	


\section{FDT for 2EC}\label{2EC}

In Section \ref{binaryfdt} our focus was on binary IPs. In this section, in an attempt to extend FDT to \{0,1,2\} problems we introduce an FDT algorithm for a 2-edge-connected multigraph problem. Given a graph $G=(V,E)$ a multi-subset of edges $F$ of $G$ is a 2-edge-connected multigraph of $G$ if for each set $\emptyset\subset U \subset V$, the number of edge in $F$ that have one endpoint in $U$ and one not in $U$ is at least 2. Recall that in the 2EC, we are given non-negative costs on the edges of $G$ and the goal is to find the minimum cost 2-edge-connected multigraph of $G$. We want to prove Theorem \ref{FDT2EC}.
\FDTEC*
We do not know the exact value for $g(\2ec)$, but we know $\frac{6}{5} \leq g(\2ec) \leq \frac{3}{2}$ \cite{alexander2006integrality,wolsey}. The FDT algorithm for 2EC is very similar to the one for binary IPs, but there are some differences as well. A natural thing to do is to have three branches for each node of the FDT tree, however, the branches that are equivalent to setting a variable to $1$, might need further decomposition. That is the main difficulty when dealing with $\{0,1,2\}$-IPs.

First, we need a branching lemma. Observe that  the following branching lemma is essentially a translation of Lemma \ref{LPClemma} for $\{0,1,2\}$ problems except for one additional clause. 

\begin{restatable}{lemma}{2ECLPC}
	\label{LPC2EC}
	Given $x\in \subtour(G)$, and $e\in E$ we can find in polynomial time vectors $x^0,x^1$ and $x^2$ and scalars $\gamma_0,\gamma_1$, and $\gamma_2$ such that: (i) $\gamma_0 + \gamma_1 +\gamma_2 \geq { 1}/{g(\2ec)}$, (ii) $x^0,x^1,$ and $x^2$ are in  $ \subtour(G)$, (iii) $x^0_e=0$, $x^1_e=1$, and $x^2_e=2$, (iv) $\gamma_0 x^0 + \gamma_1{x}^1  + \gamma_2x^2\leq {x}$, (v) for $f\in E$ with ${x}_f\geq 1$, we have $x^j_f\geq 1$ for $j=0,1,2$.
\end{restatable}

\begin{proof}
	Consider the following LP with variables $\lambda_j$ and $x^j$ for $j=0,1,2$. 
	\begin{align}
	\quad\quad& \max\quad \;\sum_{j=0,1,2}\lambda_j\\
	&\;\text{s.t.} \quad x^j(\delta(U))\geq 2\lambda_j \;& \mbox{ for $\emptyset \subset U \subset V$, and $j=0,1,2$} \label{feasibility2ec}\\
	&\;{\color{white}{\text{s.t.}} }\quad 0 \leq x^j \leq 2\lambda_j\; &\mbox{ for $j=0,1,2$}\label{bound2ec}\\
	&\;{\color{white}{\text{s.t.}} }\quad x^j_e =j\cdot \lambda_j\; &\mbox{ for $j=0,1,2$}\label{branchcoordinate2ec}\\
	&\;{\color{white}{\text{s.t.}} }\quad x^j_f \geq \lambda_j \; &\mbox{ for $f\in E$ where $x_f \geq 1$, and $j=0,1,2$}\label{1edges2ec}\\
	&\;{\color{white}{\text{s.t.}} }\quad x^0 + x^1+x^2 \leq x\label{packing2ec}\\
	&\;{\color{white}{\text{s.t.}} }\quad \lambda_0,\lambda_1,\lambda_2 \geq 0
	\end{align}	Let $x^j$, $\gamma_j$ for $j=0,1,2$ be an optimal solution solution to the LP above. Let $\hat{x}^{j}={x^j}/{\gamma_j}$ for $j=0,1,2$ where $\gamma_j>0$. If $\gamma_j=0$, let $\hat{x}^{j}=0$. Observe that  (ii), (iii), (iv), and (v) are satisfied with this choice. We can also show that $\gamma_0+\gamma_1+\gamma_2\geq {1}/{g(\2ec)}$, which means that (i) is also satisfied. The proof is similar to the proof of the claim in Lemma \ref{LPClemma}, but we need to replace each $f\in E$ with $x_f\geq 1$ with a suitably long path to ensure that Constraint (\ref{1edges2ec}) is also satisfied.	
	\begin{claim}\label{CVexists}
		We have $\gamma_0 + \gamma_1+\gamma_2\geq \frac{1}{g(\2ec)}$.
	\end{claim}
	\begin{cproof}
		Suppose for contradiction $\sum_{j=0,1,2}\gamma_j = \frac{1}{g(\2ec)} - \epsilon$ for some $\epsilon >0$. Construct graph $G'$ by removing edge $f$ with $x_f\geq 1$ and replacing it with a path $P_f$ of length $\ceil{\frac{2}{\epsilon}}$. Define $x'_h = x_h$ for each edge $h$ such that $x_h<1$. For each $h\in P_f$ let $x'_h= x_f$ for all $f$ with $x_f\geq 1$. It is easy to check that $x'\in \subtour(G')$. By Theorem \ref{CV2} there exists $\theta \in [0,1]^k$, with $\sum_{i=1}^{k}\theta_i = 1$ and 2-edge-connected multigraphs $F'_i$ of $G'$ for $i=1,\ldots,k$ such that 
		$\sum_{i=1}^{k}\theta_i \chi^{F'_i}\leq g(\2ec)x'$. 
		
		Note that each $F'_i$ contains at least one copy of every edge in any path $P_f$, except for at most one edge in the path. We will obtain 2-edge-connected multigraphs $F_1,\ldots,F_k$ of $G$ using $F'_1,\ldots,F'_k$, respectively. To obtain $F_i$ first remove all $P_f$ paths from $F'_i$. Suppose there is an edge $h$ in $P_f$ such that $\chi^{F'_i}_h=0$, this means that for any edge $p\in P_f$ such that $p\neq h$, $\chi^{F'_i}_p=2$. In this case, let $\chi^{F_i}_f=2$, i.e. add two copies of $f$ to $F_i$. If there are at least one edge $h\in P_f$ with $\chi^{F'_i}_h= 1$, let $\chi^{F_i}_f=1$, i.e. add one copy of $f$ to $F_i$. If for all edges $h\in P_f$, we have $\chi^{F'_i}_h=2$, then let $\chi^{F_i}_f=2$. For $f\in E$ with $x_f<1$ we have
		\begin{equation}
		\sum_{i=1}^{k}\theta_i \chi^{F_i}_f=\sum_{i=1}^{k}\theta_i \chi^{F'_i}_f\leq g(\2ec)x'_f= g(\2ec)x_f.
		\end{equation}
		In addition for $f\in E$ with $x_f\geq 1$ we have $\chi^{F_i}_f \leq \frac{\sum_{h\in P_f}\chi^{F'_i}_h}{\ceil{\frac{2}{\epsilon}}-1}$ by construction.
		\begin{align*}
		\sum_{i=1}^{k}\theta_i \chi^{F_i}_f&\leq \sum_{i=1}^{k}\theta_i\frac{\sum_{h\in P_f}\chi^{F'_i}_h}{\ceil{\frac{2}{\epsilon}}-1}\\
		&= \frac{\sum_{h\in P_f} \sum_{i=1}^{k}\theta_i\chi^{F'_i}_h}{\ceil{\frac{2}{\epsilon}}-1}\\
		&\leq \frac{\sum_{h\in P_f} g(\2ec)x'_h}{\ceil{\frac{2}{\epsilon}}-1}\\
		&= \frac{\sum_{h\in P_f} g(\2ec)x_f}{\ceil{\frac{2}{\epsilon}}-1}\\
		&= \frac{\ceil{\frac{2}{\epsilon}}}{\ceil{\frac{2}{\epsilon}}-1}g(\2ec)x_f.
		\end{align*}
		Therefore, since $\frac{\ceil{\frac{2}{\epsilon}}}{\ceil{\frac{2}{\epsilon}}-1}\geq 1$, we have 
		\begin{equation}
		x \geq\sum_{i\in [k]: \chi^{F_i}_e=0}\frac{\theta_i(\ceil{\frac{2}{\epsilon}}-1)}{g(\2ec)\ceil{\frac{2}{\epsilon}}}\chi^{F_i}+ \sum_{i\in [k]: \chi^{F_i}_e=1}\frac{\theta_i(\ceil{\frac{2}{\epsilon}}-1)}{g(\2ec)\ceil{\frac{2}{\epsilon}}}\chi^{F_i}+\sum_{i\in [k]: \chi^{F_i}_e=2}\frac{\theta_i(\ceil{\frac{2}{\epsilon}}-1)}{g(\2ec)\ceil{\frac{2}{\epsilon}}}\chi^{F_i}.
		\end{equation}
		Let $x^j = \sum_{i\in [k]: \chi^{F_i}_e=j}\frac{\theta_i(\ceil{\frac{2}{\epsilon}}-1)}{g(\2ec)\ceil{\frac{2}{\epsilon}}}\chi^{F_i}$ and $\theta_j =  \sum_{i\in [k]: \chi^{F_i}_e=j}\frac{\theta_i(\ceil{\frac{2}{\epsilon}}-1)}{g(\2ec)\ceil{\frac{2}{\epsilon}}}$ for $j=0,1,2$. It is easy to check that $x^j$ , $\theta_j$ for $j=0,1,2$ is a feasible solution to the LP above. Notice that $\sum_{j=0,1,2}\theta_j = \frac{\ceil{\frac{2}{\epsilon}}-1}{g(\2ec)\ceil{\frac{2}{\epsilon}}}$. By assumption, we have $\frac{\ceil{\frac{2}{\epsilon}}-1}{g(\2ec)\ceil{\frac{2}{\epsilon}}}\leq  \frac{1}{g(\2ec)}-\epsilon$, which is a contradiction.
	\end{cproof}
	This concludes the proof. \end{proof}
In contrast to FDT for binary IPs where we round up the fractional variables that are already branched on at each level, in FDT for 2EC we keep all coordinates as they are and perform a rounding procedure at the end. Formally, let $L_i$ for $i=1,\ldots,|\spp(x^*)|$ be collections of pairs of feasible points in $\subtour(G)$ together with their multipliers. Let $t=|\spp(x^*)|$ and assume without loss of generality that $\spp(x^*)=\{e_1,\ldots,e_t\}$. 

\begin{lemma}\label{2ecpruning}
	The FDT algorithm for 2EC in  polynomial time produces sets $L_0,\ldots,L_t$ of pairs $x\in \2ec(G)$ together with multipliers $\lambda$ with the following properties for $i\in [t]$:
	(a) If $x\in L_i$, then $x_{e_j}=0$ or $x_{e_j}\geq 1$ for $j=1,\ldots,i$, (b) $\sum_{(x,\lambda)\in L_i }\lambda \geq \frac{1}{g(\2ec)^i}$, (c) $\sum_{(x,\lambda)\in L_i }\lambda x \leq x^*$, (d) $|L_i|\leq t$.
\end{lemma}
The proof is similar to Lemma \ref{prune}, but we need to use property (v) in Lemma \ref{LPC2EC} to prove that (a) also holds.
\begin{proof}
	We proceed by induction on $i$. Define $L_0=\{(x^*,1)\}$. It is easy to check all the properties are satisfied. Now, suppose by induction we have $L_{i-1}$ for some $i=1,\ldots,t$ that satisfies all the properties. For each solution $x^\ell$ in $L_{i-1}$ apply Lemma \ref{LPC2EC} on $x^\ell$ and $e_{i}$ to obtain $x^{\ell j}$ and $\lambda_{\ell j}$ for $j=0,1,2$. Let $L'$ be the collection that contains $(x^{\ell j},\lambda_\ell \cdot \lambda_{\ell j})$ for $j=0,1,2$, when applied to all $(x^\ell,\lambda_\ell)$ in $L_{i-1}$. Similar to the proof in Lemma \ref{prune} one can check that $L_i$ satisfies properties (b), (c). We now verify property (a). Consider a solution $x^\ell$ in $L_{i-1}$. For $e\in \{e_1,\ldots,e_{i-1}\}$ if $x^\ell_e =0$, then by property (iv) in Lemma \ref{LPC2EC} we have $x^{\ell j}=0$ for $j=0,1,2$. Otherwise by induction we have $x^{\ell}_{e}\geq 1$ in which case property (v) in Lemma \ref{LPC2EC} ensures that $x^{\ell j}_e\geq 1$ for $j=0,1,2$. Also, $x^{\ell j}_{e_i}= j$, so $x^{\ell j}_{e_i}=0$ or $x^{\ell j}_{e_i}\geq 1$ for $j=0,1,2$. 
	
	Finally, if $|L'|\leq t$ we let $L_i=L'$, otherwise apply $\prun(L')$ to obtain $L_{i}$.
\end{proof}

Consider the solutions $x$ in $L_t$. For each variable $e$ we have $x_e=0$ or $x_e\geq 1$. 
\begin{lemma}\label{rounddown}
	Let $x$ be a solution in $L_t$. Then $\floor{x} \in \subtour(G)$. 
\end{lemma}
\begin{proof}
	Suppose not. Then there is a set of vertices $\emptyset \subset U \subset V$ such that $\sum_{e\in \delta(U)}\floor{x_e}<2$. Since $x\in \subtour(G)$ we have $\sum_{e\in \delta(U)}x_e \geq 2$. Therefore, there is an edge $f\in \delta(U)$ such that $x_f$ is fractional. By property (a) in Lemma \ref{2ecpruning}, we have $1<  x_f < 2$. Therefore, there is another edge $h$ in $\delta(U)$ such that $x_h>0$, which implies that $x_h\geq 1$. But in this case $\sum_{e\in \delta(U)}\floor{x_e}\geq  \floor{x_f}+\floor{x_h}  \geq 2$. This is a contradiction.
\end{proof}

The FDT algorithm for 2EC iteratively applies Lemmas \ref{LPC2EC} and \ref{2ecpruning} to variables $x_1,\ldots,x_t$ to obtain leaf point solutions $L_t$. Finally, we just need to apply Lemma \ref{rounddown} to obtain the 2-edge-connected multigraphs from every solution in $L_t$. Notice that since $x$ is an extreme point we have $t\leq 2|V|-1$ \cite{boydpulley}. By Lemma \ref{2ecpruning} we have
\begin{align*}
\sum_{(x,\lambda)\in L_t} \frac{\lambda}{\sum_{(x,\lambda)\in L_t}\lambda} \floor{x} \leq \frac{1}{\sum_{(x,\lambda)\in L_t}\lambda} \sum_{(x,\lambda)\in L_t} \lambda {x} \leq g^t_{\2ec} x^*.
\end{align*}
\section{Computational Experiments with FDT}\label{experiment}
We ran FDT on three network design problems: VC, TAP and 2EC. 
\paragraph{FDT on VC instances from (PACE 2019) \cite{PACE}.}



\paragraph{FDT on randomly generated instances of TAP.}
Recall that in TAP we are given a tree $T=(V,E)$, and a set of links $L$ between vertices in $V$ and costs $c\in \mathbb{R}^{L}_{\geq 0}$. A feasible augmentation is $L'\subseteq L$ such that $T+L'$ is 2-edge-connected. In TAP we wish to find the minimum-cost feasible augmentation. The integrality gap of the cut-LP for TAP is defined as 
\begin{equation*}
g(\tap) = \max_{c\in \mathbb{R}^L_{\geq 0}} \frac{\min_{x\in\tap(T,L)} cx}{\min_{x\in\cut(T,L)} cx}.
\end{equation*}  

We know $\frac{3}{2}\leq g(\tap)\leq 2$~\cite{FJ81,32gaptap}. Notice that $\min_{x\in \tap(T,L)}cx$ is a binary IP. We ran binary FDT on a set of 264 fractional extreme points of randomly generated instances of $\tap$. \cindy{How big were the trees? And how many candidate links?} Table \ref{tableTAP} shows FDT found solutions better than the integrality-gap lower bound for most instances.

\begin{table}[h!]
	\begin{small}
		\centering
		\begin{tabular}{c c c c c}
			\hline
			& $C\in [1.1,1.2]\;$ & $\;C\in (1.2,1.3]\;$ &
			$\;C\in (1.3,1.4]$ &\; $C\in (1.4,1.5]\;$ \\ \hline
			TAP & $36$ & $66$ & $170$ & $10$\\  \hline \\
		\end{tabular}\caption{The scale factor $C$ for FDT run on 264 randomly generated TAP instances with fractional extreme points: 138 instances have $74$ variables. The rest have $250$.} \label{tableTAP}
	\end{small}
	
\end{table}

\paragraph{Computational comparison between Christofides' algorithm and FDT for 2EC on Carr-Vempala points.} 

First we need to describe the polyhedral version Christofides' algorithm. We do this specifically for the Carr-Vempala points. Let $x$ be a Carr-Vempala points defined on a graph $G=(V,E)$. It is well known that $\frac{|V|-1}{|V|}\cdot x$ is in the convex hull of incidence vectors of spanning trees of $G$. Hence, we can write $\frac{|V|-1}{|V|}\cdot x= \sum_{i=1}^{k}\lambda_i\chi^{T_i}$ where $T_i$ is spanning tree of $G$, $\sum_{i=1}^{k}\lambda_i=1$, and $\lambda_i\geq 0$ for $i\in [k]$. Let $O_i$ be the set of odd degree vertices of $T_i$. It is easy to see that $\frac{x}{2}$ is in the convex hull of incidence vectors of $O_i$-join of $G$, namely $O_i\join(G)$~\footnote{For graph $G=(V,E)$ and $O\subseteq V$ with $|O|$ even, and $O$-join of $G$ is a subgraph of $G$ that has odd degree on the vertices in $O$ and even degree on vertices in $V\setminus O$.}. 

 We then solve the following LP that allows us to find parity corrections that are good for the whole convex combination.
\begin{equation}\label{ojoinaverage}
\min \{ \alpha:\;\sum_{i=1}^{k} \lambda_i y^i = \alpha \cdot x,\;  y^i \in O_i\join(G) \; \mbox{for $i\in [k]$}\}.
\end{equation}
The variables in the above LP are $y^i\in \mathbb{R}^{E_x}_{
\geq 0}$ for $i\in [k]$. For each $i\in [k]$ we have $y^i\in O_i\join(G_x)$. This formulation allows the instance specific approximation ratio of Christofides' algorithm to be below $\frac{3}{2}$. Recall that a Carr-Vempala point consists of a Hamiltonian cycle of fractional edges.  Figure \ref{fdtvschris} shows FDT's solutions on all Carr-Vempala points defined on graphs with 10 vertices are always better than those from the polyhedral version of Christofides' algorithm. In more details, in Figure \ref{fdtvschris} the horizontal axis of the plot is indexed with the 60 Carr-Vempala points that we considered. For each Carr-Vempala point $x$, there are two data points. The value of the first data point depicted by a circle on the vertical axis is $\frac{|V|-1}{|V|}+\alpha$  and $\alpha$ is the optimal solution to (\ref{ojoinaverage}). 
The value of the second data point depicted by a cross on the vertical axis is $C$ where $C$ is obtained from applying Theorem \ref{FDT2EC} to $x$. In other words, Figure \ref{fdtvschris} is comparing the upper bounds on the instances specific integrality gap certified by Christofides' algorithm  and FDT algorithm for 2EC.

\begin{figure}[h!]
	\centering
	\includegraphics[width=9cm,scale=1.4]{"fdt-vs-christofides".png}
	\caption{Polyhedral version of Christofides' algorithm vs FDT on all Carr-Vempala points defined on graphs with 10 vertices.}
	\label{fdtvschris}
\end{figure}
\paragraph{FDT for 2EC on Carr-Vempala points.}
We ran FDT for 2EC on 963 fractional extreme points of $\subtour(G)$. We enumerated all (fractional) Carr-Vempala points with $10$ and $12$ vertices. Table \ref{table2EC} shows that again FDT found solutions better than the integrality-gap lower bound for most instances. 
\begin{table}[h!]
	\begin{small}
		\centering
		\begin{tabular}{c c c c c}
			\hline
			& $C\in [1.08,1.11]\;$ & $\;C\in (1.11,1.14]\;$ &
			$\;C\in (1.14,1.17]$ &\; $C\in (1.17,1.2]\;$ \\ \hline
			2EC & $79$ & $201$ & $605$ & $78$ \\ \hline\\
		\end{tabular}	\caption{FDT for $\2ec$ implemented applied to all Carr-Vempala with 10 or 12 vertices. A Carr-Vempala point with $k$ vertices has $\frac{3k}{2}$ edges. Thus, the upper bound provided by Theorem \ref{FDT2EC} is $g(\2ec)^{3k/2}$. The lower bound on $g(\2ec)$ is $\frac{6}{5}$.}
		\label{table2EC}
	\end{small}
\end{table}
