\section{Concluding Remarks}

The results in Sections \ref{sec:binaryfdt}, \ref{sec:domTOIP} and \ref{sec:2EC} hold if the intial integer program has some auxilliary continuous variables\footnote{Here by auxilliary variables we mean a variable that does not participate in the objective function}, i.e. when we have 
\begin{equation*}
S(A,b) = \{x\in \mathbb{Z}^n\times \mathbb{R}^p: Ax\geq b\},
\end{equation*}
where the last $p$ variables are continous, $P(A,b) = \{x\in \mathbb{R}^n: Ax\geq b\}$ and define
\begin{equation*}
g(I) = \max_{c\in \mathbb{R}^n_+}\frac{\min_{x\in S(A,b)} \sum_{i=1}^{n}c_ix_i}{\min_{x\in P(A,b)} \sum_{i=1}^{n}c_ix_i},
\end{equation*}
our main results work. In fact our implementation of the subtour elimination formulation is based on an extended formulation with auxiliary variables (see \cite{subtour-extended}). We removed this extension to make the presentation simpler.

Fractional Decomposition Tree is a tool to experminetally evaluate the known bounds (and conjectured bounds) on the integrality gap of combinatorial optimization IP formulations. We hope that applying this tool to a wide range of problems can be a tool to guide conjectures on the upper bounds of integrality gap or at least narrow down the instances for which a closer look is required for the study of integrality gap.
