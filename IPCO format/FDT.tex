% This is samplepaper.tex, a sample chapter demonstrating the
% LLNCS macro package for Springer Computer Science proceedings;
% Version 2.20 of 2017/10/04
%

\documentclass[runningheads]{llncs}
%
\usepackage{graphicx}
% Used for displaying a sample figure. If possible, figure files should
% be included in EPS format.
%
% If you use the hyperref package, please uncomment the following line
% to display URLs in blue roman font according to Springer's eBook style:
% \renewcommand\UrlFont{\color{blue}\rmfamily}

\begin{document}
%
\title{FDT\thanks{Supported by organization x.}}
%
%\titlerunning{Abbreviated paper title}
% If the paper title is too long for the running head, you can set
% an abbreviated paper title here
%
\author{Robert Carr\inst{1}\orcidID{} \and
Arash Haddadan\inst{2}\orcidID{} \and
Cynthia Phillips\inst{3}\orcidID{}}
%
\authorrunning{R. Carr, A. Haddadan, C. Phillips}
% First names are abbreviated in the running head.
% If there are more than two authors, 'et al.' is used.
%
\institute{University of New Mexico, Albuquerque NM 87131, USA\\
\email{bobcarr@unm.}\\   \and
Carnegie Mellon University, Pittsburgh, PA 15213, USA\\
\email{ahaddada@cmu.edu}\and
Sandia National Laboratories, Albuquerque, NM 87185, USA\\
\email{caphill@sandia.gov}}
%
\maketitle              % typeset the header of the con	tribution
%
\begin{abstract}
We present a new algorithm for finding a feasible solution for a mixed-integer linear program. The algorithm runs in polynomial time and is guaranteed to find a feasible integer solution provided the
integrality gap is bounded. The algorithm is based on convex decomposition of scaled linear-
programming relaxations, so in general it provides a suite of integer solutions. Because of the
relationship between convex decomposition and integrality gaps, this algorithm can be seen as an approximation algorithm. We apply our algorithm our algorithm to a class of 
fractional extreme points for two known problems in combinatorial optimization: the 2-edge-connected spanning subgraph problem and the tree augmentaion problem. These computational experiments show that our algorithm can be used as a tool to evaluate the integrality gap of integer programs with their linear relaxation. Finally we provide a stronger characterization of integrality gap for a class of covering problems than that of \cite{goemans}. 

\keywords{Mixed-integer linear programming  \and  Integrality gap \and convex combinations.}
\end{abstract}
%
%
%
\section{Introduction}

Mixed-integer linear programming (MILP), the optimization of a linear objective function subject to linear and integrality constraints, models many practical optimization problems including scheduling, logistics and resource allocation.  The set of feasible points for a MILP is the set
\begin{equation}
S(A,G,b)= \{(x,y)\in \mathbb{Z}^{n}\times \mathbb{R}^p\;:\; Ax+Gy\geq b\}  \label{S}.
\end{equation}
If we drop the integrality constraints, we have the linear relaxation of set $S(A,G,b)$,
\begin{equation}
P(A,G,b) = \{(x,y)\in \mathbb{R}^{n+ p}\;:\; Ax+Gy\geq b\}. \label{P}
\end{equation}

Let $I=(A,G,b)$ be the feasible set of a specific instance. Then $S(I)$ and $P(I)$ denote $S(A,G,b)$ and $P(A,G,b)$, respectively. Given a linear objective function $c$, a Mixed-Integer-Linear Program (MILP) is $\min \;\{cx:\; (x,y) \in S(I)\}$. It is  NP-hard even to determine if an MILP instance has a feasible solution~\cite{GJ79}. However, intelligent branch-and-bound strategies allow commercial and open-source MILP solvers to give exact solutions (or near-optimal with provable bound) to many specific instances of NP-hard combinatorial optimization problems. 

Relaxing the integrality constraints gives the polynomial-time-solvable linear-programming relaxation: $\min \;\{cx:\;(x,y)\in P(I) \}$.  The optimal value of this linear program (LP), denoted $z_{LP}(I,c)$, is a lower bound on the optimal value for the MILP, denoted $z_{IP}(I,c)$. The solution can also provide some useful global structure, even though the fractional values are not directly meaningful. {\em LP-based approximation algorithms} for combinatorial problems involve modeling the problem as a MILP, solving the LP relaxation, finding a (problem-specific) integer-feasible solution from the LP solution, and proving an approximation bound by comparing the solution value to the LP lower bound.

Many researchers (see \cite{vazirani,sw}) have developed polynomial-time LP-based algorithms that find solutions for special classes of MILPs whose cost are provably smaller than $C\cdot z_{LP}(I,c)$. The approximation factor $C$ can be a constant or depend a parameter of the MILP, e.g. $O(\log(n))$. However, for many combinatorial optimization problems there is a limit to such techniques. Define the \textit{integrality gap} of the MILP formulation for instance $I$ to be $g_I= \max_{c\geq 0}\frac{z_{IP}(I,c)}{z_{LP}(I,c)}$. This value depends on the constraints in (\ref{S}).  We cannot hope to find solutions for the MILP with objective values better than $g_I\cdot z_{LP}(I,c)$. 

More generally we can define the integrality gap for a class of instances $\mathcal{I}$:% In this case, the integrality gap for problem $\mathcal{I}$ is
\begin{equation}
g_\mathcal{I} = \max_{c\geq 0 , I\in\mathcal{I}}\frac{z_{IP}(I,c)}{z_{LP}(I,c)}
\end{equation}
For example, finding a  minimum-weight 2-edge-connected multigraph has a natural formulation: every cut is crossed at least twice.  The gap for this formulation is at most $\frac{3}{2}$ \cite{Wolsey1980} and at least $\frac{6}{5}$ \cite{carr-ravi}. Therefore, we cannot hope to obtain an LP-based $(\frac{6}{5}-\epsilon)$-approximation algorithm for this problem using this LP relaxation.

\paragraph{The value of good MILP formulations:} There can be multiple correct MILP formulations for a problem with different integrality gaps. Finding MILP formulations with small integrality gap, e.g. by adding extra contraints, enables better provable approximation algorithms.  Such formulations are also likely to work better in practice when using exact solvers because branch-and-bound algorithms for MILP use LP bounds to prove whole regions of the search space can be pruned. In this paper, we provide tools to help modelers develop MILP formulations with integrality gaps closer to the optimal.

\paragraph{Decomposition} Our methods apply theory connecting integrality gaps to sets of feasible solutions. Instances $I$ with  $g_I=1$ has $P(I)=\conv(S(I))$, the convex hull of the lattice of feasible points. In this case, $P(I)$ is an \textit{integral} polyhedron. The spanning-tree polytope and the perfect-matching polytope \cite{schrijver} have this property. For such problems there is an algorithm to express vector $x\in P(I)$ as a convex combination of points in $S(I)$ in polynomial time \cite{cons-cara}.
\begin{proposition}\label{cara}
	If $g_I=1$, then for $(x,y)\in P(I)$, there exists $\theta \in [0,1]^k$, where $\sum_{i=1}^{k}\theta_i =1$ and $(\tilde{x}^i,\tilde{y}^i)\in S(I)$ for $i=1,\ldots,k$ such that $\sum_{i=1}^{k}\theta_i \tilde{x}^i\leq x$. Moreover, we can find such a convex combination in polynomial time.
\end{proposition}

Carr and Vempala~\cite{CV} gave a decomposition result for integrality gap $1<g(I)<\infty$. %A analogous result to Proposition \ref{cara} is the following theorem due to Carr and Vempala \cite{CV}, which
This is a generalization of Goemans' proof for blocking polyhedra \cite{goemans}. 

\begin{theorem}[Carr, Vempala \cite{CV}] \label{CV}
	Let $(x,y)\in P(I)$, there exists $\theta \in [0,1]^k$, where $\sum_{i=1}^{k}\theta_i =1$ and $(\tilde{x}^i,\tilde{y}^i)\in \dom(S(I))$ for $i=1,\ldots,k$ such that $\sum_{i=1}^{k}\theta_i \tilde{x}^i\leq Cx$ if and only if $g_I \leq C$.
\end{theorem}
Here $\dom(P(I))$ is the set of points $(x',y')$ such that there exists a point $(x,y)\in P$ with $x'\geq x$, also known as the dominant of $P(I)$. For covering problems the polyhedron is essentially the same as its dominant, but this is not true in general. While there is an exact algorithm for problems with gap $1$, Theorem~\ref{CV} is existenial, with no construction.
%In contrast to Proposition \ref{cara} which implies exact algorithms for problems with a gap of 1, Theorem \ref{CV} does not imply an approximation algorithm, since it does not suggest how to find such a convex combination in polynomial time.
%This points to an interesting open question. 
% We show later, for $I$ with $g(I)<\infty$, the notation of dominant is in fact useful.
\iffalse

\begin{question*}\label{question1}
	Assume reasonable complexity assumptions (such as UGC or $\textrm{P}\neq \textrm{NP}$). Given instance $I$ with $1<g_I<\infty$ and $(x,y)\in P(I)$, can we find $\theta \in [0,1]^k$, where $\sum_{i=1}^{k}\theta_i =1$ and $(\tilde{x}^i,\tilde{y}^i)\in \dom(\conv(S(I)))$ for $i=1,\ldots,k$ such that $\sum_{i=1}^{k}\theta_i \tilde{x}^i\leq g_Ix$ in polynomial time?
\end{question*}

This seems to be a very hard question. A more specific question is of more interest.

\begin{question}\label{question2}
	Assume reasonable complexity assumptions, a specific problem $\mathcal{I}$ with  $1<g_{\mathcal{I}}<\infty$, and $(x,y)\in P(I)$ for some $I\in \mathcal{I}$, can we find $\theta \in [0,1]^k$, where $\sum_{i=1}^{k}\theta_i =1$ and $(\tilde{x}^i,\tilde{y}^i)\in S(I)$ for $i=1,\ldots,k$ such that $\sum_{i=1}^{k}\theta_i \tilde{x}^i\leq g_{\mathcal{I}}x$ in polynomial time?
\end{question}
Although Question \ref{question2} is wide open, for some problems there are polynomial time algorithms closing the gap. For example, for generalized Steiner forest problem \cite{jain} gave an LP-based 2-approximation algorithm. The gap for this problem is also lower bounded by 2. Same holds for the set covering problem \cite{randomizedrounding}. In fact, for set cover the approximation algorithm achieving the same factor as the integrality gap lower bound, is a \textit{randomized rounding} algorithm. Raghavan and Thompson \cite{randomizedrounding} showed that this technique achieves provably good approximation for many combinatorial optimization problems.  

If we relax Question \ref{question1} (resp. Question \ref{question2}), but multiplying $g(I)$ (resp. $g(\mathcal{I})$) by a factor $C$, they are still very interesting, since they will provide upper bounds on the integraltiy gap the instance (resp. the problem). The results in this paper serve this purpose.
\fi
To study integrality gaps, we wish to find such a solution contructively:
\begin{question}\label{question2}
	Assume reasonable complexity assumptions, a specific problem $\mathcal{I}$ with  $1<g_{\mathcal{I}}<\infty$, and $(x,y)\in P(I)$ for some $I\in \mathcal{I}$, can we find $\theta \in [0,1]^k$, where $\sum_{i=1}^{k}\theta_i =1$ and $(\tilde{x}^i,\tilde{y}^i)\in S(I)$ for $i=1,\ldots,k$ such that $\sum_{i=1}^{k}\theta_i \tilde{x}^i\leq $C$ g_{\mathcal{I}}x$ in polynomial time? We wish to find the smallest slack factor $C$ as possible.
\end{question}

We give a general approximation framework for solving $\{0,1\}$-MILPs.  Consider the set of point described by sets $S(I)$ and $P(I)$ as in (\ref{S}) and (\ref{P}), respectively. Assume in addition that $S(I),P(I)\subseteq [0,1]^n\times \mathbb{R}^p$. For a vector $x\in \mathbb{R}_{\geq 0}^n$ such that $(x,y)\in P(I)$ for some $y\in \mathbb{R}^p$, let $\spp(x)= \{i \in \{1,\ldots,n\}: x_i \neq 0\}$. 


\textit{Fractional Decomposition Tree} (FDT) is a polynomial-time algorithm that given a point $(x,y)\in P(I)$ produces a convex combination of feasible points in $S(I)$ that are dominated by a ``factor" $C$ of $x$ in the coordinates corresponding to $x$. If $C = g_I$, it would be optimal. However we can only guarantee a factor of $g_I^{|\spp(x)|}$. FDT relies on iteratively solving linear programs that are about the same size as the description of $P(I)$.

\begin{restatable}{theorem}{binaryFDT}
	\label{binaryFDT}
	Assume $1\leq g_I 	<\infty$. 	
	The Fractional Decomposition Tree (FDT) algorithm, given $(x^*,y^*)\in P(I)$, produces in polynomial time $\lambda\in [0,1]^k$ and $(z^1,w^1),\ldots,(z^k,w^k) \in S(I)$ such that $k\leq |\spp(x^*)|$, $\sum_{i=1}^{k}\lambda_i z^i\leq Cx^*$, and $\sum_{i=1}^{k}\lambda_i = 1$. Moreover, $C\leq g_I^{|\spp(x^*)|}$.
\end{restatable}

FDT finds feasible solutions to any MILP with finite gap. This can be of independent interest, especially in proving that a model has unbounded gap.
\begin{restatable}{theorem}{DomToIP}
	\label{domtoIP}
	Assume $1\leq g_I < \infty$. The DomToIP algorithm finds $(\hat{x},\hat{y})\in S(I)$ in polynomial time.
\end{restatable}

For general $I$ it is NP-hard to even decide if $S(I)$ is empty or not. There are a number of heuristics for this purpose, such as the feasibility pump heuristic \cite{fp1,fp2}. These heuristics are often very effective and fast in practice, however, they can sometimes fail to find a feasible solution. These heuristics do not provide any bounds on the quality of the solution they find. 

We consider the following TSP-related problems.  The {\em 2-edge-connected subgraph problem (2EC)} is to find a minimum-weight 2-edge-connected multigraph (subgraph which can contain multiple copies of each edge)
in a graph $G=(V,E)$ with respect to weights $c\in \mathbb{R}^E_{\geq 0}$. In the {\em tree-augmentation problem (TAP)} we wish to add a minimum-cost set of edges to a tree to make it 2-edge-connected.  We formally define TAP in Section~\ref{experiment}.

One can extend the FDT algorithm for binary MILPs into covering $\{0,1,2\}$-MILPs by losing a factor $2^{|\spp(x)|}$ on top of the loss for FDT. In order to eradicate this extra factor, we need to treat the coordinate $i$ with $x_i=1$ differently. For 2EC we are able to achieve this. The 2EC problem has the natural linear programming relaxation is
\begin{equation}\label{2ECpol}
\EC(G)= \{x\in [0,2]^{E}\;:\; x(\delta(U))\geq 2 \text{ for $\emptyset \subset U\subset V$})\}.
\end{equation}
 We prove the following theorem.

\begin{restatable}{theorem}{FDTEC}
	\label{FDT2EC}
	Let $G=(V,E)$ and $x$ be an extreme point of  $\EC(G)$. The FDT algorithm for 2EC produces $\lambda\in [0,1]^k$ and 2-edge-connected multigraphs $F_1,\ldots,F_k$ such that $k\leq 2|V|-1$, $\sum_{i=1}^{k}\lambda_i \chi^{F_i}\leq Cx$, and $\sum_{i=1}^{k}\lambda_i = 1$. Moreover, $C\leq g_{\EC}^{k}$, where $g_{\EC}$ is the integrality gap of the 2-edge-connected multigraph problem with respect to the formulation in (\ref{2ECpol}). 
\end{restatable}

We give a stronger characterization of integrality gap than that in Theorem \ref{CV} for bounded covering problems. For this purpose assume $P= \{x\in \mathbb{R}^n_{\geq 0}: Ax\geq b\cdot \textbf{1}, x \leq b\cdot \textbf{1}\}$, where $A\in \mathbb{Z}^{m\times n}_{\geq 0}$ and $b\in \mathbb{Z}_{\geq 0}$. Let $S= P\cap \mathbb{Z}^n$ and $g= \max_{c\geq 0} \frac{\min_{x\in S}cx}{\min_{x\in P}cx}$. Examples of problems whose natural linear programming relaxation is $P$ (for some matrix $A$ and integer $b$) include the 2-edge-connected multigraph problem, Steiner tree problem, tree augmentation problem, and many others.

\begin{restatable}{theorem}{tightcuts}
	\label{tightcuts}	
	We have $g\leq C$ if and only if for each extreme point $x$ of $P$, there exists $\theta\in [0,1]^k$, where $\sum_{i=1}^{k}\theta_i = 1$ and $\tilde{x}^i \in S$ for $i=1,\ldots,k$ such that \begin{itemize}
		\item for $\ell\in \{1,\ldots,n\}$, if $x_\ell =0$, then $\tilde{x}^i_\ell=0$ for $i =1,\ldots,k$, i.e. $\tilde{x}^i$ is in the support of $x$,
		\item we have $A_j (\sum_{i=1}^{k}\theta_i\tilde{x}^i)\leq C\cdot A_j x$, for $j$ such that $A_j x =b$.
	\end{itemize}
\end{restatable}



This means in order to prove an upper bound on the integrality gap of a  bounded covering problem, we need to show there is a convex combination of integer feasible points that is ``cheap'' on all tight cuts. Notice that Theorem \ref{CV} requires the certificate convex combination to be ``cheap" on every single variable.




\paragraph{Experiments} Although the bound guaranteed in both Theorems \ref{binaryFDT} and \ref{FDT2EC} are very large for large problems, we show that in practice, the algorithm works very well for the TSP-like problems described above. We show how one might use FDT to investigate the integrality gap for such well-studied problems.

Known polyhedral structure makes it easier to study integrality gaps for such problems. Carr and Ravi \cite{carr-ravi} introduced fundamental extreme points. A point $x$ in Held-Karp relaxation for TSP (or 2EC; they have the same relaxation) is a point whose support of $x$, namely $G_x$ satisfies the following: i)  $G_x$ is a cubic graph, ii) in $G_x$ there is exactly one edge with $x_e=1$ incident to each node iii) The fractional edges of $G_x$ form a Hamiltonian cycle.  We say a fundamental extreme point (FEP) is {\em order $k$} if there are $k$ nodes on this Hamiltonian cycle. An FEP of order $k$ could represent an instance with many more than $k$ vertices. Carr and co-authors~\cite{CV,carr-ravi,BC11} proved that showing that $Cx$ is a convex combination of tours (resp. 2-edge-connected multigraphs) for all fundamental extreme points is equivalent to proving that the integrality gap for TSP (resp. 2EC) is bounded above by $C$. We use fundamental extreme point to create the ``hardest'' LP solutions to decompose.

There are fairly good bounds for the integrality gap for TSP or $\EC$. Benoit and Boyd \cite{TSPcompute} used a quadratic program to show the integrality gap for TSP is at most $\frac{20}{17}$ for graphs with at most 10 vertices. Alexander et. al \cite{abe} used the same ideas to provide an upper bound of $\frac{7}{6}$ for $\EC$ on graphs with at most 10 vertices. For $\EC$ we show that the integrality gap is at most $\frac{6}{5}$ for FEPs of order at most 12. An FEP of order $k$ might correspond to an extreme point of a much bigger graph, since each edge in a FEP with value 1 actually corresponds to a path of edges with value 1.
%In fact, since FDT can be applied to different problem, we can use it to evaluate the integrality gap of other well-known problem.
For TAP, we create random fractional extreme points and round them using FDT. For the instances that we create the blow-up factor is always below $\frac{3}{2}$ providing an upper bound for such instances.

% providing approximation ratios that are far better than the theoretical bound in the theorem statements.
%We examine FDT for binary MILPs for problems such as the tree augmentation problem (TAP) and we apply FDT for 2EC on some interesting and ``hard to decompose" points in the linear relaxation. Our computational results show that the FDT algorithm is a good tool to evaluate the integrality gap of integer programming formulations. 



\paragraph{Contributions}  The paper has the following contributions: 
\begin{itemize}
\item We give a simple algorithm, DomToIP, that can prove a binary formulation’s integrality gap is unbounded or if not provide a feasible integer solution.  Someone formulating a first MILP for a new problem can test it with DomToIP.  If the algorithm ever fails in finding a feasible solution, the MILP has an unbounded gap.
\item We give an algorithm, Fractional Decomposition Tree (FDT), to construct the convex decomposition in the Carr-Vempala theorem, perhaps scaling by a factor larger than the integrality gap.  Each step of this algorithm provides a {\em lower bound} on the instances integrality gap. This also provides a lower bound on the approximation factor of any LP-based approximation algorithm using this formulation. The overall approximation factor of the FDT algorithm is an upper bound on the integrality gap for that specific instance.
\item For a special set of problems related to TSP, where there is a notion of a fundamental extreme point and long-running attempts to exactly determine the integrality gap of classic formulations, experimental analysis with FDT can help give some intuition about which bound(s) is/are likely to be loose.  Computing on fundamental extreme points is a way to experimentally characterize the gap upper bound.  There is no guarantee. Still, this can help direct theoretical analysis in the most promising direction.  For instance for 2EC, FDT gives good approximate solutions, better than the best current competitor (Christofides' algorithm).

\item A stronger theoretical result on integrality gap that can be potentially employed to obtain improved upper bounds on the integrality gap of bounded covering problems.  
\end{itemize}


\iffalse{
Finally we give a stronger characterization of integrality gap than that in Theorem \ref{CV} for bounded covering problems. For this purpose assume $P= \{x\in \mathbb{R}^n_{\geq 0}: Ax\geq b\textbf{1}, x \leq b\textbf{1}\}$, $S= P\cap \mathbb{Z}^n$, and $g= \max_{c\geq 0} \frac{\min_{x\in S}cx}{\min_{x\in P}cx}$. Examples of such problems include the 2-edge-connected multigraph problem, the tree augmentation problem, and many others.


\begin{restatable}{theorem}{tightcuts}
	\label{tightcuts}
	Let $x\in P$, there exists $\theta\in [0,1]^k$, where $\sum_{i=1}^{k}\theta_i = 1$ and $\tilde{x}^i \in S$ for $i=1,\ldots,k$ such that \begin{itemize}
		\item for $\ell\in \{1,\ldots,n\}$, if $x_\ell =0$, then $\tilde{x}^i_\ell=0$ for $i =1,\ldots,k$, i.e. $\tilde{x}^i$ is in the support of $x$,
		\item we have $Cb = C\cdot A_j x \geq A_j (\sum_{i=1}^{k}\theta_i\tilde{x}^i)$, for $j$ such that $A_j x =b$,  
	\end{itemize}
	if and only if $C\geq g$.
\end{restatable}



This means in order to prove an upper bound on the integrality gap of a covering problem, we need to show there is a convex combination of integer feasible points that is ``cheap'' on all tight cuts. Notice that Theorem \ref{CV} requires the certificate convex combination to be ``cheap" on every single variable.
\subsection{Notations}
For vectors $x,y\in \mathbb{R}_{n}$ we say $x$ dominates $y$ if $x_i\geq y_i$ for $i= 1,\ldots,n$. For $m\times n$ matrix $A$, let $A_j$ be the $j$-th row of $A$ and $A^j$ be the $j$-th column of $A$. Let $\textbf{1}$ be the vector of all ones of a suitable dimension. For a set $S$ of vectors in $\mathbb{R}_{n}$, $\conv(S)$ is the convex hull of all the points in $S$.
}\fi

\section{Finding a feasible solution}\label{domTOIP}
Consider a formulation instance $I=(A,G,b)$. Define sets $S(I)$ and $P(I)$ as in (\ref{S'}) and (\ref{P'}), respectively. For simplicity in the notation we denote $P(I),S(I),$ and $g(I)$ with $P$, $S$, and $g$ for this section and the next section. Also, for both sections we assume $t=|\spp(x)|$. Without loss of generality we can assume $x_i = 0$ for $i=t+1,\ldots,n$.

In this section we prove Theorem \ref{domtoIP}. In fact, we prove a stronger result. 
\begin{lemma}\label{domlemma}
	Given $(x,y)\in \dom(P)$, there is an algorithm (the Dom2IP algorithm) that finds $(x^{(t)},y^{(t)})\in S$ in polynomial time, such that $x^{t}\leq x$, where $t=|\spp(x)|$\end{lemma}

We prove Lemma \ref{domlemma} by introducing an algorithm that ``fixes" the variables iteratively, starting from $x_1$ and ending at $x_t$. Suppose we run the algorithm for $\ell\in \{0,\ldots,t-1\}$ iterations and we have $(x^{(\ell)},y^{(\ell)})\in \dom(P)$  such that $x^{(\ell)}_i\in \{0,1\}$ for $i=1,\ldots,t$. Now consider the following linear program. The variables of this linear program are the $z\in \bbbr^n$ variables and $w\in \bbbr^p$.
\begin{align}
	\DOMtoS(x^{(\ell)})\quad\quad& \min\quad \;z_{\ell+1}\\
	&\;\text{s.t.} \quad \;\;Az+ Gw\geq b \\
	&\;{\color{white}{\text{s.t.}} }\quad \;\; \; z_j = x^{(\ell)}_j \quad \; j =1,\ldots, \ell\\
	&\;{\color{white}{\text{s.t.}} }\quad \; \;\; z_j \leq x^{(\ell)}_j \quad \; j = \ell+1,\ldots,n\\
	&\;{\color{white}{\text{s.t.}} }\quad \; \;\; z\;\geq 0
\end{align}

If the optimal value to $\DOMtoS(x^{(\ell)})$ is 0, then let $x^{(\ell+1)}_{\ell+1} = 0$. Otherwise if the optimal value is strictly positive let $x^{(\ell+1)}_{\ell+1} = 1$. Let $x^{(\ell+1)}_j = x^{(\ell)}_j$ for $j\in \{1,\ldots,n\}\setminus \{\ell+1\}$. 

The above procedure suggests how to find $(x^{(\ell+1)},y^{(\ell+1)})$ from $(x^{(\ell)},y^{(\ell)})$. The Dom2IP algorithm initializes with $(x^{(0)},y^{(0)})=(x,y)$ and  iteratively calls this procedure in order to obtain $(x^{(t)},y^{(t)})$. 
\iffalse{
\begin{algorithm}[H]
	\KwIn{$(x,y)\in \dom(S)$}
	\KwOut{$(x^{(t)},y^{(t)}) \in S$, $x^{(t)}\leq x$}
	$x^{(0)}\leftarrow x$\\
	\For{$\ell = 0$ \textbf{to} $t-1$}{
		$x^{(\ell+1)} \leftarrow x^{(\ell)}$\\
		$\eta \leftarrow$ optimal value of $ \DOMtoS(x^{(\ell)})$\\
		$y^{(\ell+1)}\leftarrow$ optimal solution for $w$ variables in $ \DOMtoS(x^{(\ell)})$\\
		\eIf{$\eta = 0$}{
			$x^{(\ell+1)}_{\ell+1} \leftarrow 0$\
		}{
			$x^{(\ell+1)}_{\ell+1} \leftarrow 1$
		}
	}
	\label{domtoIPalg}
	\caption{The Dom2IP algorithm}
\end{algorithm}
}\fi
We prove that indeed $(x^{(t)},y^{(t)})\in S$. 

	First, we need to show that in any iteration $\ell=  0,\ldots,t-1$ of DomtoIP the $\DOMtoS(x^{\ell})$ is feasible. We show something stronger. For $\ell=0,\ldots,t-1$ let
	\begin{align*}
	\LP^{(\ell)}&= \{(z,w)\in P\; : \; z\leq x^{(\ell)} \mbox{ and } z_j=x_j^{(\ell)} \mbox{ for } j=1,\ldots,\ell\}, \text{ and}\\
	\IP^{(\ell)}&= \{(z,w)\in \LP^{(\ell)}\; : \; z\in \{0,1\}^n\}.
	\end{align*}
	Notice that if $\LP^{(\ell)}$ is a non-empty set then $\DOMtoS(x^{(\ell)})$ is feasible. We show by induction on $\ell$ that $\LP^{(\ell)}$ and $\IP^{(\ell)}$ are not empty sets for $\ell=0,\ldots,t-1$. First notice that $\LP^{(0)}$ is clearly feasible since by definition $(x^{(0)},y^{(0)})\in \dom(P)$, meaning there exists $(z,w)\in P$ such that $z\leq x^{(0)}$. By Theorem \ref{CV}, there exists $(\tilde{z}^i,\tilde{w}^i) \in S$ and $\theta_i\geq 0$ for $i=1,\ldots,k$ such that $\sum_{i=1}^{k} \theta_i = 1$ and $\sum_{i=1}^{k}\theta_i \tilde{z}^i \leq gz$. Hence, $\sum_{i=1}^{k}\theta_i \tilde{z}^i \leq gz\leq gx^{(0)}$. So if $x^{(0)}_j=0$, then $ \sum_{i=1}^{k}\theta_i \tilde{z}_j^i =0$, which implies that $\tilde{z}^i_j=0$ for all $i=1,\ldots,k$ and $j= 1,\ldots,n$ where $x^{(0)}_j=0$. Hence, $z^i\leq x^{(0)}$ for $i=1,\ldots,k$. Therefore $(\tilde{z}^i,\tilde{w}^i)\in \IP^{(0)}$ for $i=1,\ldots,k$, which implies $\IP^{(0)}\neq \emptyset$.
	
	Now assume $\IP^{(\ell)}$ is non-empty for some $\ell \in \{0,\ldots,t-2\}$. Since $\IP^{(\ell)}\subseteq\LP^{(\ell)}$ we have $\LP^{(\ell)}\neq \emptyset$ and hence the $\DOMtoS(x^{(\ell)})$ has an optimal solution $(z^*,w^*)$. 
	
	We consider two cases. In the first case, we have $z^*_{\ell+1}=0$. In this case we have $x^{(\ell+1)}_{\ell+1}=0$. Since $z^*\leq x^{(\ell+1)}$, we have $(z^*,w^*)\in \LP^{(\ell+1)}$. Also, $(z^*,w^*)\in P$. By Theorem \ref{CV} there exists $(\tilde{z}^i,\tilde{w}^i)\in S$ and $\theta_i\geq 0$ for $i=1,\ldots,k$ such that $\sum_{i=1}^{k} \theta_i = 1$ and  $\sum_{i=1}^{k}\theta_i \tilde{z}^i \leq gz^*$ . We have
	\begin{equation}
	\sum_{i=1}^{k}\theta_i \tilde{z}^i \leq gz^*\leq gx^{(\ell+1)}
	\end{equation}
	So for $j\in \{1,\ldots,n\}$ where $x^{(\ell+1)}_j=0$, we have $z^i_j=0$ for $i=1,\ldots,k$. Hence, $\tilde{z}^i\leq x^{(\ell+1)}$ for $i=1,\ldots,k$. Hence, there exists $(z,w)\in S$ such that $z\leq x^{(\ell+1)}$. We claim that $(z,w)\in \IP^{(\ell+1)}$. If $(z,w)\notin \IP^{(\ell+1)}$ we must have $1\leq j \leq \ell$ such that $z_j < x^{(\ell+1)}_{j}$, and thus $z_j = 0$ and $x^{(\ell+1)}_j=1$. Without loss of generality assume $j$ is minimum number satisfying $z_j < x^{(\ell+1)}_{j}$. Consider iteration $j$ of the Dom2IP algorithm. Notice that $z\leq x^{(\ell+1)}\leq x^{(j)}$. We have $x^{(j)}_j=1$ which implies when we solved $\DOMtoS(x^{(j-1)})$ the optimal value was strictly larger than zero. However, $(z,w)$ is a feasible solution to $\DOMtoS(x^{(j-1)})$ and gives an objective value of 0. This is a contradiction, so $(z,w)\in \IP^{(\ell+1)}$.
	
	Now for the second case, assume $z^*_{\ell+1} > 0$. We have $x^{(\ell+1)}_{\ell+1}=1$. Notice that for each point $z\in \LP^{(\ell)}$ we have $z_{\ell+1} >0$, so for each $z\in \IP^{(\ell)}$ we have $z_{\ell+1}>0$, i.e. $z_{\ell+1}=1$. This means that $(z,w)\in \IP^{(\ell+1)}$, and $\IP^{(\ell+1)} \neq \emptyset$.
	
	Now consider $(x^{(t)},y^{(t)})$. Let $(z,y^{(t)})$ be the optimal solution to $\LP^{(t-1)}$. If $x^{(t)} = 0$, we have $x^{(t)} = z$, which implies that $(x^{(t)},y^{(t)})\in P$, and since $x^{(t)}\in \{0,1\}^n$ we have $(x^{(t)},y^{(t)})\in S$. If $x^{(t)} =1$, it must be the case that $z_t > 0$. By the argument above there is a point $(z',w')\in \IP^{(t-1)}$. We show that $x^{(t)} = z'$. Observe that for $j=1,\ldots,n-1$ we have $z'_j= x_j^{(t-1)}=x_j^{(t)}$. We just need to show that $z'_j = 1$. Assume $z'_j = 0$ for contradiction, then $(z',w')\in \LP^{(t-1)}$ has objective value of $0$ for $\DOMtoS(x^{(t-1)})$, this is a contradiction to $(z,w)$ being the optimal solution. This concludes the proof of Lemma \ref{domlemma}. 
	
	Notice that Lemma \ref{domlemma} implies Theorem \ref{domtoIP}, since it is easy to obtain an integer point in $\dom(P)$: rounding up any fractional point in $P$ gives us a point in $\dom(P)$.


\iffalse{
\begin{algorithm}[H]\label{FDTFull}
	\KwIn{$P= \{(x,y)\in \bbbr^{n\times p}: Ax+Gy\geq b\}$ and $S=\{(x,y)\in P: x\in \{0,1\}^n\}$ such that $g=\max_{c\in \bbbr^n_+ }\frac{\min_{(x,y)\in }cx}{\min_{(x,y)\in P}cx}$ is finite, $(x^*,y^*)\in P$}
	\KwOut{$(z^i,w^i)\in S$ and $\lambda_i\geq 0$ for $i=1,\ldots,k$ such that $\sum_{i=1}^{k}\lambda_i = 1$, and $\sum_{i=1}^{k}\lambda_iz^i\leq g^tx^*$ }
	$L^0\leftarrow [(x^*,y^*),1]$\\
	\For{$i=1$ \textbf{to} $t$}{
		$L'\leftarrow \emptyset$\\
		\For{$[(x,y),\lambda] \in L^i$}{
			Apply Lemma \ref{round-up} to obtain $[(\hat{x}^0,\hat{y}^0),\gamma_0]$ and $[(\hat{x}^1,\hat{y}^1),\gamma_1]$\\
			$L' \leftarrow L' \cup \{[(\hat{x}^0,\hat{y}^0),\gamma_0]\} \cup \{[(\hat{x}^1,\hat{y}^1),\gamma_1]\}$\\			
		}
		Apply Lemma \ref{prune} to $L'$ to obtain $L^{i+1}$. 
	}
	\For{$[(x,y),\lambda] \in L^t$}{
		Apply Algorithm \ref{domtoIP} to $(x,y)$ to obtain $(z,w)\in S$\\
		$F \leftarrow F \cup \{[(z,w),\lambda]\}$
	}
	\textbf{return} $F$
	\caption{Fractional Decomposition Tree Algorithm}
\end{algorithm}}\fi

\section{FDT on binary MIPs}
\label{binaryfdt}

Assume we are given a point $(x^*,y^*)\in P$. For instance, $(x^*,y^*)$ can be the optimal solution of minimizing a cost function $cx$ over set $P$, which provides a lower bound on $\min_{(x,y)\in S(I)} cx$.  In this section, we prove Theorem \ref{binaryFDT} by describing the Fractional Decomposition Tree (FDT) algorithm.
\iffalse{We also remark that if $g_I=1$, then the algorithm will give an exact decomposition of any feasible solution. }\fi


The FDT algorithm grows a tree similar to the classic branch-and-bound search tree for integer programs. Each node represents a partially integral vector $(\bar{x},\bar{y})$ in $\dom(P)$ together with a multiplier $\bar{\lambda}$. The solutions contained in the nodes of the tree become progressively more integral at each level. In each level of the tree, the algorithm maintain a conic combination of points with the properties mentioned above. Leaves of the FDT tree contain solutions with integer values for all the $x$ variables that dominate a point in $P$. We will later see how we can turn these into points in $S$. 


\paragraph{Branching on a node}
We begin with the following lemmas that show how the FDT algorithm branches on a variable.
\begin{lemma}\label{LPClemma}
	Given $(x',y')\in \dom(P)$ and $\ell\in \{1,\ldots,n\}$, we can find in polynomial time vectors $(\hat{x}^0,\hat{y}^0),(\hat{x}^1,\hat{y}^1)$ and scalars $\gamma_0,\gamma_1 \in [0,1]$ such that: (i) $\gamma_0 + \gamma_1  \geq \frac{ 1}{g}$, (ii) $(\hat{x}^0,\hat{y}^0)$ and $(\hat{x}^1,\hat{y}^1)$ are in  $ P$
		,(iii) $\hat{x}^0_\ell=0$ and $\hat{x}^1_\ell=1$, (iv) $\gamma_0 \hat{x}^0 + \gamma_1\hat{x}^1 \leq x'$.
\end{lemma}


\begin{proof} 
	Consider the following linear program which we denote by $\LPC(\ell,x',y')$. The variables of $\LPC(\ell,x',y')$ are $\gamma_0,\gamma_1$ and $(x^0,y^0)$ and $(x^1,y^1)$. 
	\begin{align}
		\LPC(\ell,x',y')\quad\quad& \max\quad \;\lambda_0+\lambda_1\\
		&\;\text{s.t.} \quad Ax^j + Gy^j\geq b\lambda_j & \mbox{ for $j=0,1$} \label{feasibility}\\
		&\;{\color{white}{\text{s.t.}} }\quad 0 \leq x^j \leq \lambda_j &\mbox{ for $j=0,1$}\label{bound}\\
		&\;{\color{white}{\text{s.t.}} }\quad x^0_\ell = 0,\; x^1_\ell =\lambda_1\label{branchcoordinate}\\
		&\;{\color{white}{\text{s.t.}} }\quad x^0 + x^1 \leq x'\label{packing}\\
		&\;{\color{white}{\text{s.t.}} }\quad \lambda_0,\lambda_1 \geq 0
	\end{align}
	
	Let $(x^0,y^0),(x^1,y^1)$, and $\gamma_0,\gamma_1$ be an optimal solution solution to the LP above. Let $(\hat{x}^0,\hat{y}^0) = (\frac{x^0}{\gamma_0},\frac{y^0}{\gamma_0})$, $(\hat{x}^1,\hat{y}^1) = (\frac{x^1}{\gamma_1},\frac{y^1}{\gamma_1})$. Observe that  (ii), (iii), (iv) are satisfied with this choice. In order to show that (i) is also satisfied we prove the following claim.
	
	\begin{claim}\label{CVexists}
		We have $\gamma_0 + \gamma_1\geq \frac{1}{g}$.
	\end{claim}
	\begin{proof}
		We show that there is a feasible solution that achieves the objective value of $\frac{1}{g}$. By Theorem \ref{CV} there exists $\theta \in [0,1]^k$, with $\sum_{i=1}^{k}\theta_i = 1$ and $(\tilde{x}^i,\tilde{y}^i)\in S$ for $i=1,\ldots,k$ such that 
		$\sum_{i=1}^{k}\theta_i \tilde{x}^i\leq gx'$. 
		
		\begin{equation}\label{splitting}
		x'\geq \sum_{i=1}^{k}\frac{\theta_i}{g} \tilde{x}^i
		={\sum_{i\in [k]: \tilde{x}^i_\ell =0}\frac{\theta_i}{g} \tilde{x}^i}+{\sum_{i\in [k]: \tilde{x}^i_\ell =1}\frac{\theta_i}{g} \tilde{x}^i}
		\end{equation}
		For $j=0,1$, let $(x^j,y^j) = \sum_{i\in [k]:\tilde{x}^i_\ell=j} \frac{\theta_i}{g}(\tilde{x}^i,\tilde{y}^i)$. Also let $\lambda_0=\sum_{i\in [k]: \tilde{x}^i_\ell =0}\frac{\theta_i}{g}$ and $\lambda_1 = \sum_{i\in [k]: \tilde{x}^i_\ell =1}\frac{\theta_i}{g}$. Note that $\lambda_0+\lambda_1 = \frac{1}{g}$. Constraint (\ref{packing}) is satisfied by Inequality (\ref{splitting}). Also, for $j=0,1$ we have
		\begin{equation}
		Ax^j+Gy^j = \sum_{i\in[k], \tilde{x}^i_\ell = j} \frac{\theta_i}{g} (A\tilde{x}^i + G\tilde{y}^i) \geq b \sum_{i\in[k], \tilde{x}^i_\ell = j} \frac{\theta_i}{g} = b\lambda_j.
		\end{equation}
		Hence, Constraints (\ref{feasibility}) holds. Constraint (\ref{branchcoordinate}) also holds since $x^0_\ell$ is obviously $0$ and $x^1_\ell= \sum_{i\in [k]: \tilde{x}^i_\ell = 1}\frac{\theta_i}{g}= \lambda_1$. The rest of the constraints trivially hold. 
	\end{proof}
	This concludes the proof of Lemma \ref{LPClemma}.	
\end{proof}

We now show if $x'$ in the statement of Lemma \ref{LPClemma} is partially integral, we can find solutions with more integral components.
\begin{lemma}\label{round-up}
	Given $(x',y')\in \dom(P)$, such that $x'_1,\ldots,x'_{\ell-1}\in \{0,1\}$ for some $\ell\geq 1$, we can find in polynomial time vectors $(\hat{x}^0,\hat{y}^0),(\hat{x}^1,\hat{y}^1)$ and scalars $\gamma_0,\gamma_1 \in [0,1]$ such that: (i) $\frac{ 1}{g}\leq \gamma_0 + \gamma_1  \leq 1$, (ii) $(\hat{x}^0,\hat{y}^0)$ and $(\hat{x}^1,\hat{y}^1)$ are in  $\dom( P)$, (iii) $\hat{x}^0_\ell=0$ and $\hat{x}^1_\ell=1$, (iv) $ \gamma_0\hat{x}^0 +\gamma_1 \hat{x}^1 \leq
		x'$,(v) $\hat{x}^i_j\in \{0,1\}$ for $i=0,1$ and $j=1,\ldots,\ell-1$.
\end{lemma} 
\begin{proof}
	By Lemma \ref{LPClemma} we can find $(\bar{x}^0,\bar{y}^0)$, $(\bar{x}^1,\bar{y}^1)$, $\gamma_0$ and $\gamma_1$ that satisfy (i), (ii), (iii), and (iv). We define $\hat{x}^0$ and $\hat{x}^1$ as follows. For $i=0,1$, for $j=1,\ldots,\ell-1$, let $\hat{x}^i_j= \ceil{\bar{x}^i_j}$, for $j=\ell,\ldots,t$ let $\hat{x}^i_j = \bar{x}^i_j$. We now show that $(\hat{x}^0,\bar{y}^0)$, $(\hat{x}^1,\bar{y}^1)$, $\gamma_0$, and $\gamma_1$ satisfy all the conditions. Note that conditions (i), (ii), (iii), and (v) are trivially satisfied. Thus we only need to show (iv) holds. We need to show that $\gamma_0 \hat{x}^0_j+\gamma_1\hat{x}^1_j\leq gx'_j$. If $j=\ell,\ldots,t$, then this clearly holds. Hence, assume $j\leq \ell-1$. By the property of $x'$ we have $x'_j\in \{0,1\}$. If $x'_j= 0$, then by Constraint (\ref{packing}) we have $\bar{x}^0_j = \bar{x}^1_j=0$. Therefore, $\hat{x}^i_j=0$ for $i=0,1$, so (iv) holds. Otherwise if $x'_j = 1$, then we have
	$\gamma_0\hat{x}^0_j+\gamma_1\hat{x}^1_j\leq \gamma_0+\gamma_1\leq 1\leq x'_j.$ 
	Therefore (v) holds.
\end{proof}

\paragraph{Growing and Pruning FDT tree} The FDT algorithm maintains nodes $L_i$ in iteration $i$ of the algorithm. The nodes in $L_i$ correspond to the nodes in level $L_i$ of the FDT tree. The points in the leaves of the FDT tree, $L_t$, are points in $\dom(P)$ and are integral for all integer variables.


\begin{lemma}\label{prune}
	There is a polynomial time algorithm that produces sets $L_0,\ldots,L_t$ of pairs of $(x,y)\in \dom(P)$ together with multipliers $\lambda$ with the following properties for $i=0,\ldots,t$:
		(a) If $(x,y)\in L_i$, then $x_j \in \{0,1\}$ for $j=1,\ldots,i$, i.e. the first $i$ coordinates of a solution in level $i$ are integral, (b) $\sum_{[(x,y),\lambda]\in L_i} \lambda\geq\frac{1}{g^i}$, (c) $\sum_{[(x,y),\lambda]\in L_i}\lambda x \leq x^*$, (d) $|L_i|\leq t$.
\end{lemma}
\begin{proof}
	We prove this lemma using induction but one can clearly see how to turn this proof into a polynomial time algorithm. Let $L_0$ be the set that contains a single node (\textit{root of the FDT tree}) with $(x^*,y^*)$ and multiplier 1. It is easy to check all the requirements in the lemma are satisfied for this choice.
	
	Suppose by induction that we have constructed sets $L_0,\ldots,L_i$. Let the solutions in $L_i$ be $(x^j,y^j)$ for $j=1,\ldots,k$ and $\lambda_j$ be their multipliers, respectively. For each $j= 1,\ldots,k$ by Lemma \ref{round-up} (setting $(x',y')= (x^j,y^j)$ and $\ell = i+1$) we can find $(x^{j0},y^{j0}), (x^{j1},y^{j1})$ and $\lambda^0_j$, $\lambda^1_j$ with the properties (i) to (v) in Lemma \ref{round-up}. Define $L'$ to be the set of nodes with solutions $(x^{j0},y^{j0}), (x^{j1},y^{j1})$ and multipliers  $\lambda_j\lambda^0_j$, $\lambda_j\lambda^1_j$, respectively, for $j=1,\ldots,k$. It is easy to check that \iffalse{
	Notice that for each $j= 1,\ldots,k$ by property (v) in Lemma \ref{round-up} we have $x_h^{j0}, x_h^{j1}\in \{0,1\}$ for $h = 0,\ldots,i+1$. We also have
	\begin{align*}
		\sum_{[(x,y),\lambda]\in L'} \lambda & =  \sum_{j=1}^{k} \lambda_j\lambda^0_j+\lambda_j\lambda^1_j&\\
		& \geq \sum_{j=1}^{k} \frac{\lambda_j}{ g} & (\text{By property (i) in Lemma \ref{round-up}})\\
		& \geq \frac{1}{g} \sum_{[(x,y),\lambda]\in L_i}\lambda&\\
		& \geq \frac{1}{g^{i+1}}.&(\text{By induction hypothesis})	 
	\end{align*} 
	Also 
	\begin{align*}
		\sum_{[(x,y),\lambda]\in L'}\lambda x & =  \sum_{j=1}^{k} \lambda_j\big(\lambda^0_jx^{j0}+\lambda^1_jx^{j1}\big)&\\ 
		& \leq \sum_{j=1}^{k} \lambda_j x^j  & (\text{By property (iv) in Lemma \ref{round-up}})\\
		& \leq \sum_{[(x,y),\lambda]\in L_i}\lambda x&\\
		& \leq x^*.&(\text{By induction hypothesis})	 
	\end{align*} \fi set $L'$ is a suitable candidate for $L_{i+1}$, i.e. set $L'$ satisfies (a), (b) and (c). However we can only ensure that $|L'|\leq 2k\leq 2t$, and might have $|L'|>t$. We call the following linear program $\prun(L')$. Let $L' = \{[(x^1,y^1),\gamma_1],\ldots,[(x^{2k},y^{2k}),\gamma_{2k}]\}$. The variables of $\prun(L')$ is a scalar variable $\theta_j$ for each node $j$ in $L'$.  
	\begin{align}
		\prun(L')\quad\quad& \max\quad \;\sum_{j=1}^{2k} \theta_j\\
		&\;\text{s.t.} \quad \;\;\sum_{j=1}^{2k} \theta_j x^j_i\leq x^*_i &\mbox{ for $i=1,\ldots,t$} \label{packs}\\
		&\;{\color{white}{\text{s.t.}} }\quad  \;\quad\quad\theta \geq 0\label{nonneg}
	\end{align}

	Notice that $\theta = \gamma$ is in fact a feasible solution to $\prun(L')$. Let $\theta^*$ be the optimal vertex point solution to this LP. Since the problem is in $\bbbr^{2k}$,  $\theta^*$ has to satisfy $2k$ linearly independent constraints at equality. However, there are only $t$ constraints of type (\ref{packs}). Therefore, there are at most $t$ coordinates of $\theta^*_j$ that are non-zero. We claim that $L_{i+1}$ which consists of $(x^j,y^j)$ for $j=1,\ldots,2k$ and their corresponding multipliers $\theta^*_j$ satisfy the properties in the statement of the lemma. Notice that, we can discard the nodes in $L_{i+1}$ that have $\theta^*_j=0$, so $|L_{i+1}| \leq t$. Also, since $\theta^*$ is optimal and $\gamma$ is feasible for $\prun(L')$, we have $\sum_{j=1}^{k} \theta^*_j \geq \sum_{j=1}^{2k}\gamma_j \geq \frac{1}{g^{i+1}}$. 
\end{proof}
\paragraph{From leaves of FDT to feasible solutions}
The leaves of the FDT tree,  $L_t$, have the property that every solution $(x,y)$ in $L_t$ has $x\in\{0,1\}^n$ and $(x,y)\in \dom(P)$. By applying Lemma \ref{domlemma} we can obtain a point $(x',y')\in S$ such that $x'\leq x$. This conclude the description of the FDT algorithm and proves Theorem \ref{binaryFDT}.


\section{FDT for 2EC}\label{2EC}

In Section \ref{binaryfdt} our focus was on binary MIPs. In this section, in an attempt to extend FDT to 0,1,2 problems we introduce an FDT algorithm for a 2-edge-connected multigraph problem. Given a graph $G=(V,E)$ a multi-subset of edges $F$ of $G$ is a 2-edge-connected multigraph of $G$ if for each set $\emptyset\subset U \subset V$, the number of edge in $F$ that have one endpoint in $U$ and one not in $U$ is at least 2. In the 2EC problem, we are given non-negative costs on the edge of $G$ and the goal is to find the minimum cost 2-edge-connected multigraph of $G$. Notice that, no optimal solution ever takes 3 copies of an edge in 2EC, hence we assume that we can take an edge at most 2 times. The natural linear programming relaxation is $\EC(G) = \{x\in [0,2]^E: x(\delta(U))\geq 2 \mbox{ for } \emptyset \subset U \subset V\}$. Notice that $\dom(\EC(G))\cap [0,2]^{E} = \EC(G)$, since 2EC is a covering problem. We want to prove the following theorem. 

\FDTEC*

We do not know the exact value for $g_{\EC}$, but we know $\frac{6}{5} \leq g_{\EC} \leq \frac{3}{2}$ \cite{carr-ravi,Wolsey1980}. Also, we need to remark that polyhedral version of Christofides' algorithm provides a $\frac{3}{2}$-approximation for 2EC, i.e. we already have an algorithm with $C\leq \frac{3}{2}$. However, we show later in Section \ref{experiment} that in practice the constant $C$ for the FDT algorithm for 2EC is much better than $\frac{3}{2}$. 

The FDT algorithm for 2EC is very similar to the one for binary MIPs, but there are some differences as well. First, we need a branching lemma. Observe that  the following branching lemma is essentially a translation of Lemma \ref{LPClemma} for 0,1,2 problems expect for one additional clause. 


\begin{restatable}{lemma}{2ECLPC}
	\label{LPC2EC}
		Given $x\in \EC(G)$, and $e\in E$ we can find in polynomial time vectors $x^0,x^1$ and $x^2$ and scalars $\gamma_0,\gamma_1$, and $\gamma_2$ such that
	\begin{itemize}
		\item[(i)] $\gamma_0 + \gamma_1 +\gamma_2 \geq \frac{ 1}{g_{\EC}}$,
		\item[(ii)] $x^0,x^1,$ and $x^2$ are in  $ \EC(G)$, 
		\item[(iii)] $x^0_e=0$, $x^1_e=1$, and $x^2_e=2$,
		\item[(iv)] $\gamma_0 x^0 + \gamma_1{x}^1  + \gamma_2x^2\leq {x}$,
		\item[(v)] for $f\in E$ with ${x}_f\geq 1$, we have $x^j_f\geq 1$ for $j=0,1,2$. 
	\end{itemize}
\end{restatable}

\begin{proof}
	Consider the following linear programing with variables $\lambda_j$ and $x^j$ for $j=0,1,2$. 
	\begin{align}
	\quad\quad& \max\quad \;\sum_{j=0,1,2}\lambda_j\\
	&\;\text{s.t.} \quad x^j(\delta(U))\geq 2\lambda_j \;& \mbox{ for $\emptyset \subset U \subset V$, and $j=0,1,2$} \label{feasibility2ec}\\
	&\;{\color{white}{\text{s.t.}} }\quad 0 \leq x^j \leq 2\lambda_j\; &\mbox{ for $j=0,1,2$}\label{bound2ec}\\
	&\;{\color{white}{\text{s.t.}} }\quad x^j_e = j\; &\mbox{ for $j=0,1,2$}\label{branchcoordinate2ec}\\
	&\;{\color{white}{\text{s.t.}} }\quad x^j_f \geq j \; &\mbox{ for $f\in E$ where $x_f \geq 1$, and $j=0,1,2$}\label{1edges2ec}\\
	&\;{\color{white}{\text{s.t.}} }\quad x^0 + x^1+x^2 \leq x\label{packing2ec}\\
	&\;{\color{white}{\text{s.t.}} }\quad \lambda_0,\lambda_1,\lambda_2 \geq 0
	\end{align}
	
	Let $x^j$, $\gamma_j$ for $j=0,1,2$ be an optimal solution solution to the LP above. Let $\hat{x}^{j}=\frac{x^j}{\gamma_j}$ for $j=0,1,2$ where $\gamma_j>0$. If $\gamma_j=0$, let $\hat{x}^{j}=0$. Observe that  (ii), (iii), (iv), and (v) are satisfied with this choice. In order to show that (i) is also satisfied we prove the following claim.
	
	
	
	\begin{claim}\label{CVexists}
		We have $\gamma_0 + \gamma_1+\gamma_2\geq \frac{1}{g_{\EC}}$.
	\end{claim}
	\begin{proof}
		Suppose for contradiction $\sum_{j=0,1,2}\gamma_j = \frac{1}{g_{\EC}} - \epsilon$ for some $\epsilon >0$. 
		
		Construct graph $G'$ by removing edge $f$ with $x_f\geq 1$ and replacing it with a path $P_f$ of length $\ceil{\frac{2}{\epsilon}}$. Define $x'_h = x_h$ for each edge $h$ such that $x_h<1$. For each $h\in P_f$ let $x'_h= x_f$ for all $f$ with $x_f\geq 1$. It is easy to check that $x'\in \EC(G')$. By Theorem \ref{CV} there exists $\theta \in [0,1]^k$, with $\sum_{i=1}^{k}\theta_i = 1$ and 2-edge-connected multigraphs $F'_i$ of $G'$ for $i=1,\ldots,k$ such that 
		$\sum_{i=1}^{k}\theta_i \chi^{F'_i}\leq g_{\EC}x'$. 
		
		Note that each $F'_i$ contains at least one copy of every edge in any path $P_f$, except for at most one edge in the path. We will obtain 2-edge-connected multigraphs $F_1,\ldots,F_k$ of $G$ using $F'_1,\ldots,F'_k$, respectively. To obtain $F_i$ first remove all $P_f$ paths from $F'_i$. Suppose there is an edge $h$ in $P_f$ such that $\chi^{F'_i}_h=0$, this means that for any edge $p\in P_f$ such that $p\neq h$, $\chi^{F'_i}_p=2$. In this case, let $\chi^{F_i}_f=2$, i.e. add two copies of $f$ to $F_i$. If there is an edge $h\in P_f$ with $\chi^{F'_i}_h = 1$, let $\chi^{F_i}_f=1$, i.e. add one copy of $f$ to $F_i$. If for all edge $h\in P_f$, we have $\chi^{F'_i}_h=2$, then let $\chi^{F_i}_f=2$. For $f\in E$ with $x_f<1$ we have
		\begin{equation}
		\sum_{i=1}^{k}\theta_i \chi^{F_i}_f=\sum_{i=1}^{k}\theta_i \chi^{F'_i}_f\leq g_{\EC}x'_f= g_{\EC}x_f.
		\end{equation}
		In addition for $f\in E$ with $x_f\geq 1$ we have $\chi^{F_i}_f \leq \frac{\sum_{h\in P_f}\chi^{F'_i}_h}{\ceil{\frac{2}{\epsilon}}-1}$ by construction.
		\begin{align*}
		\sum_{i=1}^{k}\theta_i \chi^{F_i}_f&\leq \sum_{i=1}^{k}\theta_i\frac{\sum_{h\in P_f}\chi^{F'_i}_h}{\ceil{\frac{2}{\epsilon}}-1}\\
		&= \frac{\sum_{h\in P_f} \sum_{i=1}^{k}\theta_i\chi^{F'_i}_h}{\ceil{\frac{2}{\epsilon}}-1}\\
		&\leq \frac{\sum_{h\in P_f} g_{\EC}x'_h}{\ceil{\frac{2}{\epsilon}}-1}\\
		&= \frac{\sum_{h\in P_f} g_{\EC}x'_h}{\ceil{\frac{2}{\epsilon}}-1}\\
		&= \frac{\ceil{\frac{2}{\epsilon}}}{\ceil{\frac{2}{\epsilon}}-1}g_{\EC}x_f
		\end{align*}
		Therefore, we have 
		\begin{equation}
		x' \geq \sum_{i\in [k]: \chi^{F_i}_e=1}\frac{\theta_i(\ceil{\frac{2}{\epsilon}}-1)}{g_{\EC}\ceil{\frac{2}{\epsilon}}}\chi^{F_i}+\sum_{i\in [k]: \chi^{F_i}_e=2}\frac{\theta_i(\ceil{\frac{2}{\epsilon}}-1)}{g_{\EC}\ceil{\frac{2}{\epsilon}}}\chi^{F_i}.
		\end{equation}
		Let $x^j = \sum_{i\in [k]: \chi^{F_i}_e=j}\frac{\theta_i(\ceil{\frac{2}{\epsilon}}-1)}{g_{\EC}\ceil{\frac{2}{\epsilon}}}\chi^{F_i}$ and $\theta_j =  \sum_{i\in [k]: \chi^{F_i}_e=j}\frac{\theta_i(\ceil{\frac{2}{\epsilon}}-1)}{g_{\EC}\ceil{\frac{2}{\epsilon}}}$ for $j=0,1,2$. It is easy to check that $x^j$ , $\theta_j$ for $j=0,1,2$ is a feasible solution to the LP above. Notice that $\sum_{j=0,1,2}\theta_j = \frac{\ceil{\frac{2}{\epsilon}}-1}{g_{\EC}\ceil{\frac{2}{\epsilon}}}$. By assumption, we have $\frac{\ceil{\frac{2}{\epsilon}}-1}{g_{\EC}\ceil{\frac{2}{\epsilon}}}\leq  \frac{1}{g_{\EC}}-\epsilon$. However, this means $2\leq 1$ which is a contradiction.
	\end{proof}
	This concludes the proof.
\end{proof}


In contrast to FDT for binary MIPs where we round up the fractional variables that are already branched on at each level, in FDT for 2EC we keep all coordinates as they are and preform a rounding procedure at the end. Formally, let $L_i$ for $i=1,\ldots,|\sup(x^*)|$ be collections of pairs of feasible points in $\EC(G)$ together with their multipliers. Let $t=|\sup(x^*)|$ and assume without loss of generality that $\sup(x^*)=\{e_1,\ldots,e_t\}$. 

\begin{lemma}\label{2ecpruning}
	The FDT algorithm for 2EC in  polynomial time produces sets $L_0,\ldots,L_t$ of pairs $x\in \EC(G)$ together with multipliers $\lambda$ with the following properties.
	\begin{enumerate}
			\item[a.] If $x\in L_i$, then $x_{e_j}=0$ or $x_{e_j}\geq 1$ for $j=1,\ldots,i$,
		\item [b.] $\sum_{(x,\lambda)\in L_i }\lambda \geq \frac{1}{g_{\EC}^i}$,
		\item[c.]  $\sum_{(x,\lambda)\in L_i }\lambda x \leq x^*$,
		\item[d.] $|L_i|\leq t$.
		\end{enumerate}
\end{lemma}
\begin{proof}
	We proceed by induction on the $i$. Define $L_0=\{(x^*,1)\}$. It is easy to check all the properties are satisfied. Now, suppose by induction we have $L_{i-1}$ for some $i=1,\ldots,t$ that satisfies all the properties. For each solution $x^\ell$ in $L_{i-1}$ apply Lemma \ref{LPC2EC} on $x^\ell$ and $e_{i}$ to obtain $x^{\ell j}$ and $\lambda_{\ell j}$ for $j=0,1,2$. Let $L'$ be the collection that contains $(x^{\ell j},\lambda_\ell \cdot \lambda_{\ell j})$ for $j=0,1,2$, when applied to all $(x^\ell,\lambda_\ell)$ in $L_{i-1}$. Similar to the proof in Lemma \ref{prune} one can check that $L_i$ satisfies properties (b), (c). We now verify property (a). Consider a solution $x^\ell$ in $L_{i-1}$. For $e\in \{e_1,\ldots,e_{i-1}\}$ if $x^\ell_e =0$, then by property (iv) in Lemma \ref{LPC2EC} we have $x^{\ell j}=0$ for $j=0,1,2$. Otherwise by induction we have $x^{\ell}_{e}\geq 1$ in which case property (v) in Lemma \ref{LPC2EC} ensures that $x^{\ell j}_e\geq 1$ for $j=0,1,2$. Also, $x^{\ell j}_{e_i}= j$, so $x^{\ell j}_{e_i}=0$ or $x^{\ell j}_{e_i}\geq 1$ for $j=0,1,2$. 
	
	If $|L'|\geq t$ we let $L_i=L'$, otherwise apply $\prun(L')$ to obtain $L_{i}$.
\end{proof}	

Consider the solutions $x$ in $L_t$. For each variable $e$ we have $x_e=0$ or $x_e\geq 1$. 
\begin{lemma}\label{rounddown}
	Let $x$ be a solution in $L_t$. Then $\floor{x} \in \EC(G)$. 
\end{lemma}
\begin{proof}
	Suppose not. Then there is a set of vertices $\emptyset \subset U \subset V$ such that $\sum_{e\in \delta(U)}\floor{x_e}<2$. Since $x\in \EC(G)$ we have $\sum_{e\in \delta(U)}x_e \geq 2$. Therefore, there is an edge $f\in \delta(U)$ such that $x_f$ is fractional. By property (a) in Lemma \ref{2ecpruning}, we have $1<  x_f < 2$. Therefore, there is another edge $h$ in $\delta(U)$ such that $x_h>0$, which implies that $x_h\geq 1$. But in this case $\sum_{e\in \delta(U)}\floor{x_e}\geq  \floor{x_f}+\floor{x_h}  \geq 2$. This is a contradiction.
\end{proof}

The FDT algorithm for 2EC iteratively applies Lemmas \ref{LPC2EC} and \ref{2ecpruning} to variables $x_1,\ldots,x_t$ to obtain leaf point solutions $L_t$. Then, we just need to apply Lemma \ref{rounddown} to obtain the 2-edge-connected multigraphs from every solution in $L_t$. Notice that $t\leq 2|V|-3$ \cite{}. By Lemma \ref{2ecpruning} we have
\begin{align*}
\sum_{(x,\lambda)\in L_t} \frac{\lambda}{\sum_{(x,\lambda)\in L_t}\lambda} \floor{x} \leq \frac{1}{\sum_{(x,\lambda)\in L_t}\lambda} \sum_{(x,\lambda)\in L_t} \lambda {x} \leq g^t_{\EC} x^*.
\end{align*}












\section{Computational experiments with FDT}\label{experiment}
We implement FDT to two covering problem. First, the tree argumentation problem (TAP) where given a tree $T=(V,E)$, and a set of link $L$ between the vertices in $V$ and costs $c\in \bbbr^{L}_{\geq 0}$ we with to find minimum cost subset $L'$ of $L$ such that $T+L'$ is 2-edge-connected. For $\ell\in L$, let $P_\ell$ be the set of edges in the unique path between the endpoints of $\ell$ in $T$. For TAP, $S(\mbox{TAP})=\{x\in \bbbz^{L}_{\geq 0}: \sum_{\ell: e\in P_\ell}x_\ell \geq 1, \mbox{ for $e\in E$}\}$. Relaxing the integrality constraint we get $P(\mbox{TAP})$. It is been shown that $\frac{3}{2}\leq g({\mbox{TAP}})\leq 2$ \cite{fj,32gap}. We applied the binary FDT algorithm on a set of 264 fractional extreme points of $P(\mbox{TAP})$. The result are summarized in Table \ref{tableTAP}. 
\begin{table}[h]
	\centering
	  \begin{tabular}{c c c c c}
	  \toprule
	  	& $C\in [1.1,1.2]\;$ & $\;C\in (1.2,1.3]\;$ &
               $\;C\in (1.3,1.4]$ &\; $C\in (1.4,1.5]\;$ \\ \midrule
	  	TAP & $36$ & $66$ & $170$ & $10$\\  \bottomrule \\
	  \end{tabular}\caption{FDT impletemented applied to 264 randomly generated TAP instances with fractional extreme points: 138 of the 264 have $74$ variables, so the theoretical guarantee of Theorem \ref{binaryFDT} is at least $(1.5)^{74}$. For the rest, the number of variables in $250$.}
	  \label{tableTAP}
\end{table}
Next we implemented the FDT for 2EC on a 963 fractional extreme points of $\EC(G)$. These points are obtained by considering all fundamental vertices with $10$ and $12$ vertices (See \cite{CV} for the definition of fundamental vertices). The results are summarized in Table \ref{table2EC}. 
\begin{table}[h]
	\centering
	  \begin{tabular}{c c c c c}
	  	\toprule
	  	& $C\in [1.08,1.11]\;$ & $\;C\in (1.11,1.14]\;$ &
               $\;C\in (1.14,1.17]$ &\; $C\in (1.17,1.2]\;$ \\ \midrule
	  	2EC & $79$ & $201$ & $605$ & $78$ \\ \bottomrule\\
	  \end{tabular}\caption{FDT for $\EC$ implemented applied to all fundamental extreme points with 10 and 12 vertices. The number of variables for a fundamental extreme point with $k$ vertices is $\frac{3k}{2}$ and the lower bound on $g_\EC$ is $\frac{6}{5}$.}
	  \label{table2EC}
\end{table}
We also implemented polyhedral version of Christofides' algorithm \cite{Wolsey1980} and compared its performance on fundamental extreme points with 10 vertices. The result are in Figure \ref{fdtvschris}.
\begin{figure*}%
\centering
\includegraphics[width=8cm]{"fdt vs christofides".png}}
\caption{Christofides' algorithm vs FDT on all fundamental extreme points with 10 vertices.}
\label{fdtvschris}
\end{figure*}
\iffalse{
\input{strongerCV}
}\fi





%
% ---- Bibliography ----
%
% BibTeX users should specify bibliography style 'splncs04'.
% References will then be sorted and formatted in the correct style.
%
% \bibliographystyle{splncs04}
% \bibliography{mybibliography}
%
\begin{thebibliography}{10}
	
	
	
	\bibitem{carr-ravi}
	Robert Carr and R.~Ravi.
	\newblock A new bound for the 2-edge connected subgraph problem.
	\newblock In {\em Proceedings of the 6th International IPCO Conference on
		Integer Programming and Combinatorial Optimization}, pages 112--125, London,
	UK, UK, 1998. Springer-Verlag.
	
	\bibitem{CV}
	Robert Carr and Santosh Vempala.
	\newblock On the {H}eld-{K}arp relaxation for the asymmetric and symmetric
	traveling salesman problems.
	\newblock {\em Mathematical Programming}, 100(3):569--587, Jul 2004.
	
		\bibitem{fj}
	Greg N. Frederickson and Joseph Ja'Ja'.
	\newblock Approximation algorithms for several graph
	augmentation problems.
	\newblock {\em SIAM Journal on Computing}, 10(2):270--283, 1981.
	
		\bibitem{32gap}
	Cheriyan, Joseph and Karloff, Howard and Khandekar, Rohit and K{\"o}nemann, Jochen.
	\newblock On the integrality ratio for tree augmentation.
	\newblock {\em Operations Research Letters}, 36(4):399--401, 2008.
    	
	\bibitem{abe}
    Anthony Alexander, Sylvia Boyd, and Paul Elliott-Magwood.
    \newblock {On the Integrality Gap of the 2-Edge Connected Subgraph Problem}.
    \newblock Technical report, University of Ottawa, Ottawa, Canada, 04 2006.
    
    \bibitem{TSPcompute}
    Geneviève Benoit and Sylvia Boyd.
    \newblock Finding the exact integrality gap for small traveling salesman
      problems.
    \newblock {\em Mathematics of Operations Research}, 33(4):921--931, 2008.

	\bibitem{gerard}
	G{\'e}rard Cornu{\'e}jols, Jean Fonlupt, and Denis Naddef.
	\newblock The traveling salesman problem on a graph and some related integer
	polyhedra.
	\newblock {\em Mathematical Programming}, 33(1):1--27, Sep 1985.
	
	\bibitem{fp1}
	Matteo Fischetti, Fred Glover, and Andrea Lodi.
	\newblock The feasibility pump.
	\newblock {\em Mathematical Programming}, 104(1):91--104, Sep 2005.
	
	\bibitem{fp2}
	Matteo Fischetti and Domenico Salvagnin.
	\newblock Feasibility pump 2.0.
	\newblock {\em Mathematical Programming Computation}, 1(2):201--222, Oct 2009.
	
	\bibitem{GJ79}
	M.R. Garey and D.S. Johnson.
	\newblock {\em Computers and Intractability -- A Guide to the Theory of
		{NP}-Completeness}.
	\newblock Freeman, San Francisco, CA, 1979.
	
	\bibitem{goemans}
	Michel~X. Goemans.
	\newblock Worst-case comparison of valid inequalities for the {TSP}.
	\newblock {\em Math. Program.}, 69(2):335--349, August 1995.
	
	\bibitem{cons-cara}
	Martin Gr{\"o}tschel, L{\'a}szl{\'o} Lov{\'a}sz, and Alexander Schrijver.
	\newblock {\em Geometric Algorithms and Combinatorial Optimization}, volume~2.
	\newblock Second corrected edition edition, 1993.
	
	\bibitem{jain}
	Kamal Jain.
	\newblock A factor 2 approximation algorithm for the generalized steiner
	network problem.
	\newblock In {\em Proceedings of the 39th Annual Symposium on Foundations of
		Computer Science}, FOCS '98, pages 448--, Washington, DC, USA, 1998. IEEE
	Computer Society.
	
	\bibitem{randomizedrounding}
	Prabhakar Raghavan and Clark~D. Tompson.
	\newblock Randomized rounding: A technique for provably good algorithms and
	algorithmic proofs.
	\newblock {\em Combinatorica}, 7(4):365--374, Dec 1987.
	
	\bibitem{schrijver}
	A.~Schrijver.
	\newblock {\em Combinatorial Optimization - Polyhedra and Efficiency}.
	\newblock Springer, 2003.
	
	\bibitem{vazirani}
	Vijay~V. Vazirani.
	\newblock {\em Approximation Algorithms}.
	\newblock Springer-Verlag, Berlin, Heidelberg, 2001.
	
	\bibitem{sw}
	David~P. Williamson and David~B. Shmoys.
	\newblock {\em The Design of Approximation Algorithms}.
	\newblock Cambridge University Press, New York, NY, USA, 1st edition, 2011.
	
	\bibitem{Wolsey1980}
	Laurence~A. Wolsey.
	\newblock {\em Heuristic analysis, linear programming and branch and bound},
	pages 121--134.
	\newblock Springer Berlin Heidelberg, Berlin, Heidelberg, 1980.
	
\end{thebibliography}

\appendix
\section{Proof of Lemma 4}\label{lemma4}
	First, we need to show that in any iteration $\ell=  0,\ldots,t-1$ of Algorithm \ref{domtoIPalg} the $\DOMtoS(x^{\ell})$ is feasible. We show something stronger. For $\ell=0,\ldots,t-1$ let
	\begin{align*}
	\LP^{(\ell)}&= \{(z,w)\in P\; : \; z\leq x^{(\ell)} \mbox{ and } z_j=x_j^{(\ell)} \mbox{ for } j=1,\ldots,\ell\}, \text{ and}\\
	\IP^{(\ell)}&= \{(z,w)\in \LP^{(\ell)}\; : \; z\in \{0,1\}^n\}.
	\end{align*}
	Notice that if $\LP^{(\ell)}$ is a non-empty set then $\DOMtoS(x^{(\ell)})$ is feasible. We show by induction on $\ell$ that $\LP^{(\ell)}$ and $\IP^{(\ell)}$ are not empty sets for $\ell=0,\ldots,t-1$. First notice that $\LP^{(0)}$ is clearly feasible since by definition $(x^{(0)},y^{(0)})\in \dom(P)$, meaning there exists $(z,w)\in P$ such that $z\leq x^{(0)}$. By Theorem \ref{CV}, there exists $(\tilde{z}^i,\tilde{w}^i) \in S$ and $\theta_i\geq 0$ for $i=1,\ldots,k$ such that $\sum_{i=1}^{k} \theta_i = 1$ and $\sum_{i=1}^{k}\theta_i \tilde{z}^i \leq gz$. Hence, $\sum_{i=1}^{k}\theta_i \tilde{z}^i \leq gz\leq gx^{(0)}$. So if $x^{(0)}_j=0$, then $ \sum_{i=1}^{k}\theta_i \tilde{z}_j^i =0$, which implies that $\tilde{z}^i_j=0$ for all $i=1,\ldots,k$ and $j= 1,\ldots,n$ where $x^{(0)}_j=0$. Hence, $z^i\leq x^{(0)}$ for $i=1,\ldots,k$. Therefore $(\tilde{z}^i,\tilde{w}^i)\in \IP^{(0)}$ for $i=1,\ldots,k$, which implies $\IP^{(0)}\neq \emptyset$.
	
	Now assume $\IP^{(\ell)}$ is non-empty for some $\ell \in \{0,\ldots,t-2\}$. Since $\IP^{(\ell)}\subseteq\LP^{(\ell)}$ we have $\LP^{(\ell)}\neq \emptyset$ and hence the $\DOMtoS(x^{(\ell)})$ has an optimal solution $(z^*,w^*)$. 
	
	We consider two cases. In the first case, we have $z^*_{\ell+1}=0$. In this case we have $x^{(\ell+1)}_{\ell+1}=0$. Since $z^*\leq x^{(\ell+1)}$, we have $(z^*,w^*)\in \LP^{(\ell+1)}$. Also, $(z^*,w^*)\in P$. By Theorem \ref{CV} there exists $(\tilde{z}^i,\tilde{w}^i)\in S$ and $\theta_i\geq 0$ for $i=1,\ldots,k$ such that $\sum_{i=1}^{k} \theta_i = 1$ and  $\sum_{i=1}^{k}\theta_i \tilde{z}^i \leq gz^*$ . We have
	\begin{equation}
	\sum_{i=1}^{k}\theta_i \tilde{z}^i \leq gz^*\leq gx^{(\ell+1)}
	\end{equation}
	So for $j\in \{1,\ldots,n\}$ where $x^{(\ell+1)}_j=0$, we have $z^i_j=0$ for $i=1,\ldots,k$. Hence, $\tilde{z}^i\leq x^{(\ell+1)}$ for $i=1,\ldots,k$. Hence, there exists $(z,w)\in S$ such that $z\leq x^{(\ell+1)}$. We claim that $(z,w)\in \IP^{(\ell+1)}$. If $(z,w)\notin \IP^{(\ell+1)}$ we must have $1\leq j \leq \ell$ such that $z_j < x^{(\ell+1)}_{j}$, and thus $z_j = 0$ and $x^{(\ell+1)}_j=1$. Without loss of generality assume $j$ is minimum number satisfying $z_j < x^{(\ell+1)}_{j}$. Consider iteration $j$ of Algorithm \ref{domtoIPalg}. Notice that $z\leq x^{(\ell+1)}\leq x^{(j)}$. We have $x^{(j)}_j=1$ which implies when we solved $\DOMtoS(x^{(j-1)})$ the optimal value was strictly larger than zero. However, $(z,w)$ is a feasible solution to $\DOMtoS(x^{(j-1)})$ and gives an objective value of 0. This is a contradiction, so $(z,w)\in \IP^{(\ell+1)}$.
	
	Now for the second case, assume $z^*_{\ell+1} > 0$. We have $x^{(\ell+1)}_{\ell+1}=1$. Notice that for each point $z\in \LP^{(\ell)}$ we have $z_{\ell+1} >0$, so for each $z\in \IP^{(\ell)}$ we have $z_{\ell+1}>0$, i.e. $z_{\ell+1}=1$. This means that $(z,w)\in \IP^{(\ell+1)}$, and $\IP^{(\ell+1)} \neq \emptyset$.
	
	Now consider $(x^{(t)},y^{(t)})$. Let $(z,y^{(t)})$ be the optimal solution to $\LP^{(t-1)}$. If $x^{(t)} = 0$, we have $x^{(t)} = z$, which implies that $(x^{(t)},y^{(t)})\in P$, and since $x^{(t)}\in \{0,1\}^n$ we have $(x^{(t)},y^{(t)})\in S$. If $x^{(t)} =1$, it must be the case that $z_t > 0$. By the argument above there is a point $(z',w')\in \IP^{(t-1)}$. We show that $x^{(t)} = z'$. Observe that for $j=1,\ldots,n-1$ we have $z'_j= x_j^{(t-1)}=x_j^{(t)}$. We just need to show that $z'_j = 1$. Assume $z'_j = 0$ for contradiction, then $(z',w')\in \LP^{(t-1)}$ has objective value of $0$ for $\DOMtoS(x^{(t-1)})$, this is a contradiction to $(z,w)$ being the optimal solution.
\end{document}
