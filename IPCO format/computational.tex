\section{Computational experiments with FDT}\label{experiment}
We ran FDT on two covering problem: TAP and 2EC. For the tree augmentation problem (TAP) we are given a tree $T=(V,E)$, and a set of non-edge links $L$ between the vertices in $V$ and costs $c\in \bbbr^{L}_{\geq 0}$. We wish to find the minimum-cost subset $L'$ of $L$ such that $T+L'$ is 2-edge-connected. For $\ell\in L$, let $P_\ell$ be the set of edges in the unique path between the endpoints of $\ell$ in $T$. For TAP, $S(\mbox{TAP})=\{x\in \bbbz^{L}_{\geq 0}: \sum_{\ell: e\in P_\ell}x_\ell \geq 1, \mbox{ for $e\in E$}\}$. Relaxing the integrality constraint we get $P(\mbox{TAP})$. We know $\frac{3}{2}\leq g({\mbox{TAP}})\leq 2$~\cite{fj,32gap}. We applied the binary FDT algorithm on a set of 264 fractional extreme points of $P(\mbox{TAP})$. The result are summarized in Table \ref{tableTAP}. FDT found solutions better than the integrality-gap lower bound for most instances.
\begin{table}[h]
	\centering
	  \begin{tabular}{c c c c c}
	  \toprule
	  	& $C\in [1.1,1.2]\;$ & $\;C\in (1.2,1.3]\;$ &
               $\;C\in (1.3,1.4]$ &\; $C\in (1.4,1.5]\;$ \\ \midrule
	  	TAP & $36$ & $66$ & $170$ & $10$\\  \bottomrule \\
	  \end{tabular}\caption{The scale factor $C$ for FDT when run on 264 randomly generated tree-augmentation (TAP) instances with fractional extreme points: 138 of the 264 have $74$ variables. The rest have $250$ variables.}
	  \label{tableTAP}
\end{table}
%\vspace*{-30pt}
We ran FDT for 2EC on 963 fractional extreme points of $\EC(G)$. We enumerated all fundamental vertices of order $10$ and $12$. Table \ref{table2EC} shows that again FDT found solutions better than the integrality-gap lower bound for most instances. 
\begin{table}[h]
	\centering
	  \begin{tabular}{c c c c c}
	  	\toprule
	  	& $C\in [1.08,1.11]\;$ & $\;C\in (1.11,1.14]\;$ &
               $\;C\in (1.14,1.17]$ &\; $C\in (1.17,1.2]\;$ \\ \midrule
	  	2EC & $79$ & $201$ & $605$ & $78$ \\ \bottomrule\\
	  \end{tabular}	\caption{FDT for $\EC$ implemented applied to all fundamental extreme points of order 10 or 12. A fundamental extreme point or order $k$ has $\frac{3k}{2}$ variables. The lower bound on $g(\EC)$ is $\frac{6}{5}$.}
	  \label{table2EC}
\end{table}
We also implemented the polyhedral version of Christofides' algorithm \cite{Wolsey1980}. Figure \ref{fdtvschris} shows FDT's solutions on fundamental extreme points of order 10 are always better than those from Christofides' algorithm.
%\vspace*{-.5in}
\begin{figure}[h!]
\centering
\includegraphics[width=9cm]{"fdt vs christofides".png}
\caption{Christofides' algorithm vs FDT on all fundamental extreme points of order 10.}
\label{fdtvschris}
\end{figure}
